\documentclass[10pt]{article}
\usepackage[utf8]{inputenc}
\usepackage{amscd}
\usepackage{amsmath}
\usepackage{amssymb}
\usepackage{amsthm}
\usepackage{listings}
\usepackage{enumerate}

\textwidth=15cm \textheight=22cm \topmargin=0.5cm \oddsidemargin=0.5cm \evensidemargin=0.5cm

\newcommand{\sk}{\vskip 10mm}
\newcommand{\bb}[1]{\mathbb{#1}}
\newcommand{\ra}{\rightarrow}

\theoremstyle{plain}
\newtheorem{problem}{Problem}
\newtheorem{lemma}{Lemma}[problem]

\theoremstyle{remark}
\newtheorem{tpart}{}[problem]
\newtheorem*{ppart}{}

\begin{document}

\begin{problem}[2.7.1]
  
\end{problem}

\begin{proof}
  First we will show that $N$ is normal. Then given $g\in N$ for all $x\in G$ there
  exists an $h\in H$ such that $g=xhx^{-1}$. Let $k\in G$ then $kgk^{-1}=h\in H$ as
  $g\in k^{-1}Hk$. However $g \in k^{-1}xHx^{-1}k$ for any $x\in G$. It then follows that
  \[ h=kgk^{-1}=kk^{-1}xh'x^{-1}kk^{-1}=xh'x\]
  which implies that $h\in xHx^{-1}$ for all $x\in G$ and therefore $h\in N$.
  Therefore $N$ is a normal subgroup of $G$.

  Now we will show that $N$ is the largest normal subgroup of $G$ contained in $H$.
  Let $M$ be a subgroup of $G$ such that $M\trianglelefteq G$ and $M\leq H$. Then
  given $g\in M$ and $x\in G$ we have $x^{-1}gx=h\in M$. However this implies that
  $g=xhx^{-1}$ for all $x$ and since $M\leq H$ we have that $h\in H$ and therefore
  $g\in N$ and $M\leq N$.

  Therefore since $N$ is normal and normal subgroup contained in $H$ is contained
  in $N$ we have that $N$ is the largest normal subgroup contained in $H$.
\end{proof}

\sk

\begin{problem}[2.7.2]
  
\end{problem}

\begin{proof}
  \begin{itemize}
  \item[a)] For reflexivity, since $H,K\leq G$ we have $exe=x\in HxK$.

    For symmetry if $x\sim y$ then $x\in HyK$ which implies that $x=hyk$ for some $h\in H$ and
    $k\in K$. However since $H,G\leq G$ we have that $h^{-1}xk^{-1}=y$ which implies that
    $y\in HxK$ and therefore $y\sim x$.

    For transitivity suppose that we have $x\sim y$ and $y\sim z$. Then as before we have
    $x=hyk$ and $y=h'zk'$. It then follows that $x=hh'zk'k$ which implies that $x\in HzK$
    and therefore $x\sim z$.

    Let $\bar{x}$ denote the equivalence class of $x$. Then if $y\in HxK$ by definition
    $y\sim x$ and $y\in X$. Otherwise if $y\in \bar{x}$ by definition $y\sim x$ which implies that
    $y\in HxK$.

    Therefore $\sim$ is an equivalence relation, the equivalence classes are
    of the form $HxK$, and as such $H\setminus G/ K$ forms a partition of $G$.
  \item[b)]
  \item[c)]
  \end{itemize}
\end{proof}

\sk

\begin{problem}[2.8]
  
\end{problem}

\begin{proof}
  \begin{enumerate}
  \item We'll start by showing that $C_G(A)$ is a subgroup. If we have $g,h\in C_G(A)$
    then $gha=gah=agh$ so it is closed under the group operation. Then if $g\in C_G(A)$
    we have $ga=ag$. Multiply on the left by and right by $g^{-1}$ and we get
    $ag^{-1}=g^{-1}a$. Therefore $C_G(A)$ is a subgroup.

    Next consider $N_G((A))$. If we have $g,h\in N_G(A)$ then $hah^{-1}=a'\in A$. This
    implies that $ga'g^{-1}\in A$ and therefore $ghah^{-1}g^{-1}\in A$. Next let $g\in N_G(A)$
    and $a\in A$. Then \textbf{Closed under inverses}.

    Let $g\in C_G(A)$ then for $a\in A$ we have $gag^{-1}=agg^{-1}=a$ which implies that
    $g\in N_G(A)$. Therefore $C_G(A)\subset N_G(A)$.
  \item Let $a\in A$ and $n\in N_G(A)$. Then $nan^{-1}\in A$ be definition of $N_G(A)$. Therefore
    if $A\leq G$ then $A\trianglelefteq N_G(A)$.
  \item Let $z\in Z(G)$ and let $g\in G$. Then $zgz^{-1}=gzz^{-1}=g$ which implies that
    $Z(G)$ is a normal subgroup of $G$.
  \item For the group $S_3$ take the subgroup $\left<(1\ 2)\right>$. Then
    $(1\ 2\ 3)(1\ 2)(3\ 2\ 1)=(2\ 3)\notin \left<(1\ 2)\right>$ which implies that
    $\left<(1\ 2)\right>$ is not normal.
  \item If $n$ is even then the center consists of $r^{n/2}$ and the identity. Otherwise
    if $n$ is odd then the center is just the identity.
  \item The subgroups of $\mathcal{Q}_8$ are $<1>,<-1>,<i>,<j>,<k>,\mathcal{Q}_8$. The
    trivial group and $\mathcal{Q}_8$ are both normal. For the others we have
    \begin{itemize}
    \item[$<-1>$]
      \[
        \begin{array}{c|c}
          g & g(-1)g^{-1}\\
          \hline 
          1 & -1\\
          -1 & -1\\
          i & -1\\
          -i & -1\\
          j & -1\\
          -j & -1\\
          k & -1\\
          -k & -1\\
        \end{array}
      \]
    \item[$<i>$]
      \[
        \begin{array}{c|c}
          g & g(i)g\\
          \hline 
          1 & i\\
          -1 & i\\
          i & -i\\
          -i & i\\
          j & -i\\
          -j & -i\\
          k & -i\\
          -k & \\
        \end{array}
      \]
    \item[$<j>$]
      \[
        \begin{array}{c|c}
          g & g(j)g\\
          \hline 
          1 & j\\
          -1 & \\
          i & \\
          -i & \\
          j & \\
          -j & \\
          k & \\
          -k & \\
        \end{array}
      \]
    \item[$<k>$]
      \[
        \begin{array}{c|c}
          g & g(k)g\\
          \hline 
          1 & k\\
          -1 & \\
          i & \\
          -i & \\
          j & \\
          -j & \\
          k & \\
          -k & \\
        \end{array}
      \]
    \end{itemize}
  \end{enumerate}
\end{proof}

\sk

\begin{problem}[2.9.1]
  
\end{problem}

\begin{proof}
  Since $|G|=p$ where $p$ is prime by Lagrange's Theorem (2.12 in notes)
  the order of any subgroup must be either $1$ or $p$. However the
  only subgroups that fulfill these criterion are either the trivial
  group or $G$ itself. Therefore $G$ cannot have any non-trivial subgroups.

  Let $g\in G\setminus \{e\}$. Since there is no non-trivial subgroup the element
  $\left<g\right>$ must generate the whole group. Then by definition
  $G$ is cyclic.
\end{proof}

\sk

\begin{problem}[2.9.2]
  
\end{problem}

\begin{proof}
  
\end{proof}

\sk

\begin{problem}[2.10]
  
\end{problem}

\begin{proof}
  
\end{proof}

%%%%%%%%%%%%%%%%%%%%%%%%%%%%%%%%%%%%%%%%%%%%%%%%%%%%%%%%%%%%%%%%%%%%%%%%%%%%%
\end{document}
