\documentclass[10pt]{article}
\usepackage[utf8]{inputenc}
\usepackage{amscd}
\usepackage{amsmath}
\usepackage{amssymb}
\usepackage{amsthm}
\usepackage{listings}
\usepackage{enumerate}

\textwidth=15cm \textheight=22cm \topmargin=0.5cm \oddsidemargin=0.5cm \evensidemargin=0.5cm

\newcommand{\sk}{\vskip 10mm}
\newcommand{\bb}[1]{\mathbb{#1}}
\newcommand{\ra}{\rightarrow}

\theoremstyle{plain}


\newtheorem{problem}{Problem}
\newtheorem{lemma}{Lemma}[problem]

\theoremstyle{remark}
\newtheorem{tpart}{}[problem]
\newtheorem*{ppart}{}

\begin{document}

\begin{problem}[7.14]
  \begin{enumerate}
  \item Let $R$ be a commutative ring with $1\neq 0$ and $I_1,\ldots,I_n$
    pairwise comaximal ideals of $R$. Prove that
    \[ (R/(I_1\ldots I_n))^X\cong (R/I_1)^X\times\ldots\times(R/I_n)^X \]
    as groups.
  \item Let $m,n$ be relatively prime positive integers. Prove that
    \[ (\bb{Z}_{mn})^X\cong (\bb{Z}_m)^X\times(\bb{Z}_n)^X \]
    as groups.
  \item Solve the system of congruences:
    \begin{align*}
      x &\equiv 2 \mod 9\\
      x &\equiv 3 \mod 5\\
      x &\equiv 1 \mod 7\\
      x &\equiv 5 \mod 11\\
    \end{align*}
  \end{enumerate}
\end{problem}

\begin{proof}
  \begin{enumerate}
  \item By the Chinese Remainder Theorem we know that
    \[(R/(I_1\ldots I_n))\cong (R/I_1)\times\ldots\times(R/I_n) \]
    as rings. As such there is an isomorphism
    $\varphi:R\rightarrow S$ and since ring isomorphisms send units to units we have
    a bijection $\varphi|_{R^X}:R^X\rightarrow S^X$. However since $\varphi$ is an isomorphism
    it will send identity to identity and preserve multiplication.

    Thus $\varphi|_{R^X}$ is a group homomorphism and
    \[ (R/(I_1\ldots I_n))^X\cong (R/I_1)^X\times\ldots\times(R/I_n)^X \]
    are isomorphic as groups.
    
  \item From a prior homework we have that $(n)(m)=(nm)$. Therefore
    $\bb{Z}_{mn}\cong \bb{Z}/m\bb{Z}n\bb{Z}$ in addition to
    $\bb{Z}_n\cong\bb{Z}/n\bb{Z}$ and $\bb{Z}_m\cong \bb{Z}/m\bb{Z}$.
    Finally note that since $\gcd(n,m)=1$ there exist $\alpha,\beta\in\bb{Z}$
    such that $\alpha a+\beta b = 1$ and therefore $(n)+(m)=\bb{Z}$. Thus
    we can apply part 1 of this problem to get that
    \[ \bb{Z}_{mn}^X\cong\bb{Z}_m^X\times\bb{Z}_n^X \]
  \item $x\cong 533 \mod 3465$
  \end{enumerate}
\end{proof}

\sk

\begin{problem}[8.1]
  Prove that the division algorithm holds for any polynomial ring over a field.
\end{problem}

\begin{proof}
  Let $f,g\in k[x]$ where $k$ is a field. We will prove the division algorithm
  holds via induction over the degree of $f$.

  Let $\text{deg}(f)=0$. Then $f=0\cdot g+f$ and since $\deg(f)=0<\deg(g)$ this is
  a valid choice for the division algorithm.

  Assume that the division algorithm holds for polynomials $f\in k[x]$ when
  $\deg(f)= n$. Then given $g\in k[x]$ we have $f=qg+r$ where $\deg(g)>\deg(r)$.
  However we can form any given a polynomial $f'=\sum_0^{n+1}a_ix^i$ let
  $f:= \sum_0^na_{i+1}x^i+a_0$. Then $f'=x\cdot f+a_0$. Apply the division algorithm to
  $f$ and we get
  \[ f'=x(qg+r)=(q\cdot x)g + x\cdot r\]
  which shows that the division algorithm holds for $\deg=k+1$ if we assume
  it for $\deg =k$.

  Therefore the division algorithm holds for any polynomial ring over a field.
\end{proof}

\sk

\begin{problem}[8.2]
  \begin{enumerate}
  \item Prove that $a|b$ iff $b\in(a)$ iff $(b)\subseteq(a)$.
  \item If $a|b$ and $a|c$, prove that $a|(bx+cy)$ for all $x,y\in R$.
  \item Suppose $b\neq 0$. If $a|b$ and $b|c$, then $a|c$.
  \item If $d$ is a greatest common divisor of $a,b$ then $du$
    is also a greatest common divisor of $a,b$ for any unit $u$
    of $R$.
  \end{enumerate}
\end{problem}

\begin{proof}
  \begin{enumerate}
  \item Suppose that $a|b$. Then there exists a $c$ such that $ac=b$ which implies
    that $b\in (a)$.

    Next suppose that $b\in(a)$ and let $d\in (b)$. Then $d=fb$. However $b=ca$ since
    $b\in (a)$ and as such $d=fca\in (a)$.

    Finally suppose that $(b)\subseteq (a)$. Then $b\in(a)$ which implies that $b=ca$ for
    some $c$. This is the definition of $a|b$.

    Therefore $a|b$ iff $b\in(a)$ iff $(b)\subseteq(a)$.
    
  \item Suppose that $a|b$ and that $a|c$. Then there exist $\beta,\gamma\in R$ such that
    $a\beta = b$ and $a\gamma = c$. If we have $(bx+cy)$ we can substitute $b,c$ to get
    \[ bx+cy = a\beta x + a\gamma c = a(\beta x + \gamma y)\]
    which implies that $a|(bx+cy)$.

    Therefore if $a|b$ and $a|c$ then $a|(bx+cy)$ for all $x,y\in R$.

  \item Suppose that $a|b$ and $b|c$. Then there exist $\alpha,\beta\in R$ such that
    $\alpha a = b$ and $\beta b = c$. Thus $\beta\alpha a =c$ and therefore $a|c$.

  \item Let $d$ be a greatest common divisor of $a,b$. Then
    $d|a$, $d|b$ and if $d'|a,b$ then $d|d'$. Consider $ud$ where $u$ is a unit.
    Then $du$ divides $a,b$ as $\alpha d=a$ and $\beta d=b$ which implies that
    $\alpha u^{-1}(ud) =a $ and $\beta u^{-1}(ud) = b$. Thus $ud|a$ and $ud|b$.
    Now suppose that $f|a$ and $f|b$. Then $f|d$ which implies that
    $\gamma f = d$. It then follows that $u\gamma f=ud$ and thus $f|ud$.

    Therefore $ud$ is a greatest common divisor of $a$ and $b$.
  \end{enumerate}
\end{proof}

\sk

\begin{problem}[8.3]
  An element $p$ in an integral domain $R$ is prime if, and only if,
  $p|ab$ implies $p|a$ or $p|b$ for any $a,b\in R$.
\end{problem}

\begin{proof}
  Let $p\in R$ be prime. Suppose that $p|ab$. Then $\alpha p=ab$ which implies
  that $ab\in (p)$. However since $p$ is prime, $(p)$ is a prime ideal.
  This means that either $a\in (p)$, in which case $p|a$, or $b\in (p)$ with $p|b$.

  Otherwise suppose that when $p|ab$ then $p|a$ or $p|b$. Let $ab\in (p)$. Then
  $p|ab$ which implies that $p|a$, in which case $a\in (p)$, or $p|b$ and $b\in (p)$
  which shows that $(p)$ is prime.

  Therefore an element $p$ in an integral domain $R$ is prime if, and only if,
  $p|ab$ implies that $p|a$ or $p|b$ for any $a,b\in R$.
\end{proof}

\sk

\begin{problem}[8.4]
  Let $R$ be a UFD and $a,b\in R\setminus\{0\}$. Then $a,b$ has a greatest
  common divisor in $R$. If $a,b$ are relatively prime and $a|bc$ for some
  $c\in R$, then $a|c$.
\end{problem}

\begin{proof}
  Let $a,b\in R\setminus\{0\}$. Since $R$ is a UFD we have unique factorizations
  $a=p_1\cdots p_s$ and $b=q_1\cdots q_t$ which are unique up to associates and permutations.
  Define $f:R\rightarrow \bb{N}$ by $f(r)$ as the minimum number of occurrences of
  $r$ in the factorization of $a$ or $b$ up to associates. Then
  define $d:=\prod_{r\in\{p_i,q_i\}/\text{associates}}r^{f(r)}$. We know that $d|a$ and $d|b$
  since $d$ contains only factors of $a$ and $b$. Moreover if $d'|a$ and $d'|b$
  then $d'|d$ as $d$ was defined to contain as many factors as possible while
  still dividing $a,b$. Thus $d$ is a $\gcd$ of $a,b$.

  Therefore if $a,b\in R\setminus\{0\}$ and $R$ is a UFD, then $a,b$ have a greatest common
  divisor.
\end{proof}

\sk

\begin{problem}[G1]
  Let $H$ be a normal subgroup of a group $G$, and let $K$ be a subgroup of $H$.
  \begin{enumerate}
  \item Give an example of this situation where $K$ is not a normal
    subgroup of $G$.
  \item Prove that if the normal subgroup $H$ is cyclic, then $K$ is normal
    in $G$.
  \end{enumerate}
\end{problem}

\begin{proof}
  \begin{enumerate}
  \item Consider $S_5$ and $A_5$. We know that $A_5\trianglelefteq S_5$ however
    the subgroup $\langle(1\ 2\ 3)\rangle$ is not normal in $S_5$.
  \item Since $H=\langle h\rangle$ is cyclic we know that $K=\langle h^a\rangle$ as well. Thus
    if we consider $ghg^{-1}=h^k\in H$. Then if we raise both sides to the
    power $ap$ we get $(ghg^{-1})^{ap}=gh^{ap}g^{-1}=h^{(kp)a}\in K$.

    Therefore $K\trianglelefteq G$.
  \end{enumerate}
\end{proof}

\sk

\begin{problem}[G2]
  Prove that every finite group of order at least three has a nontrivial
  automorphism.
\end{problem}

\begin{proof}
  Suppose that $G$ is abelian. Then the map $a\mapsto a^{-1}$ is an automorphism.

  Otherwise if $G$ is not abelian then exists $gh\neq hg$. Then the map
  $f\mapsto gfg^{-1}$ will be an automorphism that is not trivial.

  Therefore every finite group of order at least three has a non-trivial
  automorphism.
\end{proof}

\sk

\begin{problem}[R1]
  Let $R=\bb{Z}[\sqrt{-3}]=\{a+b\sqrt{-3}|a,b\in\bb{Z}\}$.
  \begin{enumerate}
  \item Why is $R$ an integral domain?
  \item What are the units in $R$?
  \item Is the element 2 irreducible in $R$?
  \item If $x,y\in R$, and $2|xy$, does it follow that $2$ divides
    either $x$ or $y$? Justify your answer.
  \end{enumerate}
\end{problem}

\begin{proof}
  \begin{enumerate}
  \item Note that $\sqrt{-3}=i\sqrt{3}$. As such $R$ is a subring of $\bb{C}$
    which is a field. As such $R$ cannot contain any zero divisors.
  \item The units of $R$ will be the elements $a+b\sqrt{-3}$ such that
    $a^2+3b^2=\pm 1$. The negative case cannot happen and any nonzero value for $b$ will
    make it too large. Therefore the only units of $R$ are $\pm 1$.
  \item Since the norm is multiplicative if $rs=2$ then
    $4=N(r)N(s)=(ac)^2+3b^2c^2+3a^2d^2+9b^2d^2$. There are three was this can be fulfilled
    up to swapping $r$ and $s$. Either $r=2$ and $s=\pm 1$ or $r=(1+\sqrt{3}),s=1$.
    However the latter doesn't fulfill $rs=2$ so we can discard it. Therefore
    $2$ is irreducible in $R$
  \item The number $2$ is not prime in $R$ as $2|(4=(1+\sqrt{-3})(1-\sqrt{-3}))$
    however $2\nmid (1\pm \sqrt{-3})$.
  \end{enumerate}
\end{proof}

\sk

\begin{problem}[R2]
  \begin{enumerate}
  \item Give an example of an integral domain with exactly 9 elements.
  \item Is there an integral domain with exactly 10 elements? Justify
    your answer.
  \end{enumerate}
\end{problem}

\begin{proof}
  \begin{enumerate}
  \item $\bb{Z}_3[\sqrt{2}]$
  \item When a ring is finite it is an integral domain if and only if it is
    a field. However fields must have prime power order and since $10$ is not
    a prime power there cannot exist a field, and thus an integral domain, of
    order $10$.
  \end{enumerate}
\end{proof}

\sk

\begin{problem}[R3]
  Let
  \[
    F =
    \left\{\left(
        \begin{array}{cc}
          a&b\\
          2b&a
        \end{array}
      \right)| a,b\in \bb{Q}\right\}
  \]
  \begin{enumerate}
  \item Prove that $F$ is a field under the usual matrix operations of addition
    and multiplication.
  \item Prove that $F$ is isomorphic to the field $\bb{Q}(\sqrt{2})$.
  \end{enumerate}
\end{problem}

\begin{proof}
  \begin{enumerate}
  \item First we will show that $F$ is closed under matrix addition
    and multiplication. Let \[
      \left(
        \begin{array}{cc}
          a&b\\
          2b&a\\
        \end{array}
      \right),
      \left(
        \begin{array}{cc}
          c&d\\
          2d&c\\
        \end{array}
      \right)\in F\]
    Then
    \[
      \left(
        \begin{array}{cc}
          a&b\\
          2b&a\\
        \end{array}
      \right)+
      \left(
        \begin{array}{cc}
          c&d\\
          2d&c\\
        \end{array}
      \right)
      = 
      \left(
        \begin{array}{cc}
          a+c&b+d\\
          2(b+d)&a+c
        \end{array}
      \right)
    \]
    and
    \[
      \left(
        \begin{array}{cc}
          a&b\\
          2b&a\\
        \end{array}
      \right)
      \left(
        \begin{array}{cc}
          c&d\\
          2d&c\\
        \end{array}
      \right)
      = 
      \left(
        \begin{array}{cc}
          ac+2bd&bc+ad\\
          2(bc+ad)&ac+2bd\\
        \end{array}
      \right)
    \]
    Therefore $F$ is a subring of $M_2(\bb{Q})$ and the only thing left to
    check is that all nonzero elements are units.

    Suppose that we have an element $x\in F$ such that $a,b$ are not both zero.
    Then $\det(x)\neq 0$ as that would require $a^2-2b^2=0$ which cannot happen
    in the rationals. Therefore $x$ does indeed have an inverse.
  \item Define $\varphi:F\rightarrow\bb{Q}(\sqrt{2})$ as
    \[ \varphi
      \left(
        \begin{array}{cc}
          a&b\\
          2b&a\\

        \end{array}
      \right)=a+b\sqrt{2}\]
    Then we can define an inverse $\psi:\bb{Q}(\sqrt{2})\rightarrow F$ as
    \[ \psi(a+b\sqrt{2})=
      \left(
        \begin{array}{cc}
          a&b\\
          2b&a\\
        \end{array}
      \right)\]
    Clearly $\varphi$ and $\psi$ are inverses and as such $\varphi$ is a bijection.

    To show that $\varphi$ is a ring homomorphism note that in part 1 of the
    problem the top two lines coincide with the values of the rational
    part and the $\sqrt{2}$ part of addition and multiplication respectively.
    Therefore $\varphi$ is a ring isomorphism and as such $F\cong \bb{Q}(\sqrt{2})$.
  \end{enumerate}
\end{proof}

\sk

%%%%%%%%%%%%%%%%%%%%%%%%%%%%%%%%%%%%%%%%%%%%%%%%%%%%%%%%%%%%%%%%%%%%%%%%%%%%% 
\end{document}
