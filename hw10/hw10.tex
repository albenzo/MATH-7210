\documentclass[10pt]{article}
\usepackage[utf8]{inputenc}
\usepackage{amscd}
\usepackage{amsmath}
\usepackage{amssymb}
\usepackage{amsthm}
\usepackage{listings}
\usepackage{enumerate}

\textwidth=15cm \textheight=22cm \topmargin=0.5cm \oddsidemargin=0.5cm \evensidemargin=0.5cm

\newcommand{\sk}{\vskip 10mm}
\newcommand{\bb}[1]{\mathbb{#1}}
\newcommand{\ra}{\rightarrow}

\theoremstyle{plain}


\newtheorem{problem}{Problem}
\newtheorem{lemma}{Lemma}[problem]

\theoremstyle{remark}
\newtheorem{tpart}{}[problem]
\newtheorem*{ppart}{}

\begin{document}

\begin{problem}[7.14]
  \begin{enumerate}
  \item Let $R$ be a commutative ring with $1\neq 0$ and $I_1,\ldots,I_n$
    pairwise comaximal ideals of $R$. Prove that
    \[ (R/(I_1\ldots I_n))^X\cong (R/I_1)^X\times\ldots\times(R/I_n)^X \]
    as groups.
  \item Let $m,n$ be relatively prime positive integers. Prove that
    \[ (\bb{Z}_{mn})^X\cong (\bb{Z}_m)^X\times(\bb{Z}_n)^X \]
    as groups.
  \item Solve the system of congruences:
    \begin{align*}
      x &\equiv 2 \mod 9\\
      x &\equiv 3 \mod 5\\
      x &\equiv 1 \mod 7\\
      x &\equiv 5 \mod 11\\
    \end{align*}
  \end{enumerate}
\end{problem}

\begin{proof}
  \begin{enumerate}
  \item
  \item
  \item $x\cong 533 \mod 3465$
  \end{enumerate}
\end{proof}

\sk

\begin{problem}[8.1]
  Prove that the division algorithm holds for any polynomial ring over a field.
\end{problem}

\begin{proof}
  
\end{proof}

\sk

\begin{problem}[8.2]
  \begin{enumerate}
  \item Prove that $a|b$ iff $b\in(a)$ iff $(b)\subseteq(a)$.
  \item If $a|b$ and $a|c$, prove that $a|(bx+cy)$ for all $x,y\in R$.
  \item Suppose $b\neq 0$. If $a|b$ and $b|c$, then $a|c$.
  \item If $d$ is a greatest common divisor of $a,b$ then $du$
    is also a greatest common divisor of $a,b$ for any unit $u$
    of $R$.
  \end{enumerate}
\end{problem}

\begin{proof}
  
\end{proof}

\sk

\begin{problem}[8.3]
  An element $p$ in an integral domain $R$ is prime if, and only if,
  $p|ab$ implies $p|a$ or $p|b$ for any $a,b\in R$.
\end{problem}

\begin{proof}
  
\end{proof}

\sk

\begin{problem}[8.4]
  Let $R$ be a UFD and $a,b\in R\setminus\{0\}$. Then $a,b$ has a greatest
  common divisor in $R$. If $a,b$ are relatively prime and $a|bc$ for some
  $c\in R$, then $a|c$.
\end{problem}

\begin{proof}
  
\end{proof}

\sk

\begin{problem}[G1]
  Let $H$ be a normal subgroup of a group $G$, and let $K$ be a subgroup of $H$.
  \begin{enumerate}
  \item Give an example of this situation where $K$ is not a normal
    subgroup of $G$.
  \item Prove that if the normal subgroup $H$ is cyclic, then $K$ is normal
    in $G$.
  \end{enumerate}
\end{problem}

\begin{proof}
  \begin{enumerate}
  \item Consider $S_5$ and $A_5$. We know that $A_5\trianglelefteq S_5$ however
    the subgroup $\langle(1\ 2\ 3)\rangle$ is not normal in $S_5$.
  \end{enumerate}
\end{proof}

\sk

\begin{problem}[G2]
  Prove that every finite group of order at least three has a nontrivial
  automorphism.
\end{problem}

\begin{proof}
  
\end{proof}

\sk

\begin{problem}[R1]
  Let $R=\bb{Z}[\sqrt{-3}]=\{a+b\sqrt{-3}|a,b\in\bb{Z}\}$.
  \begin{enumerate}
  \item Why is $R$ an integral domain?
  \item What are the units in $R$?
  \item Is the element 2 irreducible in $R$?
  \item If $x,y\in R$, and $2|xy$, does it follow that $2$ divides
    either $x$ or $y$? Justify your answer.
  \end{enumerate}
\end{problem}

\begin{proof}
  
\end{proof}

\sk

\begin{problem}[R2]
  \begin{enumerate}
  \item Give an example of an integral domain with exactly 9 elements.
  \item Is there an integral domain with exactly 10 elements? Justify
    your answer.
  \end{enumerate}
\end{problem}

\begin{proof}
  \begin{enumerate}
  \item $\bb{Z}_3[\sqrt{2}]$
  \item
  \end{enumerate}
\end{proof}

\sk

\begin{problem}[R3]
  Let
  \[
    F =
    \left\{\left(
        \begin{array}{cc}
          a&b\\
          2b&a
        \end{array}
      \right)| a,b\in \bb{Q}\right\}
  \]
  \begin{enumerate}
  \item Prove that $F$ is a field under the usual matrix operations of addition
    and multiplication.
  \item Prove that $F$ is isomorphic to the field $\bb{Q}(\sqrt{2})$.
  \end{enumerate}
\end{problem}

\begin{proof}
  
\end{proof}

\sk

%%%%%%%%%%%%%%%%%%%%%%%%%%%%%%%%%%%%%%%%%%%%%%%%%%%%%%%%%%%%%%%%%%%%%%%%%%%%%
\end{document}
