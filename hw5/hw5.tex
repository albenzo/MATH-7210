\documentclass[10pt]{article}
\usepackage[utf8]{inputenc}
\usepackage{amscd}
\usepackage{amsmath}
\usepackage{amssymb}
\usepackage{amsthm}
\usepackage{listings}
\usepackage{enumerate}

\textwidth=15cm \textheight=22cm \topmargin=0.5cm \oddsidemargin=0.5cm \evensidemargin=0.5cm

\newcommand{\sk}{\vskip 10mm}
\newcommand{\bb}[1]{\mathbb{#1}}
\newcommand{\ra}{\rightarrow}

\theoremstyle{plain}
\newtheorem{problem}{Problem}
\newtheorem{lemma}{Lemma}[problem]

\theoremstyle{remark}
\newtheorem{tpart}{}[problem]
\newtheorem*{ppart}{}

\begin{document}

\begin{problem}[3.1.3] %1
  
\end{problem}

\begin{proof}
  Let $a,b\in G$. Then $\text{Inn}(ab)=\phi_{ab}$. For any $g\in G$ we have
  \[\phi_{ab}(g)=abgb^{-1}a^{-1}=a\phi_b(g)a^{-1}=\phi_a\circ\phi_b(g)\]
  Which implies that $\phi_{ab}=\phi_a\circ\phi_b=\text{Inn}(a)\circ\text{Inn}(b)$
  completing the proof that $\text{Inn}$ is a group homomorphism.

  Next, for an element $a\in G$ to be in the kernel of $\text{Inn}$
  it is required to have $\phi_a(g)=g$. Then
  $gag^{-1}=a$ which by cancellation we get $ga=ag$ for all $g\in G$.
  Therefore the kernel of $\text{Inn}$ is $Z(G)$.

  Finally let $\phi_a\in\text{Inn}(G)$ and $\varphi\in\text{Aut}(G)$. Then
  the function $\varphi\circ\phi_a\circ\varphi^{-1}$ for an element $g\in G$ is
  \[ \varphi\circ\phi_a\circ\varphi^{-1}(g)=\varphi(a\varphi^{-1}(g)a^{-1})=\varphi(a)\varphi\circ\varphi^{-1}(g)\circ\varphi(a^{-1})=\varphi(a)g\varphi(a)^{-1}=\phi_{\varphi(a)}(g) \]
  which shows that $\text{Inn}(G)$ is closed under conjugation and is therefore a
  normal subgroup of $\text{Aut}(G)$.
\end{proof}


\textbf{Elaborate on why this is the case}.
The Automorphism group for $D_8$ is isomorphic to $D_8$ 

The Inner Automorphism group for $D_8$ is isomorphic to $\mathcal{K}_4$

\sk

\begin{problem}[3.4] %2
  
\end{problem}

\begin{proof}
  Let $\phi:G\rightarrow \bar{G}$ be an epimorphism with $N:=\text{ker}\phi$ and $H$ a subgroup of $G$ containing $N$. Then
  we will show that $H$ is normal in $G$ if and only if $\phi(H)$ is normal in $\bar{G}$. For the forward
  direction suppose that $H$ is normal in $G$. Let $g'\in \bar{G}$ and $h'\in \phi(H)$. Since $\phi$ is an epimorphism
  there exist $g\in G$ and $h\in H$ such that $\phi(g)=g'$ and $\phi(h)=h'$. Then
  \[ g'h'g'^{-1}=\phi(g)\phi(h)\phi(g^{-1})=\phi(ghg^{-1})\]
  which is in $\phi(H)$ since $H$ is normal. Therefore $\phi(H)$ is normal.

  Now suppose that $\phi(H)$ is normal in $\bar{G}$. Let $g\in G$ and $h\in H$. Then
  $\phi(ghg^{-1})=\phi(g)\phi(h)\phi(g^{-1})\in\phi(H)$ and since $\phi(ghg^{-1})$ lies in the image $\phi(H)$ this implies
  that $ghg^{-1}\in H$ and therefore $H$ is normal.

  Next consider the map $\varphi:H\mapsto \phi(H)$ on the lattice of subgroups containing $N$.
  We will show that it is a bijection by proving that it is surjective and
  injective. For surjectivity let $K\leq \bar{G}$. Then $\phi(K)\leq G$. Note that
  $\phi^{-1}(\bar{e})\leq\phi^{-1}(K)$ which implies that $N\leq\phi^{-1}(K)$. Therefore $\phi(\phi^{-1}(K))=K$.

  For injectivity let $\phi(H)=\phi(K)$ for $H,K\leq G$ where $H,K$ contain $N$. Without loss of generality let
  $h\in H$. Then $\phi(h)\in\phi(K)$ and $\phi(h)=\phi(k)$ for some $k\in K$. This implies that $\phi(k^{-1}h)=\bar{e}$ and
  it follows that $k^{-1}h\in N\leq K$. However since $k\in K$ we have that $h\in K$. Therefore $H\subset K$ and via the
  same reasoning $K\subset H$ and therefore $H=K$.

  Therefore the map $\varphi$ is a bijection.

  Finally let $B\subset A$ be subgroups of $G$ that contain $N$. Then define a map $C:A/B\rightarrow \phi(A)/\phi(B)$
  via $C(aB)=\phi(a)\phi(B)$. To show that the map is well defined let $aB=a'B$. Then $a=a'b$ for some $b\in B$.
  So
  \[ \phi(a)\phi(B)=\phi(a'b)\phi(B)=\phi(a')\phi(b)\phi(B)=\phi(a')\phi(B)\]
  which shows that the map $C$ is well defined.

  To show surjectivity let $a'\phi(B)\in \phi(A)/\phi(B)$. Then because $\phi$ is surjective there exists an $a\in A$ such that
  $\phi(a)=a'$. Then $C(aB)=a'\phi(B)$ which shows is surjective.

  Finally to show injectivity let $C(aB)=C(a'B)$. Then we have
  \[ \phi(a)\phi(B)=\phi(a')\phi(B)\]
  which implies that
  \[ \phi(B)=\phi(a^{-1})\phi(a')\phi(B)\]
  and thus $\phi(a^{-1}a')\in N\leq B$ which implies that $aB=a'B$.

  Therefore there is a bijection between $A/B$ and $\phi(A)/\phi(B)$ and as such $|A:B|=|\phi(A):\phi(B)|$.
\end{proof}

\sk

\begin{problem}[3.5] %3
  
\end{problem}

\begin{proof}
  Let $H$ be a normal subgroup of $G$ where $|G:H|=p$. Let $K\leq G$ then there are two cases.
  If $K\leq H$ then $K\leq H$ and we are done. Otherwise suppose that $K\nleq H$. We know that
  $|G/H|=p$ which implies that $G/H$ is cyclic. There exists a $k\in K\setminus H$ such that
  $\langle kH\rangle = G/H$. However this means that for any $g\in G$ we have $g\in k^iH$ and therefore
  $g=k^ih$ where $h\in H$. Therefore $g\in HK$ and as such $G=HK$.

  By the second isomorphism theorem we have $G/H\cong HK/H\cong K/(H\cap K)$ which implies that
  $|K:K\cap H|=p$.
\end{proof}

\sk

\begin{problem}[3.6] %4
  
\end{problem}

\begin{proof}
  \begin{enumerate}
  \item Let $G,G'$ be groups. Then we will show that $G\times G'$ is a group
    under pointwise multiplication.

    \begin{enumerate}
    \item[associativity:] Consider elements $(a,a'),(b,b'),(c,c')\in G\times G'$.
      Then we have
      \begin{align*}
        ((a,a')\cdot(b,b'))\cdot(c,c')&=(ab,a'b')\cdot(c,c')\\
                              &=((ab)c,(a'b')c')\\
                              &= (a(bc),a'(b'c'))\\
                              &=(a,a')\cdot(bc,b'c')\\
                              &=(a,a')\cdot((b,b')\cdot(c,c'))
      \end{align*}
      Which shows that the group operation is associative.
    \item[identity:] Consider $(e,e')$ made up of the identity elements of
      $G$ and $G'$ respectively. Then for $(g,g')\in G\times G'$ we have
      \[ (e,e')\cdot(g,g')=(eg,e'g')=(g,g')=(ge,g'e')=(g,g')(e,e')\]
      which shows the existence of an identity.
    \item[inverse:] Let $(g,g')\in G$. Then 
      \begin{align*}
        (g,g')\cdot(g^{-1},g'^{-1}) &=(gg^{-1},g'g'^{-1})\\
                                &=(e,e')\\
                                &=(g^{-1}g,g'^{-1}g')\\
                                &=(g^{-1},g'^{-1})(g,g')\\
      \end{align*}
      Which shows that for any element we have a two sided inverse.
    \end{enumerate}
    Therefore the Cartesian product of groups $G\times G'$ is a group under pointwise
    multiplication.

  \item Let $M,N\trianglelefteq G$  such that $G=MN$. Define a map
    $\phi:G\rightarrow G/M\times G/N$ as $\phi(g)=(gM,gN)$. First we will show the
    map is surjective. Let $(gM,g'N)\in G/M\times G/N$. Since $MN=G$ we can rewrite
    $g=mn$ and $g'=m'n'$ to get
    \[ (mnM,m'n'N)=(Mmn,m'N)=(Mn,m'N)=(nM,m'N)\]
    using normality. Then consider $\phi(m'n)$ to get
    \[ \phi(m'n)=(m'nM,m'nN)=(Mm'n,m'N)=(Mn,m'N)=(nM,m'N)\]
    completing the proof that $\phi$ is surjective.

    Then $\text{Ker}\phi$ is the elements $g$ such that $gM=M$ and $gN=N$ which
    consists entirely of $g\in M\cap N$. If we apply the first isomorphism theorem
    we get
    \[ G/M\cap N \cong G/M\times G/N\]
    completing the proof.
  \end{enumerate}
\end{proof}

\sk

\begin{problem}[3.1.17] %5
  
\end{problem}

\begin{enumerate}
\item The order of $\langle r^4\rangle$ is $2$ which implies that
  $|D_{16}/\langle r^4\rangle|=\frac{|D_{16}|}{|\langle r^4\rangle|}=8$.
\item The elements of $D_{16}/\langle r^4\rangle$ are
  \[\{\bar{r}^0,\bar{r}^1,\bar{r}^2,\bar{r}^3,\bar{s},\bar{s}\bar{r},\bar{s}\bar{r}^2,\bar{s}\bar{r}^3\}\]
\item The order of each element in the order listed above is
  \[ 1,4,2,4,2,2,2,2\]
\item $\overline{rs}=\bar{s}\bar{r}^3$\\
  $\overline{sr^2s}=\bar{r}^2$\\
  $\overline{s^{-1}r^{-1}sr}=\bar{r}^2$
\item The size of $\bar{H}$ is 4 and as such is of index $2$ in $\bar{G}$ which
  is sufficient to show the normality of $\bar{H}$. Since $\bar{H}$ is of size
  4 and each non-identity element, $\bar{s},\bar{s}\bar{r}^2,\bar{r^2}$ is of order
  2 it must be isomorphic to $\mathcal{V}_4$.

  The preimage of $\bar{H}$ in $D_{16}$ is $\{e,r^2,r^4,r^6,s,sr^2,sr^4,sr^6\}$ which is
  isomorphic to $D_8$.
\item The center of $\bar{G}$ is $\bar{e},\bar{r^2}$. Then
  the size of $\bar{G}/Z(\bar{G})$ is 4. Then our elements will be similar to
  $\{\bar{e},\bar{r},\bar{s},\bar{rs}\}$ and since each is of order $2$ aside from the
  identity the quotient is isomorphic to $\mathcal{V}_4$.
\end{enumerate}

\sk

\begin{problem}[3.1.32] %6
  
\end{problem}

\begin{proof}
  The proof that all subgroups of $\mathcal{Q}_8$ are normal has been copied from
  the last homework at the end of the problem. For the isomorphism types of the
  quotients we have:
  \begin{itemize}
  \item $\mathcal{Q}_8/\langle 1\rangle\cong \mathcal{Q}_8$
  \item $\mathcal{Q}_8/\langle\mathcal{Q}_8\rangle\cong \langle 1\rangle$
  \item $\mathcal{Q}_8/\langle-1\rangle\cong \mathcal{V}_4$. To see this note that the
    quotient group will consist of four elements each of order two. Since
    there are only two groups of order four and the other is cyclic it must
    be the Klein four group.
  \item $\mathcal{Q}_8/\langle i\rangle\cong\mathcal{Q}_8/\langle j\rangle\cong\mathcal{Q}_8/\langle k\rangle\cong \bb{Z}_2$. To
    see this note that the order of the quotient is two and there is only one
    group of order two.
  \end{itemize}

  \noindent Redux:
  The subgroups of $\mathcal{Q}_8$ are $<1>,<-1>,<i>,<j>,<k>,\mathcal{Q}_8$. The
    trivial group and $\mathcal{Q}_8$ are both normal. For the others we have
    \begin{itemize}
    \item[$<-1>$]
      \[
        \begin{array}{c|c}
          g & g(-1)g^{-1}\\
          \hline 
          1 & -1\\
          -1 & -1\\
          i & -1\\
          -i & -1\\
          j & -1\\
          -j & -1\\
          k & -1\\
          -k & -1\\
        \end{array}
      \]
    \item[$<i>$]
      \[
        \begin{array}{c|c}
          g & g(i)g\\
          \hline 
          1 & i\\
          -1 & i\\
          i & i\\
          -i & i\\
          j & -i\\
          -j & -i\\
          k & -i\\
          -k & -i\\
        \end{array}
      \]
    \item[$<j>$]
      \[
        \begin{array}{c|c}
          g & g(j)g\\
          \hline 
          1 & j\\
          -1 & j\\
          i & -j\\
          -i & -j\\
          j & j\\
          -j & j\\
          k & -j\\
          -k & -j\\
        \end{array}
      \]
    \item[$<k>$]
      \[
        \begin{array}{c|c}
          g & g(k)g\\
          \hline 
          1 & k\\
          -1 & k\\
          i & -k\\
          -i & -k\\
          j & -k\\
          -j & -k\\
          k & k\\
          -k & k\\
        \end{array}
      \]
    \end{itemize}
\end{proof}

\sk

%%%%%%%%%%%%%%%%%%%%%%%%%%%%%%%%%%%%%%%%%%%%%%%%%%%%%%%%%%%%%%%%%%%%%%%%%%%%%
\end{document}
