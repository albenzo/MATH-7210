\documentclass[10pt]{article}
\usepackage[utf8]{inputenc}
\usepackage{amscd}
\usepackage{amsmath}
\usepackage{amssymb}
\usepackage{amsthm}
\usepackage{listings}
\usepackage{enumerate}

\textwidth=15cm \textheight=22cm \topmargin=0.5cm \oddsidemargin=0.5cm \evensidemargin=0.5cm

\newcommand{\sk}{\vskip 10mm}
\newcommand{\bb}[1]{\mathbb{#1}}
\newcommand{\ra}{\rightarrow}

\theoremstyle{plain}
\newtheorem{problem}{Problem}
\newtheorem{lemma}{Lemma}[problem]

\theoremstyle{remark}
\newtheorem{tpart}{}[problem]
\newtheorem*{ppart}{}

\begin{document}

\begin{problem}[3.1.3] %1
  
\end{problem}

\begin{proof}
  Let $a,b\in G$. Then $\text{Inn}(ab)=\phi_{ab}$. For any $g\in G$ we have
  \[\phi_{ab}(g)=abgb^{-1}a^{-1}=a\phi_b(g)a^{-1}=\phi_a\circ\phi_b(g)\]
  Which implies that $\phi_{ab}=\phi_a\circ\phi_b=\text{Inn}(a)\circ\text{Inn}(b)$
  completing the proof that $\text{Inn}$ is a group homomorphism.

  Next, for an element $a\in G$ to be in the kernel of $\text{Inn}$
  it is required to have $\phi_a(g)=g$. Then
  $gag^{-1}=a$ which by cancellation we get $ga=ag$ for all $g\in G$.
  Therefore the kernel of $\text{Inn}$ is $Z(G)$.

  Finally let $\phi_a\in\text{Inn}(G)$ and $\varphi\in\text{Aut}(G)$. Then
  the function $\varphi\circ\phi_a\circ\varphi^{-1}$ for an element $g\in G$ is
  \[ \varphi\circ\phi_a\circ\varphi^{-1}(g)=\varphi(a\varphi^{-1}(g)a^{-1})=\varphi(a)\varphi\circ\varphi^{-1}(g)\circ\varphi(a^{-1})=\varphi(a)g\varphi(a)^{-1}=\phi_{\varphi(a)}(g) \]
  which shows that $\text{Inn}(G)$ is closed under conjugation and is therefore a
  normal subgroup of $\text{Aut}(G)$.
\end{proof}


\textbf{Elaborate on why this is the case}.
The Automorphism group for $D_8$ is isomorphic to $D_8$ 

The Inner Automorphism group for $D_8$ is isomorphic to $\mathcal{K}_4$

\sk

\begin{problem}[3.4] %2
  
\end{problem}

\begin{proof}
  
\end{proof}

\sk

\begin{problem}[3.5] %3
  
\end{problem}

\begin{proof}
  
\end{proof}

\sk

\begin{problem}[3.6] %4
  
\end{problem}

\begin{proof}
  \begin{enumerate}
  \item Let $G,G'$ be groups. Then we will show that $G\times G'$ is a group
    under pointwise multiplication.

    \begin{enumerate}
    \item[associativity:] Consider elements $(a,a'),(b,b'),(c,c')\in G\times G'$.
      Then we have
      \begin{align*}
        ((a,a')\cdot(b,b'))\cdot(c,c')&=(ab,a'b')\cdot(c,c')\\
                              &=((ab)c,(a'b')c')\\
                              &= (a(bc),a'(b'c'))\\
                              &=(a,a')\cdot(bc,b'c')\\
                              &=(a,a')\cdot((b,b')\cdot(c,c'))
      \end{align*}
      Which shows that the group operation is associative.
    \item[identity:] Consider $(e,e')$ made up of the identity elements of
      $G$ and $G'$ respectively. Then for $(g,g')\in G\times G'$ we have
      \[ (e,e')\cdot(g,g')=(eg,e'g')=(g,g')=(ge,g'e')=(g,g')(e,e')\]
      which shows the existence of an identity.
    \item[inverse:] Let $(g,g')\in G$. Then 
      \begin{align*}
        (g,g')\cdot(g^{-1},g'^{-1}) &=(gg^{-1},g'g'^{-1})\\
                                &=(e,e')\\
                                &=(g^{-1}g,g'^{-1}g')\\
                                &=(g^{-1},g'^{-1})(g,g')\\
      \end{align*}
      Which shows that for any element we have a two sided inverse.
    \end{enumerate}
    Therefore the Cartesian product of groups $G\times G'$ is a group under pointwise
    multiplication.

  \item Let $M,N\trianglelefteq G$  such that $G=MN$.
  \end{enumerate}
\end{proof}

\sk

\begin{problem}[3.1.17] %5
  
\end{problem}

\begin{proof}
  
\end{proof}

\sk

\begin{problem}[3.1.32] %6
  
\end{problem}

\begin{proof}
  
\end{proof}

\sk

%%%%%%%%%%%%%%%%%%%%%%%%%%%%%%%%%%%%%%%%%%%%%%%%%%%%%%%%%%%%%%%%%%%%%%%%%%%%%
\end{document}
