\documentclass[10pt]{article}
\usepackage[utf8]{inputenc}
\usepackage{amscd}
\usepackage{amsmath}
\usepackage{amssymb}
\usepackage{amsthm}
\usepackage{listings}
\usepackage{enumerate}

\textwidth=15cm \textheight=22cm \topmargin=0.5cm \oddsidemargin=0.5cm \evensidemargin=0.5cm

\newcommand{\sk}{\vskip 10mm}
\newcommand{\bb}[1]{\mathbb{#1}}
\newcommand{\ra}{\rightarrow}

\theoremstyle{plain}
\newtheorem{problem}{Problem}
\newtheorem{lemma}{Lemma}[problem]

\theoremstyle{remark}
\newtheorem{tpart}{}[problem]
\newtheorem*{ppart}{}

\begin{document}

\begin{problem}[6.1]
  \begin{enumerate}
  \item If $K\trianglelefteq G$ and $G/K$ are solvable, then $G$ is solvable.
  \item Prove that $S_n$ is not solvable for $n\geq 5$.
  \end{enumerate}
\end{problem}

\begin{proof}
  \begin{enumerate}
  \item Let $K\trianglelefteq G$ such that $G/K$ and $K$ are both solvable.
    Then there is a solvability series
    $G_0'\trianglelefteq\ldots\trianglelefteq G_n' =G/K$.
    Next define $G_i$ as $G_{i}'=G_i/K$. Since $K\trianglelefteq G$ and
    $G_i/K\trianglelefteq G_{i+1}/K$ we have that $G_i\trianglelefteq G_{i+1}$.
    We know that $G_{i+1}/G_i$ is abelian because
    $G_{i+1}/G_i\cong (G_{i+1}/K)/(G_i/K)=G_{i+1}'/G_i'$ which are abelian.
    Since $K$ has a solvability series
    $H_0\trianglelefteq \ldots\trianglelefteq H_m=K$ we can stitch
    them together to get
    \[ \{e\}=H_0\trianglelefteq\ldots\trianglelefteq H_m=K=G_0\trianglelefteq
      \ldots\trianglelefteq\]
    which is a solvability series for $G$.

    Therefore the group $G$ is solvable.
  \item Since $A_n$ for $n\geq 5$ is simple and not Abelian we know that
    $A_n$ is not solvable. Since $A_n$ is the only nontrivial subgroup of
    $S_n$ and $S_n$ is not abelian for $n\geq 5$ any solvability series of $S_n$
    would be required to include $A_n$. However since $A_n$ is not solvable
    this cannot occur.

    Therefore $S_n$ is not solvable for $n\geq 5$.
  \end{enumerate}
\end{proof}

\sk

\begin{problem}[6.2]
  A finite group $G$ is solvable if, and only if, every composition factor
  of a composition series of $G$ is cyclic of prime order.
\end{problem}

\begin{proof}
  Suppose that there is a composition series for $G$ such that
  every factor of the composition series is cyclic of prime order. Then this
  composition series fulfills the condition to show that $G$ is solvable as
  cyclic groups are abelian.

  Otherwise suppose that $G$ is solvable. Then there is a normal series
  $\{e\}=G_0\trianglelefteq \ldots\trianglelefteq G_n=G$ such that the factors
  are finite and abelian. We can assume that the factors are not trivial as this
  would signify that $G_i=G_{i+1}$ for some $i$ and then we could remove $G_{i+1}$
  and the normal series would still witness solvability. Now suppose that
  $G_{i+1}/G_i$ was not simple. Then there would exist a normal subgroup
  $K'=K/G_i\trianglelefteq G_{i+1}/G_i$. This implies that $G_i\trianglelefteq K$
  and that $K\trianglelefteq G_{i+1}$. Since $K/G_i$ is a subgroup of an abelian
  group we know that it is abelian and $G_{i+1}/K$ will also be abelian because
  $G_i\leq K$ and this implies that $K$ contains the commutator. Thus we have a new
  series $G_o\trianglelefteq\ldots\trianglelefteq G_i\trianglelefteq K
  \trianglelefteq G_{i+1}\trianglelefteq\ldots\trianglelefteq G_n$. Now we can
  repeat this process until there are no non-simple factors left in our series.
  This process will terminate as $G$ is finite.

  This gives us a normal series $H_0\trianglelefteq\ldots\trianglelefteq H_m$
  wherein the factors are finite, abelian, non-trivial, and simple. Since
  the factors are non-trivial and simple this normal series is in fact
  a composition series. Moreover finite, abelian, and simple imply cyclic
  of prime order. Which means that all of the factors of this normal
  series are cyclic of prime order.

  Therefore a group $G$ is solvable if and only if there is a composition
  series wherein every factor is cyclic of prime order.
\end{proof}

\sk

\begin{problem}[6.3]
  \begin{enumerate}
  \item Let $G$ be a group and $\phi:M(X)\rightarrow G$ a monoid homomorphism which
    satisfies
    \[ \phi(s^{-1})=\phi(s)^{-1}\ \text{for all}\ s\in S \]
    then for any $w\in M(X)$, $\phi(w)=\phi(r(w))$
  \item Let $S$ be a set, $R$ a subset of $F(S)$, $G$ a group, $\phi:S\rightarrow G$
    a function and $\widetilde{\phi}:F(S)\rightarrow G$ the induced group homomorphism.
    If $\widetilde{\phi}(r)=e$ for all $r\in R$, then there exists a homomorphism
    $\overline{\phi}$ from $\langle S|R\rangle$ to $G$ such that $\overline{\phi}\circ\pi\circ i = \phi$ where
    $i:S\rightarrow F(S)$ is the inclusion map, $\pi:F(S)\rightarrow \langle S|R\rangle$ is the natural
    surjection, and $\widetilde{\phi}:F(S)\rightarrow G$ is the homomorphism
    satisfying $\widetilde{\phi}\circ i=\phi$.
  \end{enumerate}
\end{problem}

\begin{proof}
  \begin{enumerate}
  \item We will prove this statement by induction.
    Suppose that $|w|=0$. Then the only valid word is $\epsilon$ for which
    $\epsilon=r(\epsilon)$ and as such $\phi(\epsilon)=\phi(r(\epsilon))$.

    Now assume that for $|w|=n$ that $\phi(w)=\phi(r(w))$. Then let
    $|w|=n+1$. Decompose $w=sw_1$. Then we have
    \[ \phi(w)=\phi(sw_1)=\phi(s)\phi(w_1)\]
    From our inductive hypothesis we get
    \[ \phi(s)\phi(w_1)=\phi(s)\phi(r(w_1)) \]
    There are two cases to consider. Either $r(w_1)=s^{-1}w_2$ or it does not.
    In the latter case we can simply recombine to get
    \[ \phi(s)\phi(r(w_1))=\phi(sr(w_1))=\phi(r(sw_1))=\phi(r(w))\]
    completing the proof in that case. Otherwise we get
    \[ \phi(s)\phi(r(s^{-1}w_2))=\phi(s)\phi(s^{-1}r(w_2))=\phi(s)\phi(s)^{-1}\phi(r(w_2))=\phi(r(w_2))=\phi(r(w))\]
    completing the proof in that direction as well.

    Therefore $\phi(w)=\phi(r(w))$.
  \item First note that $\langle S|R\rangle\cong F(S)/N(R)$ where $N(R)$ denotes the normalizer
    of $R$. Then define $\overline{\phi}(sN(R)):=\tilde{\phi}(s)$. Now we show that
    $\overline{\phi}\circ\pi\circ i=\phi$. First note that $N(R)\subseteq \ker\phi$ as the kernel is a
    normal subgroup. Thus
    \[ \overline{\phi}\circ\pi\circ i(s)=\overline{\phi}(i(s)N(R))= \tilde{\phi}\circ i(s) \]
    and by definition of the free group we know that
    $\tilde{\phi}\circ i(s)=\phi$
    completing the proof.
  \end{enumerate}
\end{proof}

\sk

\begin{problem}[6.5.1]
  Using the Todd-Coxeter algorithm to determine and identify the group
  \[ G = \langle x,y| x^2=1,y^2=1, xyx=yxy\rangle \]
\end{problem}

First let $H=\langle y\rangle:= 1$. Then $y\cdot 1 = 1$ and we'll say that
$x\cdot 1 = 2$. Since both $x$ and $y$ are either order 1 or 2 we
have that
\[
  \begin{array}{ccccc}
    &y&&y&\\
    \hline    
    1 &&1&&1\\
    2&&a&&2\\
  \end{array}
\]

where we say that $y\cdot 2=a$ and
\[
  \begin{array}{ccccc}
    &x&&x&\\
    \hline
    1&&2&&1\\
    2&&1&&2\\
    a&&b&&a\\
  \end{array}
\]

However using the last relation of $G$ we have that
\[x\cdot a=(x(yx\cdot 1))=yxy\cdot 1\]
implying that $b=a$. If we let $3:=a$. Then we can
denote $y=(2\ 3)$ and $x=(1\ 2)$. The group generated
by $x,y$ is $S_3$.

Therefore $G\cong S_3$.

\sk

\begin{problem}[7.2]
  \begin{enumerate}
  \item Prove that $R^X$ is a group under the multiplication of $R$.
  \item Prove that $Z(R)\cap R^X=\emptyset$.
  \end{enumerate}
\end{problem}

\begin{proof}
  \begin{enumerate}
  \item We will show that $R^X$ is closed, has identity, is associative, and
    contains inverses.
    \begin{itemize}
    \item For closure let $a,b\in R^X$. Then $ab\in R^X$ as $(ab)\cdot(b^{-1}a^{-1})=e$.
    \item For identity $1=1\cdot 1=1\cdot 1^{-1}$ which implies that $1\in R^X$. Since
      $R$ is a multiplicative identity and $0\notin R^X$ this will be the identity
      for $R^X$.
    \item For associativity we have that for $a,b,c\in R^X$ that $a(bc)=(ab)c$
      as a property inherited from the overlying ring.
    \item Finally given $a\in R^X$ we let the inverse  its inverse in the ring.
    \end{itemize}
    Therefore since $R^X$ is closed, has identity, is associative, and contains
    inverses it is a group.
  \item Suppose that $a\in R^X$ and that there exists a $b\in R\setminus\{0\}$ such that
    $ab=0$ or $ba=0$. Without loss of generality assume that it is the
    former. Then we have
    \[ b=(a^{-1}a)b=a^{-1}(ab)=a^{-1}0=0\]
    which is a contradiction.

    Therefore $Z(R)\cap R^X=\emptyset$.
  \end{enumerate}
\end{proof}

\sk

\begin{problem}[7.3]
  \begin{enumerate}
  \item Find the set of all zero divisors of the commutative ring $C([0,1])$
    defined in example $7.3$. Determine the $C([0,1])^X$.
  \item Let $D\in\mathcal{Q}$ such that the equation $x^2=D$ has no solution
    $x\in\mathcal{Q}$. Prove that the set
    \[ \mathcal{Q}(\sqrt{D})=\{a+b\sqrt{D}|a,b\in\mathcal{Q}\}\]
    forms a field under the ordinary addition and multiplication of
    complex numbers.
   \item Prove $\bb{Z}_n$ is an integral domain if, and only if, $n$ is prime.
  \end{enumerate}
\end{problem}

\begin{proof}
  \begin{enumerate}
  \item The multiplicative inverse of a function if it exists will be
    $\frac{1}{f(x)}$. So $C([0,1])^X=\{f|f(x)\neq 0,x\in[0,1]\}$ since this will
    garentee a well defined inverse.

    For zero divisors it will be the set of functions where there is at least
    one value $x\in[0,1]$ such that $f(x)=0$ and there is an open neighborhood
    $x\in(a,b)\subseteq[0,1]$ such that $f(y)=0$ for all $y\in(a,b)$. If a function is
    in this form we can define its multiplicand to zero by taking a function
    that is zero everywhere but the interval and attaching a continuous
    nonzero portion to the interval. For the other direction suppose that
    all the zeros of $f$ are isolated. Then any multiplicand to get $f$ to
    zero would have to be zero everywhere except that isolated point breaking
    continuity.

    Therefore $Z(C[0,1])$ is the set of continuous functions such that
    at least one zero is not isolated.
  \item We will show that $\mathcal{Q}(\sqrt{D})$ is a field by showing that
    we have additive and multiplicative identities and that the operations are
    associative, commutative, distribute, and have inverses.
    \begin{itemize}
    \item For additive identity we have $0\in \mathcal{Q}$.
      \[0+(a+b\sqrt{D})=(a+b\sqrt{D})+0=a+b\sqrt{D}\]
    \item For multiplicative identity we have $1\in\mathcal{Q}$.
      \[ 1(a+b\sqrt{D})=(a+b\sqrt{D})1=a+b\sqrt{D} \]
    \item For associativity of addition we have
      \[ (a+b\sqrt{D})((c+d\sqrt{D})+(e+f\sqrt{D}))
        = (a+c+e)+(b+d+f)\sqrt{D}=((a+b\sqrt{D})+(c+d\sqrt{D}))+(e+f\sqrt{D})\]
    \item For commutativity of addition we have
      \[ (a+b\sqrt{D})+(c+d\sqrt{D})=(a+c)+(b+d)\sqrt{D}=(c+a)+(d+b)\sqrt{D}=
        (c+d\sqrt{D})+(a+b\sqrt{D})\]
    \item Additive inverse we have
      \[ (a+b\sqrt{D})+(-a-b\sqrt{D})=0\]
    \item For associativity of multiplication we have
      \begin{align*}
        ((a+b\sqrt{D})(c+d\sqrt{D}))(e+f\sqrt{D}) &= ((ac+Dbd)+(ad+bc)\sqrt{D})(e+f\sqrt{D})\\
                                                  &= (ace+Dbd+adfD+bcfD)+(ade+bce+acf+Dbdf)\sqrt{D}\\
                                                  &= (a+b\sqrt{D})((ce+Ddf)+(cf+de)\sqrt{D})\\
                                                  &= (a+b\sqrt{D})(c+d\sqrt{D})((e+f\sqrt{D}))
      \end{align*}
    \item For commutativity of multiplication we have
      \[ (a+b\sqrt{D})(b+d\sqrt{D})=(ac+Dbd)+(ad+bc)\sqrt{D}
        = (ca+Ddb)+(da+cb)\sqrt{D} = (b+d\sqrt{D})(a+b\sqrt{D})\]
    \item For multiplicative inverse if $a,b\neq 0$ we have
      \[ (a+b\sqrt{D})\frac{a-b\sqrt{D}}{a^2-Db^2}
        =\frac{a^2-Db^2+(ab-ba)\sqrt{D}}{a^2-Db^2}=\frac{a^2-Db^2}{a^2-Db^2}=1 \]
    \end{itemize}
    Therefore $\mathcal{Q}(\sqrt{D})$ is a field.
  \item Suppose that $n$ is prime. Then for any $x\in \bb{Z}_n\setminus\{0\}$ the $\gcd(x,n)=1$
    which from a prior homework means that $x\in\bb{Z}^X$ and as such is not a zero
    divisor. Thus $\bb{Z}_n$ is an integral domain if $n$ is prime.

    Otherwise suppose that $n$ is not prime. Then there exist $ab=n$
    such that $a,b\neq 1$ or $n$. However then $ab\equiv 0 \mod n$ which implies
    that $a\in \bb{Z}_n$ is a zero divisor and thus $\bb{Z}_n$ is not an
    integral domain.

    Therefore $\bb{Z}_n$ is an integral domain if and only if $n$ is prime.
  \end{enumerate}
\end{proof}

\sk

\begin{problem}[7.4]
  \begin{enumerate}
  \item Prove that the set
    \[ \bb{Z}[\sqrt{D}]=\{a+b\sqrt{D}|a,b\in\bb{Z}\}\]
    is a subring of $\mathcal{Q}(\sqrt{D})$ and
    $\bb{Z}[\sqrt{D}]$ is an integral domain.
  \item Define the norm function $N:\mathcal{Q}(\sqrt{D})\rightarrow\mathcal{Q}$ by
    \[ N(a+b\sqrt{D})=a^2-Db^2\]
    Prove that $N(\alpha\beta)=N(\alpha)N(\beta)$ for all $\alpha,\beta\in\mathcal{Q}(\sqrt{D})$.
  \item Show that for any $\alpha\in\bb{Z}[\sqrt{D}]$, $\alpha$ is a unit of
    $\bb{Z}[\sqrt{D}]$ if, and only if, $N(\alpha)=\pm 1$.
  \end{enumerate}
\end{problem}

\begin{proof}
  \begin{enumerate}
  \item Both $0$ and $1$ are integers so they are contained in $\bb{Z}[\sqrt{D}]$.
    Then if $a,b,c,d\in \bb{Z}$ then for addition
    $(a+b\sqrt{D})+(c+d\sqrt{D})=(a+b)+(c+d)\sqrt{D}\in\bb{Z}[\sqrt{D}]$.\\
    For multiplication we have
    $(a+b\sqrt{D})(c+d\sqrt{D})=(ac+Dbd)+(ad+bc)\sqrt{D}\in\bb{Z}[\sqrt{D}]$
    Finally if $-a,-b\in\bb{Z}$ so it is closed under addition, multiplication,
    and additive inverses. Therefore $\bb{Z}[\sqrt{D}]$ is a subring of
    $\mathcal{Q}(\sqrt{D})$. Moreover because $\mathcal{Q}(\sqrt{D})$ is a field
    $\bb{Z}[\sqrt{D}]$ there will be no zero divisors. As such $\bb{Z}[\sqrt{D}]$
    is an integral domain.
  \item Let $\alpha=a+b\sqrt{D}$ and $\beta=c+d\sqrt{D}$. Then
    \begin{align*}
      N(\alpha\beta)&=N((ac+Dbd)+(ad+bc)\sqrt{D})\\
           &=(ac+Dbd)^2-D(ad+bc)^2\\
           &=a^2c^2+2acDbd+D^2b^2d^2-Da^2d^2-D2adbc-Db^2c^2\\
           &=(a^2-Db^2)(c^2-Dd^2)\\
           &=N(\alpha)N(\beta)
    \end{align*}
  \item Note that if we view $a+b\sqrt{D}\in\bb{Z}[\sqrt{D}]$ in $\mathcal{Q}(\sqrt{D})$
    then $(a+b\sqrt{D})^{-1}=\frac{a-b\sqrt{D}}{a^2-Db^2}$. Now suppose that
    $N(a+b\sqrt{D})=\pm 1$. Then the denominator of $\frac{a-b\sqrt{D}}{a^2-Db^2}$
    is $\pm 1$ which means that $(a+b\sqrt{D})^{-1}$  has integer coefficients and
    as such is a unit.

    Otherwise suppose that $(a+b\sqrt{D})$ is a unit. Then
    $\frac{a}{a^2-Db^2},\frac{b}{a^2-Db^2}\in\bb{Z}$. However this can only
    occur if the denominator $(a^2-Db^2)=N(a+b\sqrt{D})=\pm 1$.

    Therefore $a+b\sqrt{D}$ is a unit if and only if $N(a+b\sqrt{D})=\pm 1$.
  \end{enumerate}
\end{proof}

\sk

\begin{problem}[7.5]
  Let $R$ be a ring. For any $a,b\in R$, if $1-ab$ is a unit, then so is $1-ba$.
\end{problem}

\begin{proof}
  Let $(1-ab)^{-1}=u$. Then
  \begin{align*}
    (1-bua)(1-ba) &= 1-ba+bua(1-ba)\\
                  &= 1-ba+bu(a-aba)\\
                  &= 1-ba+bu(1-ab)a\\
                  &= 1-ba+ba
  \end{align*}
  Which means that the inverse of $(1-ba)^{-1}=1+bua$ and as such $1-ba$ is a
  unit if $1-ab$ is.
\end{proof}

\sk

\begin{problem} 
  Compute the commutator subgroup of $S_4$.
\end{problem}

The commutator subgroup of $S_4$ is $A_4$. To see this first note that
the elements of the commutator are of the form $\sigma\tau\sigma^{-1}\tau^{-1}$ which
implies that there are an even number of transpositions since the number
will add and any cancellations will occur in pairs. Thus $[S_4,S_4]\subseteq A_4$.

Next note that $A_4$ is generated by 3-cycles. For a given 3-cycle
$(i\ j\ k)$ we can decompose it as $(i\ j)(i\ k)(i\ j)(i\ k)$
which means that $(i\ j\ k)\in [S_4,S_4]$. However this implies that
$A_4\subseteq [S_4,S_4]$ and therefore $A_4=[S_4,S_4]$.


\sk

%%%%%%%%%%%%%%%%%%%%%%%%%%%%%%%%%%%%%%%%%%%%%%%%%%%%%%%%%%%%%%%%%%%%%%%%%%%%%
\end{document}
