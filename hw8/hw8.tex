\documentclass[10pt]{article}
\usepackage[utf8]{inputenc}
\usepackage{amscd}
\usepackage{amsmath}
\usepackage{amssymb}
\usepackage{amsthm}
\usepackage{listings}
\usepackage{enumerate}

\textwidth=15cm \textheight=22cm \topmargin=0.5cm \oddsidemargin=0.5cm \evensidemargin=0.5cm

\newcommand{\sk}{\vskip 10mm}
\newcommand{\bb}[1]{\mathbb{#1}}
\newcommand{\ra}{\rightarrow}

\theoremstyle{plain}
\newtheorem{problem}{Problem}
\newtheorem{lemma}{Lemma}[problem]

\theoremstyle{remark}
\newtheorem{tpart}{}[problem]
\newtheorem*{ppart}{}

\begin{document}

\begin{problem}[6.1]
  \begin{enumerate}
  \item If $K\trianglelefteq G$ and $G/K$ are solvable, then $G$ is solvable.
  \item Prove that $S_n$ is not solvable for $n\geq 5$.
  \end{enumerate}
\end{problem}

\begin{proof}
  
\end{proof}

\sk

\begin{problem}[6.2]
  A finite group $G$ is solvable if, and only if, every composition factor
  of a composition series of $G$ is cyclic of prime order.
\end{problem}

\begin{proof}
  
\end{proof}

\sk

\begin{problem}[6.3]
  \begin{enumerate}
  \item Let $G$ be a group and $\phi:M(X)\rightarrow G$ a monoid homomorphism which
    satisfies
    \[ \phi(s^{-1})=\phi(s)^{-1}\ \text{for all}\ s\in S \]
    then for any $w\in M(X)$, $\phi(w)=\phi(r(w))$
  \item Let $S$ be a set, $R$ a subset of $F(S)$, $G$ a group, $\phi:S\rightarrow G$
    a function and $\widetilde{\phi}:F(S)\rightarrow G$ the induced group homomorphism.
    If $\widetilde{\phi}(r)=e$ for all $r\in R$, then there exists a homomorphism
    $\overline{\phi}$ from $\langle S|R\rangle$ to $G$ such that $\overline{\phi}\circ\pi\circ i = \phi$ where
    $i:S\rightarrow F(S)$ is the inclusion map, $\pi:F(S)\rightarrow \langle S|R\rangle$ is the natural
    surjection, and $\widetilde{\phi}:F(S)\rightarrow G$ is the homomorphism
    satisfying $\widetilde{\phi}\circ i=\phi$.
  \end{enumerate}
\end{problem}

\begin{proof}
  
\end{proof}

\sk

\begin{problem}[6.5.1]
  Using the Todd-Coxeter algorithm to determine and identify the group
  \[ G = \langle x,y| x^2=1,y^2=1, xyx=yxy\rangle \]
\end{problem}

\begin{proof}
  
\end{proof}

\sk

\begin{problem}[7.2]
  \begin{enumerate}
  \item Prove that $R^X$ is a group under the multiplication of $R$.
  \item Prove that $Z(R)\cap R^X=\emptyset$.
  \end{enumerate}
\end{problem}

\begin{proof}
  
\end{proof}

\sk

\begin{problem}[7.3]
  \begin{enumerate}
  \item Find the set of all zero divisors of the commutative ring $C([0,1])$
    defined in example $7.3$. Determine the $C([0,1])^X$.
  \item Let $D\in\mathcal{Q}$ such that the equation $x^2=D$ has no solution
    $x\in\mathcal{Q}$. Prove that the set
    \[ \mathcal{Q}(\sqrt{D})=\{a+b\sqrt{D}|a,b\in\mathcal{Q}\}\]
    forms a field under the ordinary addition and multiplication of
    complex numbers.
   \item Prove $\bb{Z}_n$ is an integral domain if, and only if, $n$ is prime.
  \end{enumerate}
\end{problem}

\begin{proof}
  
\end{proof}

\sk

\begin{problem}[7.4]
  \begin{enumerate}
  \item Prove that the set
    \[ \bb{Z}[\sqrt{D}]=\{a+b\sqrt{D}|a,b\in\bb{Z}\}\]
    is a subring of $\mathcal{Q}(\sqrt{D})$ and
    $\bb{Z}[\sqrt{D}]$ is an integral domain.
  \item Define the norm function $N:\mathcal{Q}(\sqrt{D})\rightarrow\mathcal{Q}$ by
    \[ N(a+b\sqrt{D})=a^2-Db^2\]
    Prove that $N(\alpha\beta)=N(\alpha)N(\beta)$ for all $\alpha,\beta\in\mathcal{Q}(\sqrt{D})$.
  \item Show that for any $\alpha\in\bb{Z}[\sqrt{D}]$, $\alpha$ is a unit of
    $\bb{Z}[\sqrt{D}]$ if, and only if, $N(\alpha)=\pm 1$.
  \end{enumerate}
\end{problem}

\begin{proof}
  
\end{proof}

\sk

\begin{problem}[7.5]
  Let $R$ be a ring. For any $a,b\in R$, if $1-ab$ is a unit, then so is $1-ba$.
\end{problem}

\begin{proof}
  
\end{proof}

\sk

\begin{problem} 
  Compute the commutator subgroup of $S_4$.
\end{problem}

\begin{proof}
  
\end{proof}

\sk

%%%%%%%%%%%%%%%%%%%%%%%%%%%%%%%%%%%%%%%%%%%%%%%%%%%%%%%%%%%%%%%%%%%%%%%%%%%%%
\end{document}
