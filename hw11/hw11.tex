\documentclass[10pt]{article}
\usepackage[utf8]{inputenc}
\usepackage{amscd}
\usepackage{amsmath}
\usepackage{amssymb}
\usepackage{amsthm}
\usepackage{listings}
\usepackage{enumerate}

\textwidth=15cm \textheight=22cm \topmargin=0.5cm \oddsidemargin=0.5cm \evensidemargin=0.5cm

\newcommand{\sk}{\vskip 10mm}
\newcommand{\bb}[1]{\mathbb{#1}}
\newcommand{\ra}{\rightarrow}

\theoremstyle{plain}
\newtheorem{problem}{Problem}
\newtheorem{lemma}{Lemma}[problem]

\theoremstyle{remark}
\newtheorem{tpart}{}[problem]
\newtheorem*{ppart}{}

\begin{document}

 \begin{problem}[8.5]
  Let $R=\bb{Z}[\sqrt{-5}]$. Show that $2,3,1+\sqrt{-5},1-\sqrt{-5}$ are
  irreducibles of $R$ and no two of which are associate in $R$, and that
  $6=2\cdot 3=(1+\sqrt{-5})(1-\sqrt{-5})$ are two distinct factorizations of
  $6$ into irreducibles in $R$. So $R$ is not a UFD.
\end{problem}

\begin{proof}
  
\end{proof}

\sk

\begin{problem}[9.1]
  Prove that every irreducible element of a UFD is a prime.
\end{problem}

\begin{proof}
  Let $R$ be a UFD and $r\in R$ irreducible. Then consider $a,b\in R$ such that
  $r|ab$. This implies that $cr=ab$ for some $c\in R$. As $R$ is a UFD take the
  factorization for both sides and we get $t_1\cdots t_sr=p_1\cdots p_nq_1\cdots q_m$.
  As $r$ is irreducible and factorizations are unique it must be that
  $r$ is an associate of something on the right. Thus either
  $r|p_i$ or $r|q_j$ it then follows that $r|a$ or $r|b$ respectively
  which implies that $r$ is in fact prime.
\end{proof}

\sk


\begin{problem}[9.3]
  Give an example of a UFD which is not a PID.
\end{problem}

\begin{proof}
  Consider $\bb{Z}[x]$. This is a UFD because $\bb{Z}$ is a UFD. However
  the ideal $\langle x^2-1,x\rangle$ cannot be generated by a single polynomial. Therefore
  $\bb{Z}[x]$ is a PID which is not a UFD.
\end{proof}

\sk


\begin{problem}[9.4]
  \begin{enumerate}
  \item Determine whether the following polynomials are irreducible in
    the rings indicated and prove your assertions. For those that are reducible,
    determine their factorization into irreducibles.
    \begin{enumerate}
    \item $x^3+x+1$ in $\bb{Z}_3[x]$.
    \item $x^4+1$ in $\bb{Z}_5[x]$.
    \item $x^4+10x^2+1$ in $\bb{Z}[x]$.
    \item $x^4-4x^3+6$ in $\bb{Z}[x]$.
    \item $x^6+30x^5-15x^3+6x-120$ in $\bb{Z}[x]$.
    \item $x^2+y^2+xy+1$ in $\bb{Q}[x,y]$.
    \end{enumerate}
  \item Prove that the following polynomials are irreducible in $\bb{Z}[x]$.
    \begin{enumerate}
    \item $x^4+4x^3+6x^2+2x+1$ (Substitute $x-1$ for $x$).
    \item $\frac{(x+2)^p-2^p}{x}$ where $p$ is an odd prime.
    \item $\prod_1^n(x-i) - 1$, where $n\in\bb{Z}_{>0}$
    \end{enumerate}
  \item Find all irreducible polynomials of degree $\leq 3$ in $\bb{Z}_2[x]$,
    and the same for $\bb{Z}_3[x]$.
  \item Prove that if $n$ is composite number, then $\sum_0^{n-1}x^{n-1}$ is
    reducible over $\bb{Z}$.
  \end{enumerate}
\end{problem}

\begin{proof}
  
\end{proof}

\sk


\begin{problem}[9.5]
  Let $R$ be a PID and $a,b\in R$. Prove that if $a,b$ are relatively
  prime, then $(a)+(b)=R$, and $a^i,b^j$ are relatively prime for all
  $i,j\in\bb{Z}_{>0}$.
\end{problem}

\begin{proof}
  
\end{proof}

\sk


\begin{problem}[9.6]
  \begin{enumerate}
  \item Let $F$ be a finite field of order $q$ and $f(x)$ a polynomial of degree
    $n$. Prove that the quotient ring $F[x]/(f(x))$ has $q^n$ elements.
  \item Show that $f(x)=x^3+x+1$ is irreducible in $\bb{Z}_2[x]$ and that
    $K=\bb{Z}_2/(f(x))$ is a field. Find a generator of the cyclic group
    $K^X$.
  \end{enumerate}
\end{problem}

\begin{proof}
  
\end{proof}

\sk


\begin{problem}[G4]
  Let $G=\text{GL}(2,\bb{F}_p)$ be the group of invertible $2\times 2$ matrices
  with entries in the finite field $\bb{F}_p$, where $p$ is prime.
  \begin{enumerate}
  \item Show that $G$ has order $(p^2-1)(p^2-p)$.
  \item Show that for $p=2$ the group $G$ is isomorphic to the symmetric group
    $S_3$.
  \end{enumerate}
\end{problem}

\begin{proof}
  
\end{proof}

\sk


\begin{problem}[G5]
  Let $G$ be the group of units of the ring $\bb{Z}/247\bb{Z}$.
  \begin{enumerate}
  \item Determine the order of $G$.
  \item Determine the structure of $G$ (as in the classification theorem for
    finitely generated abelian groups).
    (Hint: Use the Chinese Remainder Theorem).
  \end{enumerate}
\end{problem}

\begin{proof}
  
\end{proof}

\sk


\begin{problem}[G8]
  List all abelian groups of order 8 up to isomorphism. Identify which groups on
  your list is isomorphic to each of the following groups of order $8$. Justify
  your answer.
  \begin{enumerate}
  \item $(\bb{Z}/15\bb{Z})^*=$ the group of units of the ring $\bb{Z}/15\bb{Z}$.
  \item The roots of the equation $z^8-1=0$ in $\bb{C}$.
  \item $\bb{F}_8^+=$ the additive group of the field $\bb{F}_8$ with
    eight elements.
  \end{enumerate}
\end{problem}

\begin{proof}
  
\end{proof}

\sk


\begin{problem}[R4]
  Let $\bb{F}$ be a field and let $R=\bb{F}[X,Y]$ be the ring of polynomials
  in $X$ and $Y$ with coefficients from $\bb{F}$.
  \begin{enumerate}
  \item Show that $M=\langle X+1,Y-2\rangle$ is a maximal ideal of $R$.
  \item Show that $P=\langle X+Y+1\rangle$ is a prime ideal of $R$.
  \item Is $P$ a maximal ideal of $R$. Justify your answer.
  \end{enumerate}
\end{problem}

\begin{proof}
  
\end{proof}

\sk


\begin{problem}[R6]
  Let $R$ be a commutative ring with identity and let $I$ and $J$
  be ideals of $R$.
  \begin{enumerate}
  \item Define
    \[ (I:J)=\{r\in R|rx\in I, \forall x\in J\} \]
    Show that $(I:J)$ is an ideal of $R$ containing $I$.
  \item Show that if $P$ is a prime ideal of $R$ and $x\notin P$,
    then $(P:\langle x\rangle)=P$, where $\langle x\rangle$ denotes the
    principal ideal generated by $x$.
  \end{enumerate}
\end{problem}

\begin{proof}
  
\end{proof}

\sk


\begin{problem}[R7]
  Let $R$ be a commutative ring with identity, and let $I$ and $J$ be ideals
  of $R$.
  \begin{enumerate}
  \item Define what is meant by the sum $I+J$ and the product $IJ$ of
    the ideals $I$ and $J$.
  \item If $I$ and $J$ are distinct maximal ideals, show that $I+J=R$ and
    $I\cap J=IJ$.
  \end{enumerate}
\end{problem}

\begin{proof}
  
\end{proof}

%%%%%%%%%%%%%%%%%%%%%%%%%%%%%%%%%%%%%%%%%%%%%%%%%%%%%%%%%%%%%%%%%%%%%%%%%%%%%
\end{document}
