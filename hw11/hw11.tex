\documentclass[10pt]{article}
\usepackage[utf8]{inputenc}
\usepackage{amscd}
\usepackage{amsmath}
\usepackage{amssymb}
\usepackage{amsthm}
\usepackage{listings}
\usepackage{enumerate}

\textwidth=15cm \textheight=22cm \topmargin=0.5cm \oddsidemargin=0.5cm \evensidemargin=0.5cm

\newcommand{\sk}{\vskip 10mm}
\newcommand{\bb}[1]{\mathbb{#1}}
\newcommand{\ra}{\rightarrow}
\newcommand{\mm}[4]{\left(\begin{array}{cc}  #1 &  #2 \\ #3 & #4\end{array}\right)}

\theoremstyle{plain}
\newtheorem{problem}{Problem}
\newtheorem{lemma}{Lemma}[problem]

\theoremstyle{remark}
\newtheorem{tpart}{}[problem]
\newtheorem*{ppart}{}

\begin{document}

 \begin{problem}[8.5]
  Let $R=\bb{Z}[\sqrt{-5}]$. Show that $2,3,1+\sqrt{-5},1-\sqrt{-5}$ are
  irreducibles of $R$ and no two of which are associate in $R$, and that
  $6=2\cdot 3=(1+\sqrt{-5})(1-\sqrt{-5})$ are two distinct factorizations of
  $6$ into irreducibles in $R$. So $R$ is not a UFD.
\end{problem}

\begin{proof}
  
\end{proof}

\sk

\begin{problem}[9.1]
  Prove that every irreducible element of a UFD is a prime.
\end{problem}

\begin{proof}
  Let $R$ be a UFD and $r\in R$ irreducible. Then consider $a,b\in R$ such that
  $r|ab$. This implies that $cr=ab$ for some $c\in R$. As $R$ is a UFD take the
  factorization for both sides and we get $t_1\cdots t_sr=p_1\cdots p_nq_1\cdots q_m$.
  As $r$ is irreducible and factorizations are unique it must be that
  $r$ is an associate of something on the right. Thus either
  $r|p_i$ or $r|q_j$ it then follows that $r|a$ or $r|b$ respectively
  which implies that $r$ is in fact prime.
\end{proof}

\sk


\begin{problem}[9.3]
  Give an example of a UFD which is not a PID.
\end{problem}

\begin{proof}
  Consider $\bb{Z}[x]$. This is a UFD because $\bb{Z}$ is a UFD. However
  the ideal $\langle x^2-1,x\rangle$ cannot be generated by a single polynomial. Therefore
  $\bb{Z}[x]$ is a PID which is not a UFD.
\end{proof}

\sk


\begin{problem}[9.4]
  \begin{enumerate}
  \item Determine whether the following polynomials are irreducible in
    the rings indicated and prove your assertions. For those that are reducible,
    determine their factorization into irreducibles.
    \begin{enumerate}
    \item $x^3+x+1$ in $\bb{Z}_3[x]$.
    \item $x^4+1$ in $\bb{Z}_5[x]$.
    \item $x^4+10x^2+1$ in $\bb{Z}[x]$.
    \item $x^4-4x^3+6$ in $\bb{Z}[x]$.
    \item $x^6+30x^5-15x^3+6x-120$ in $\bb{Z}[x]$.
    \item $x^2+y^2+xy+1$ in $\bb{Q}[x,y]$.
    \end{enumerate}
  \item Prove that the following polynomials are irreducible in $\bb{Z}[x]$.
    \begin{enumerate}
    \item $x^4+4x^3+6x^2+2x+1$ (Substitute $x-1$ for $x$).
    \item $\frac{(x+2)^p-2^p}{x}$ where $p$ is an odd prime.
    \item $\prod_1^n(x-i) - 1$, where $n\in\bb{Z}_{>0}$
    \end{enumerate}
  \item Find all irreducible polynomials of degree $\leq 3$ in $\bb{Z}_2[x]$,
    and the same for $\bb{Z}_3[x]$.
  \item Prove that if $n$ is composite number, then $\sum_0^{n-1}x^{i}$ is
    reducible over $\bb{Z}$.
  \end{enumerate}
\end{problem}

\begin{proof}
  \begin{enumerate}
  \item
    \begin{enumerate}
    \item $x^3+x+1=(x+2)(x^2+x+2)$
    \item $x^4+1=(x^2+2)(x^2+3)$
    \item The polynomial has no roots. As such it must be the product of two
      degree two irreducibles. However the only way this could occur is
      if $a+b=10$ and $ab=1$ which cannot happen with integers.
      Thus $x^4+10x^2+1$ is irreducible.
    \item This polynomial is irreducible by Eisenstein's Criterion with
      $p=1$.
    \item This polynomial is irreducible by Eisenstein's Criterion with
      $p=3$.
    \item First consider the polynomial in $\bb{Z}[x,y]/(y-1)$. Then the
      polynomial we get is $x^2+x+1$. The roots of the original are then
      forced to be $\pm 1$ for $x$. However this is not the case and as
      such by Gauss' Lemma the polynomial is irreducible.

      Consider $\bb{Z}[x,y]/(y-1)$. Get $x^2+x+1$ root must be either $\pm 1$.
      Use Gauss' lemma.
    \end{enumerate}
  \item
    \begin{enumerate}
    \item Substitute $x-1$ for $x$ in the polynomial and it simplifies to
      $x^4-2x+2$. Then it is irreducible by Eisenstein's Criterion with $p=2$.
    \item 
    \item 
    \end{enumerate}
  \item For $\bb{Z}_2[x]$ we have
    \[  x, x^{3} + x, x + 1, x^{3} + x + 1, x^{2} + x + 1, x^{3} + x^{2} + x + 1\]
    For $\bb{Z}_3[x]$ we have
    \[  y, y^{3} + y, 2 y^{3} + y, y + 1, y^{3} + y + 1, 2 y^{3} + y + 1, y + 2, y^{3} + y + 2, 2 y^{3} + y + 2, 2 y, y^{3} + 2 y, 2 y^{3} + 2 y, 2 y + 1, y^{3} + 2 y + 1, 2 y^{3} + 2 y + 1, 2 y + 2, y^{3} + 2 y + 2, 2 y^{3} + 2 y + 2, y^{2} + 1, y^{3} + y^{2} + 1, 2 y^{3} + y^{2} + 1, y^{2} + y + 2, y^{3} + y^{2} + y + 2, 2 y^{3} + y^{2} + y + 2, y^{2} + 2 y + 2, y^{3} + y^{2} + 2 y + 2, 2 y^{3} + y^{2} + 2 y + 2, 2 y^{2} + 2, y^{3} + 2 y^{2} + 2, 2 y^{3} + 2 y^{2} + 2, 2 y^{2} + y + 1, y^{3} + 2 y^{2} + y + 1, 2 y^{3} + 2 y^{2} + y + 1, 2 y^{2} + 2 y + 1, y^{3} + 2 y^{2} + 2 y + 1, 2 y^{3} + 2 y^{2} + 2 y + 1\]
  \end{enumerate}
  \end{proof}

\sk


\begin{problem}[9.5]
  Let $R$ be a PID and $a,b\in R$. Prove that if $a,b$ are relatively
  prime, then $(a)+(b)=R$, and $a^i,b^j$ are relatively prime for all
  $i,j\in\bb{Z}_{>0}$.
\end{problem}

\begin{proof}
  Let $R$ be a PID and $a,b\in R$ such that $a$ and $b$ are relatively prime.
  Then $1$ is a gcd of $a$ and $b$. However this means that
  there exists $\alpha,\beta\in R$ such that $\alpha a+\beta b = 1\in (a)+(b)$ (Prop 8.11) implying that
  $(a)+(b)=1$.

  Now we will show that $a^i$ and $b$ are relatively prime. We have the case where
  $i=1$ be assumption. Next assume that we have $\alpha a^i+b=1$. Then if we square
  both sides we get
  \[ \alpha^2a^{2i}+\beta^2b^2+\alpha a^i\beta b+\beta b \alpha a^i \beta b = (\alpha^2a^{i-1})a^{i+1}+(\beta b+\alpha a^i \beta +\alpha a^i\beta)b=1 \]
  which shows that $a^{i+1}$ is relatively prime to $b$ with the assumption that
  $a^i$ is relatively prime to $b$. Therefore $a^i$ is relatively prime to $b$ where
  $i\in \bb{Z}_{>0}$. To get arbitrary powers of $b$ just set $a:=b$ and $b:=a^i$
  and repeat the process.

  Therefore if $a,b$ are relatively prime then $(a)+(b)=R$ and $a^i,b^j$ are relatively
  prime for $i,j\in\bb{Z}_{>0}$.
\end{proof}

\sk


\begin{problem}[9.6]
  \begin{enumerate}
  \item Let $F$ be a finite field of order $q$ and $f(x)$ a polynomial of degree
    $n$. Prove that the quotient ring $F[x]/(f(x))$ has $q^n$ elements.
  \item Show that $f(x)=x^3+x+1$ is irreducible in $\bb{Z}_2[x]$ and that
    $K=\bb{Z}_2/(f(x))$ is a field. Find a generator of the cyclic group
    $K^X$.
  \end{enumerate}
\end{problem}

\begin{proof}
  \begin{enumerate}
  \item We proceed by induction. Suppose that $\deg f=0$. Then $(f)=F[x]$
    implies that $F[x]/(f)\cong F[x]/F[x]=\{0\}$ which shows that the order is one.

    Now assume that if $\deg g \leq n$ then $F[x]/(f)$ is of order $q^{\deg g}$.
    Then suppose that $\deg f = n+1$. In the case where $f$ is reducible
    by Proposition 9.23 we have
    \[ f=f_1^{n_1}\cdots f_k^{n_k} \]
    where $\sum n_i=n+1$ and $n_i\leq n$ and that
    $F[x]/(f) \cong F[x]/(f_1^{n_1}\times\cdots\times f_k^{n_k})$
    The order of $F[x]/(f_i^{n_ii})$ is $q^{n_i}$ by our inductive hypothesis
    which implies that $|F[x]/(f)|=q^{n_1}\cdots q^{n_k}=q^{n+1}$.

    However if $f$ is irreducible, then $F[x]/(f)$ is the $n+1$th degree
    field extension and which the field with $q^{n+1}$ elements.

    Therefore if $\deg f=n$ then the order of $F[x]/(f)$ is $q^n$ where
    $F$ is the field with $q$ elements.
  \end{enumerate}
\end{proof}

\sk


\begin{problem}[G4]
  Let $G=\text{GL}(2,\bb{F}_p)$ be the group of invertible $2\times 2$ matrices
  with entries in the finite field $\bb{F}_p$, where $p$ is prime.
  \begin{enumerate}
  \item Show that $G$ has order $(p^2-1)(p^2-p)$.
  \item Show that for $p=2$ the group $G$ is isomorphic to the symmetric group
    $S_3$.
  \end{enumerate}
\end{problem}

\begin{proof}
  \begin{enumerate}
  \item For the first column there are $p^2$  possibilities to choose. However
    both values cannot be zero so we end up with $p^2-1$ choices for
    the first column. For the second column there are also $p^2$ choices
    but we must avoid the $p$ multiples of the first column.  As such
    there are $p^2-p$ choices for the second column and as such
    the order of $G$ is $(p^2-1)(p^2-p)$.
  \item The order of $G$ is 6. The only groups of order 6 are
    $\bb{Z}_6$ and $S_3$. However  we have
    \[ \mm{1}{1}{0}{1}\mm{1}{0}{1}{1}=\mm{0}{1}{1}{1}\neq \mm{1}{1}{1}{0}=\mm{1}{0}{1}{1} \mm{1}{1}{0}{1} \]
    which implies that $G$ is not abelian. Therefore $G\cong S_3$.
  \end{enumerate}
\end{proof}

\sk


\begin{problem}[G5]
  Let $G$ be the group of units of the ring $\bb{Z}/247\bb{Z}$.
  \begin{enumerate}
  \item Determine the order of $G$.
  \item Determine the structure of $G$ (as in the classification theorem for
    finitely generated abelian groups).
    (Hint: Use the Chinese Remainder Theorem).
  \end{enumerate}
\end{problem}

\begin{proof}
  \begin{enumerate}
  \item The order of $G$ is $\varphi(247)=\varphi(13\cdot 19)=(12)(18)=216$.
  \item By the Chinese Remainder Theorem we have that $\bb{Z}_{247}\cong\bb{Z}_{13}\times\bb{Z}_{19}$.
    This implies that $\bb{Z}_{247}^X=(\bb{Z}_{13}\times\bb{Z}_{19})^X$. For each component is
    $2$. Thus the largest order in $\bb{Z}_{247}^X$ is $\text{lcm}(12,18)=36$. By the
    structure theorem for finite abelian groups there the only possible structure for
    $\bb{Z}_{247}$ is $\bb{Z}_{36}\oplus\bb{Z}_6$.
  \end{enumerate}
\end{proof}

\sk


\begin{problem}[G8]
  List all abelian groups of order 8 up to isomorphism. Identify which groups on
  your list is isomorphic to each of the following groups of order $8$. Justify
  your answer.
  \begin{enumerate}
  \item $(\bb{Z}/15\bb{Z})^*=$ the group of units of the ring $\bb{Z}/15\bb{Z}$.
  \item The roots of the equation $z^8-1=0$ in $\bb{C}$.
  \item $\bb{F}_8^+=$ the additive group of the field $\bb{F}_8$ with
    eight elements.
  \end{enumerate}
\end{problem}

\begin{proof}
  By the structure theorem for finite abelian groups there are three
  possibilities for groups of order 8. They are
  \[ \bb{Z}_8,\bb{Z}_2\oplus\bb{Z}_4, \bb{Z}_2\oplus\bb{Z}_2\oplus\bb{Z}_2 \]

  \begin{enumerate}
  \item The elements of $\bb{Z}_{15}^X$ are $\{1,2,4,7,8,11,13,14\}$. The
    orders respectively are $1,4,2,4,4,2,4,2$ which implies that
    the structure of the group is $\bb{Z}_2\oplus\bb{Z}_4$.
  \item The group of roots is generated by $e^{\frac{\pi i}{4}}$. As such the
    structure is $\bb{Z}_8$.
  \item The field with 8 elements has characteristic two. As such
    all elements in the additive group will have order 2. Therefore
    the structure is $\bb{Z}_2\oplus\bb{Z}_2\oplus\bb{Z}_2$.
  \end{enumerate}
\end{proof}

\sk


\begin{problem}[R4]
  Let $\bb{F}$ be a field and let $R=\bb{F}[X,Y]$ be the ring of polynomials
  in $X$ and $Y$ with coefficients from $\bb{F}$.
  \begin{enumerate}
  \item Show that $M=\langle X+1,Y-2\rangle$ is a maximal ideal of $R$.
  \item Show that $P=\langle X+Y+1\rangle$ is a prime ideal of $R$.
  \item Is $P$ a maximal ideal of $R$. Justify your answer.
  \end{enumerate}
\end{problem}

\begin{proof}
  \begin{enumerate}
  \item In $F[x,y]/\langle x+1,y-2\rangle$ we have that $x+1=0$ and $y-2=0$ which
    implies that $x=-1$ and $y=2$ in the quotient. As such any polynomial
    can be reduced to an element in $F$ and as such the quotient is a field.
    Therefore $\langle x+1,y-2\rangle$ is a maximal ideal.
  \item As above in $F[x,y]/P$ we get the relation that $X+Y+1=0$. Since
    $F[x,y]$ is an integral domain the only way to get zero divisors in
    $F[x,y]/P$ would be if there are two nonzero polynomials that multiplied
    to $X+Y+1$. However this cannot happen because the degree of $X+Y+1=0$.
    Therefore the quotient $F[x,y]/P$ is an integral domain and as such $P$
    is a prime ideal.
  \item It is not a maximal ideal. Note that $X\notin P$ which means that $\langle X,X+Y+1\rangle=\langle X,Y+1\rangle$
    is an distinct ideal containing $P$.  However this ideal is not all of $F[x,y]$.
    Therefore $P$ is not maximal.
  \end{enumerate}
\end{proof}

\sk


\begin{problem}[R6]
  Let $R$ be a commutative ring with identity and let $I$ and $J$
  be ideals of $R$.
  \begin{enumerate}
  \item Define
    \[ (I:J)=\{r\in R|rx\in I, \forall x\in J\} \]
    Show that $(I:J)$ is an ideal of $R$ containing $I$.
  \item Show that if $P$ is a prime ideal of $R$ and $x\notin P$,
    then $(P:\langle x\rangle)=P$, where $\langle x\rangle$ denotes the
    principal ideal generated by $x$.
  \end{enumerate}
\end{problem}

\begin{proof}
  \begin{enumerate}
  \item Let $f\in I$. Then $fg\in I$ for all $g\in J$ as $I$ is an ideal. Therefore
    $I\subseteq I:J$.
  \item We know that $P\subseteq P:\langle x\rangle$ by the previous part of the problem. Let
    $f\in P:\langle x\rangle$. Then $fx\in P$ however since $P$ is prime and $x\notin P$ it follows
    that $f\in P$ and as such $P:\langle x\rangle\subseteq P$.

    Therefore $P=P:\langle x\rangle$ when $P$ is prime and $x\notin P$.
  \end{enumerate}
\end{proof}

\sk


\begin{problem}[R7]
  Let $R$ be a commutative ring with identity, and let $I$ and $J$ be ideals
  of $R$.
  \begin{enumerate}
  \item Define what is meant by the sum $I+J$ and the product $IJ$ of
    the ideals $I$ and $J$.
  \item If $I$ and $J$ are distinct maximal ideals, show that $I+J=R$ and
    $I\cap J=IJ$.
  \end{enumerate}
\end{problem}

\begin{proof}
  \begin{enumerate}
  \item For a commutative ring we have
    \[ I+J = \{f+g|f\in I,g\in J\}\]
    and
    \[ IJ = \{fg|f\in I, g\in J \}\]
  \item Since $I\subseteq I+J$, $I,J$ are distinct, and $I$ is maximal it follows that $I+J=R$.

    Next we'll show that $I\cap J=IJ$. Let $fg\in IJ$ where $f\in I$ and $g\in J$. Then $fg\in I$ and $fg\in J$ which implies that
    $fg\in I\cap J$ and as such $IJ\subseteq I\cap J$.

    Now suppose that $f\in I\cap J$. Then since $I,J$ are maximal there exists $g\in I$ and $h\in J$ such that
    $1=g+h$. Multiply by $f$ to get $f=fg+fh$. We know that $fg\in IJ$ since $f\in J$ and $g\in I$. We also have
    that $fh \in IJ$ as $f\in I$ and $h\in J$. This implies that $fg+fh=f\in IJ$ and as such $I\cap J \subseteq IJ$.

    Therefore $I\cap J = IJ$ and $I+J = R$.
  \end{enumerate}
\end{proof}

%%%%%%%%%%%%%%%%%%%%%%%%%%%%%%%%%%%%%%%%%%%%%%%%%%%%%%%%%%%%%%%%%%%%%%%%%%%%%
\end{document}
