\documentclass[10pt]{article}
\usepackage[utf8]{inputenc}
\usepackage{amscd}
\usepackage{amsmath}
\usepackage{amssymb}
\usepackage{amsthm}
\usepackage{listings}
\usepackage{enumerate}

\textwidth=15cm \textheight=22cm \topmargin=0.5cm \oddsidemargin=0.5cm \evensidemargin=0.5cm

\newcommand{\sk}{\vskip 10mm}
\newcommand{\bb}[1]{\mathbb{#1}}
\newcommand{\ra}{\rightarrow}
\newcommand{\mm}[4]{\begin{array}[cc] #1 & #2 \\ #3 & #4\end{array}}

\theoremstyle{plain}
\newtheorem{problem}{Problem}
\newtheorem{lemma}{Lemma}[problem]

\theoremstyle{remark}
\newtheorem{tpart}{}[problem]
\newtheorem*{ppart}{}

\begin{document}

 \begin{problem}[8.5]
  Let $R=\bb{Z}[\sqrt{-5}]$. Show that $2,3,1+\sqrt{-5},1-\sqrt{-5}$ are
  irreducibles of $R$ and no two of which are associate in $R$, and that
  $6=2\cdot 3=(1+\sqrt{-5})(1-\sqrt{-5})$ are two distinct factorizations of
  $6$ into irreducibles in $R$. So $R$ is not a UFD.
\end{problem}

\begin{proof}
  
\end{proof}

\sk

\begin{problem}[9.1]
  Prove that every irreducible element of a UFD is a prime.
\end{problem}

\begin{proof}
  Let $R$ be a UFD and $r\in R$ irreducible. Then consider $a,b\in R$ such that
  $r|ab$. This implies that $cr=ab$ for some $c\in R$. As $R$ is a UFD take the
  factorization for both sides and we get $t_1\cdots t_sr=p_1\cdots p_nq_1\cdots q_m$.
  As $r$ is irreducible and factorizations are unique it must be that
  $r$ is an associate of something on the right. Thus either
  $r|p_i$ or $r|q_j$ it then follows that $r|a$ or $r|b$ respectively
  which implies that $r$ is in fact prime.
\end{proof}

\sk


\begin{problem}[9.3]
  Give an example of a UFD which is not a PID.
\end{problem}

\begin{proof}
  Consider $\bb{Z}[x]$. This is a UFD because $\bb{Z}$ is a UFD. However
  the ideal $\langle x^2-1,x\rangle$ cannot be generated by a single polynomial. Therefore
  $\bb{Z}[x]$ is a PID which is not a UFD.
\end{proof}

\sk


\begin{problem}[9.4]
  \begin{enumerate}
  \item Determine whether the following polynomials are irreducible in
    the rings indicated and prove your assertions. For those that are reducible,
    determine their factorization into irreducibles.
    \begin{enumerate}
    \item $x^3+x+1$ in $\bb{Z}_3[x]$.
    \item $x^4+1$ in $\bb{Z}_5[x]$.
    \item $x^4+10x^2+1$ in $\bb{Z}[x]$.
    \item $x^4-4x^3+6$ in $\bb{Z}[x]$.
    \item $x^6+30x^5-15x^3+6x-120$ in $\bb{Z}[x]$.
    \item $x^2+y^2+xy+1$ in $\bb{Q}[x,y]$.
    \end{enumerate}
  \item Prove that the following polynomials are irreducible in $\bb{Z}[x]$.
    \begin{enumerate}
    \item $x^4+4x^3+6x^2+2x+1$ (Substitute $x-1$ for $x$).
    \item $\frac{(x+2)^p-2^p}{x}$ where $p$ is an odd prime.
    \item $\prod_1^n(x-i) - 1$, where $n\in\bb{Z}_{>0}$
    \end{enumerate}
  \item Find all irreducible polynomials of degree $\leq 3$ in $\bb{Z}_2[x]$,
    and the same for $\bb{Z}_3[x]$.
  \item Prove that if $n$ is composite number, then $\sum_0^{n-1}x^{i}$ is
    reducible over $\bb{Z}$.
  \end{enumerate}
\end{problem}

\begin{proof}
  \begin{enumerate}
  \item
    \begin{enumerate}
    \item $x^3+x+1=(x+2)(x^2+x+2)$
    \item $x^4+1=(x^2+2)(x^2+3)$
    \item No roots, must be product of two irreducibles of deg 2. But $a+b=10$
      and $ab=1$ which cannot occur. \textbf{Make this pretty.}
    \item Eisenstein $p=2$
    \item Eisenstein $p=3$
    \item Consider $\bb{Z}[x,y]/(y-1)$. Get $x^2+x+1$ root must be either $\pm 1$.
      Use gauss's lemma.
    \end{enumerate}
  \item
    \begin{enumerate}
    \item Sub $x-1$ for $x$ and that simplifies to $x^4-2x+2$ by Eisenstein with
      $p=2$ is irreducible and thus the rest of it is as well.
    \item 
    \item 
    \end{enumerate}

  \end{enumerate}
  \end{proof}

\sk


\begin{problem}[9.5]
  Let $R$ be a PID and $a,b\in R$. Prove that if $a,b$ are relatively
  prime, then $(a)+(b)=R$, and $a^i,b^j$ are relatively prime for all
  $i,j\in\bb{Z}_{>0}$.
\end{problem}

\begin{proof}
  Let $R$ be a PID and $a,b\in R$ such that $a$ and $b$ are relatively prime.
  Then $1$ is a gcd of $a$ and $b$. However this means that
  there exists $\alpha,\beta\in R$ such that $\alpha a+\beta b = 1\in (a)+(b)$ (Prop 8.11) implying that
  $(a)+(b)=1$.

  Now we will show that $a^i$ and $b$ are relatively prime. We have the case where
  $i=1$ be assumption. Next assume that we have $\alpha a^i+b=1$. Then if we square
  both sides we get
  \[ \alpha^2a^{2i}+\beta^2b^2+\alpha a^i\beta b+\beta b \alpha a^i \beta b = (\alpha^2a^{i-1})a^{i+1}+(\beta b+\alpha a^i \beta +\alpha a^i\beta)b=1 \]
  which shows that $a^{i+1}$ is relatively prime to $b$ with the assumption that
  $a^i$ is relatively prime to $b$. Therefore $a^i$ is relatively prime to $b$ where
  $i\in \bb{Z}_{>0}$. To get arbitrary powers of $b$ just set $a:=b$ and $b:=a^i$
  and repeat the process.

  Therefore if $a,b$ are relatively prime then $(a)+(b)=R$ and $a^i,b^j$ are relatively
  prime for $i,j\in\bb{Z}_{>0}$.
\end{proof}

\sk


\begin{problem}[9.6]
  \begin{enumerate}
  \item Let $F$ be a finite field of order $q$ and $f(x)$ a polynomial of degree
    $n$. Prove that the quotient ring $F[x]/(f(x))$ has $q^n$ elements.
  \item Show that $f(x)=x^3+x+1$ is irreducible in $\bb{Z}_2[x]$ and that
    $K=\bb{Z}_2/(f(x))$ is a field. Find a generator of the cyclic group
    $K^X$.
  \end{enumerate}
\end{problem}

\begin{proof}
  \begin{enumerate}
  \item We proceed by induction. Suppose that $\deg f=0$. Then $(f)=F[x]$
    implies that $F[x]/(f)\cong F[x]/F[x]=\{0\}$ which shows that the order is one.

    Now assume that if $\deg g \leq n$ then $F[x]/(f)$ is of order $q^{\deg g}$.
    Then suppose that $\deg f = n+1$. In the case where $f$ is reducible
    by Proposition 9.23 we have
    \[ f=f_1^{n_1}\cdots f_k^{n_k} \]
    where $\sum n_i=n+1$ and $n_i\leq n$ and that
    $F[x]/(f) \cong F[x]/(f_1^{n_1}\times\cdots\times f_k^{n_k})$
    The order of $F[x]/(f_i^{n_ii})$ is $q^{n_i}$ by our inductive hypothesis
    which implies that $|F[x]/(f)|=q^{n_1}\cdots q^{n_k}=q^{n+1}$.

    However if $f$ is irreducible, then $F[x]/(f)$ is the $n+1$th degree
    field extension and which the field with $q^{n+1}$ elements.

    Therefore if $\deg f=n$ then the order of $F[x]/(f)$ is $q^n$ where
    $F$ is the field with $q$ elements.
  \end{enumerate}
\end{proof}

\sk


\begin{problem}[G4]
  Let $G=\text{GL}(2,\bb{F}_p)$ be the group of invertible $2\times 2$ matrices
  with entries in the finite field $\bb{F}_p$, where $p$ is prime.
  \begin{enumerate}
  \item Show that $G$ has order $(p^2-1)(p^2-p)$.
  \item Show that for $p=2$ the group $G$ is isomorphic to the symmetric group
    $S_3$.
  \end{enumerate}
\end{problem}

\begin{proof}
  
\end{proof}

\sk


\begin{problem}[G5]
  Let $G$ be the group of units of the ring $\bb{Z}/247\bb{Z}$.
  \begin{enumerate}
  \item Determine the order of $G$.
  \item Determine the structure of $G$ (as in the classification theorem for
    finitely generated abelian groups).
    (Hint: Use the Chinese Remainder Theorem).
  \end{enumerate}
\end{problem}

\begin{proof}
  \begin{enumerate}
  \item
  \item \[ \left(1, 1\right), \left(2, 36\right), \left(3, 18\right), \left(4, 18\right), \left(5, 36\right), \left(6, 36\right), \left(7, 12\right), \left(8, 12\right), \left(9, 9\right), \left(10, 18\right), \left(11, 12\right), \left(12, 6\right),\]
    \[ \left(14, 18\right), \left(15, 36\right), \left(16, 9\right), \left(17, 18\right), \left(18, 4\right), \left(20, 12\right), \left(21, 36\right), \left(22, 18\right), \left(23, 18\right), \left(24, 36\right), \left(25, 18\right),\]
    \[ \left(27, 6\right), \left(28, 36\right), \left(29, 18\right), \left(30, 6\right), \left(31, 12\right), \left(32, 36\right), \left(33, 36\right), \left(34, 36\right), \left(35, 9\right), \left(36, 18\right), \left(37, 12\right),\]
    \[\left(40, 18\right), \left(41, 36\right), \left(42, 9\right), \left(43, 18\right), \left(44, 36\right), \left(45, 12\right), \left(46, 12\right), \left(47, 36\right), \left(48, 18\right), \left(49, 6\right), \left(50, 12\right),\]
    \[ \left(51, 18\right), \left(53, 18\right), \left(54, 36\right), \left(55, 9\right), \left(56, 6\right), \left(58, 12\right), \left(59, 36\right), \left(60, 36\right), \left(61, 9\right), \left(62, 18\right), \left(63, 36\right),\]
    \[ \left(64, 6\right), \left(66, 9\right), \left(67, 36\right), \left(68, 3\right), \left(69, 6\right), \left(70, 36\right), \left(71, 36\right), \left(72, 36\right), \left(73, 36\right), \left(74, 9\right), \left(75, 6\right),\]
    \[ \left(77, 2\right), \left(79, 18\right), \left(80, 36\right), \left(81, 9\right), \left(82, 18\right), \left(83, 12\right), \left(84, 12\right), \left(85, 36\right), \left(86, 36\right), \left(87, 3\right), \left(88, 6\right),\]
    \[ \left(89, 36\right), \left(90, 18\right), \left(92, 9\right), \left(93, 36\right), \left(94, 6\right), \left(96, 4\right), \left(97, 36\right), \left(98, 36\right), \left(99, 36\right), \left(100, 9\right), \left(101, 18\right),\]
    \[ \left(102, 12\right), \left(103, 6\right), \left(105, 18\right), \left(106, 12\right), \left(107, 6\right), \left(108, 18\right), \left(109, 36\right), \left(110, 36\right), \left(111, 36\right),\]
    \[ \left(112, 36\right), \left(113, 6\right), \left(115, 12\right), \left(116, 18\right), \left(118, 9\right), \left(119, 36\right), \left(120, 9\right), \left(121, 6\right), \left(122, 12\right), \left(123, 36\right), \]
    \[\left(124, 36\right),\left(125, 12\right), \left(126, 6\right), \left(127, 18\right), \left(128, 36\right), \left(129, 18\right), \left(131, 9\right), \left(132, 12\right), \left(134, 6\right), \left(135, 36\right),\]
    \[ \left(136, 36\right), \left(137, 36\right), \left(138, 36\right), \left(139, 9\right), \left(140, 6\right), \left(141, 12\right), \left(142, 18\right), \left(144, 3\right), \left(145, 12\right), \left(146, 18\right),\]
    \[ \left(147, 18\right), \left(148, 36\right), \left(149, 36\right), \left(150, 36\right), \left(151, 4\right), \left(153, 6\right), \left(154, 36\right), \left(155, 18\right), \left(157, 9\right), \left(158, 36\right),\]
    \[ \left(159, 3\right), \left(160, 6\right), \left(161, 36\right), \left(162, 36\right), \left(163, 12\right), \left(164, 12\right), \left(165, 18\right), \left(166, 18\right), \left(167, 36\right),\]
    \[ \left(168, 18\right), \left(170, 2\right), \left(172, 3\right), \left(173, 18\right), \left(174, 36\right), \left(175, 36\right), \left(176, 36\right), \left(177, 36\right), \left(178, 3\right), \left(179, 6\right),\]
    \[ \left(180, 36\right), \left(181, 18\right), \left(183, 6\right), \left(184, 36\right), \left(185, 18\right), \left(186, 18\right), \left(187, 36\right), \left(188, 36\right), \left(189, 12\right),\]
    \[ \left(191, 3\right), \left(192, 18\right), \left(193, 36\right), \left(194, 18\right), \left(196, 9\right), \left(197, 12\right), \left(198, 6\right), \left(199, 18\right), \left(200, 36\right),\]
    \[ \left(201, 12\right), \left(202, 12\right), \left(203, 36\right), \left(204, 18\right), \left(205, 18\right), \left(206, 36\right), \left(207, 18\right), \left(210, 12\right), \left(211, 18\right),\]
    \[ \left(212, 18\right), \left(213, 36\right), \left(214, 36\right), \left(215, 36\right), \left(216, 12\right), \left(217, 6\right), \left(218, 18\right), \left(219, 36\right), \left(220, 6\right),\]
    \[ \left(222, 18\right), \left(223, 36\right), \left(224, 18\right), \left(225, 18\right), \left(226, 36\right), \left(227, 12\right), \left(229, 4\right), \left(230, 18\right), \left(231, 18\right),\]
    \[ \left(232, 36\right), \left(233, 18\right), \left(235, 3\right), \left(236, 12\right), \left(237, 9\right), \left(238, 18\right), \left(239, 12\right), \left(240, 12\right), \left(241, 36\right),\]
    \[ \left(242, 36\right), \left(243, 18\right), \left(244, 18\right), \left(245, 36\right), \left(246, 2\right)\]
  \end{enumerate}
\end{proof}

\sk


\begin{problem}[G8]
  List all abelian groups of order 8 up to isomorphism. Identify which groups on
  your list is isomorphic to each of the following groups of order $8$. Justify
  your answer.
  \begin{enumerate}
  \item $(\bb{Z}/15\bb{Z})^*=$ the group of units of the ring $\bb{Z}/15\bb{Z}$.
  \item The roots of the equation $z^8-1=0$ in $\bb{C}$.
  \item $\bb{F}_8^+=$ the additive group of the field $\bb{F}_8$ with
    eight elements.
  \end{enumerate}
\end{problem}

\begin{proof}
  
\end{proof}

\sk


\begin{problem}[R4]
  Let $\bb{F}$ be a field and let $R=\bb{F}[X,Y]$ be the ring of polynomials
  in $X$ and $Y$ with coefficients from $\bb{F}$.
  \begin{enumerate}
  \item Show that $M=\langle X+1,Y-2\rangle$ is a maximal ideal of $R$.
  \item Show that $P=\langle X+Y+1\rangle$ is a prime ideal of $R$.
  \item Is $P$ a maximal ideal of $R$. Justify your answer.
  \end{enumerate}
\end{problem}

\begin{proof}
  
\end{proof}

\sk


\begin{problem}[R6]
  Let $R$ be a commutative ring with identity and let $I$ and $J$
  be ideals of $R$.
  \begin{enumerate}
  \item Define
    \[ (I:J)=\{r\in R|rx\in I, \forall x\in J\} \]
    Show that $(I:J)$ is an ideal of $R$ containing $I$.
  \item Show that if $P$ is a prime ideal of $R$ and $x\notin P$,
    then $(P:\langle x\rangle)=P$, where $\langle x\rangle$ denotes the
    principal ideal generated by $x$.
  \end{enumerate}
p\end{problem}

\begin{proof}
  
\end{proof}

\sk


\begin{problem}[R7]
  Let $R$ be a commutative ring with identity, and let $I$ and $J$ be ideals
  of $R$.
  \begin{enumerate}
  \item Define what is meant by the sum $I+J$ and the product $IJ$ of
    the ideals $I$ and $J$.
  \item If $I$ and $J$ are distinct maximal ideals, show that $I+J=R$ and
    $I\cap J=IJ$.
  \end{enumerate}
\end{problem}

\begin{proof}
  
\end{proof}

%%%%%%%%%%%%%%%%%%%%%%%%%%%%%%%%%%%%%%%%%%%%%%%%%%%%%%%%%%%%%%%%%%%%%%%%%%%%%
\end{document}
