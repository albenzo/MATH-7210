\documentclass[10pt]{article}
\usepackage[utf8]{inputenc}
\usepackage{amscd}
\usepackage{amsmath}
\usepackage{amssymb}
\usepackage{amsthm}
\usepackage{listings}
\usepackage{enumerate}

\textwidth=15cm \textheight=22cm \topmargin=0.5cm \oddsidemargin=0.5cm \evensidemargin=0.5cm

\newcommand{\sk}{\vskip 10mm}
\newcommand{\bb}[1]{\mathbb{#1}}
\newcommand{\ra}{\rightarrow}

\theoremstyle{plain}
\newtheorem{problem}{Problem}
\newtheorem{lemma}{Lemma}[problem]

\theoremstyle{remark}
\newtheorem{tpart}{}[problem]
\newtheorem*{ppart}{}

\begin{document}

\begin{problem}
  Let $f:A\ra B$. Then:
  \begin{itemize}
  \item[a)] $f$ is injective if and only if it has a left inverse.
  \item[b)] $f$ is surjective if and only if it has a right inverse.
  \item[c)] $f$ is bijective if and only if it has a left and right inverse.
  \item[d)] If $|A|=|B|=n\in \bb{Z}_{\geq0}$ then $f$ is injective if and only if f is
    surjective if and only if $f$ is bijective.
  \end{itemize}
\end{problem}

\begin{proof}\ \\
  \begin{itemize}
  \item[a)] Suppose that $f$ is injective. This implies that $f^{-1}(f(a))=\{a\}$ for all
    $a\in A$ since by the definition of injectivity if $f(a_0)=f(a_1)$ then $a_0=a_1$.
    Define $g:B\ra A$ as $g(b)=f^{-1}(a)$ where $b\in f(A)$. If $b\notin f(A)$
    then send it to any arbitrary $a\in A$. Then $g\circ f(a)=f^{-1}(f(a))=a$ which implies
    that $g\circ f=id_A$ and that $g$ is a left inverse of $f$.

    Now suppose that there exists a function $g:B\ra A$ such that $g\circ f = id_A$.
    Let $a_0,a_1\in A$ such that $f(a_0)=f(a_1)$. Then
    $g\circ f(a_0)=a_0=a_1=g\circ f(a_1)$ which is the definition of injectivity.

    Therefore $f$ is injective if and only if it has a left inverse.
  \item[b)] Suppose that $f$ is surjective. Then given $b\in B$ there exists an $a\in A$
    such that $f(a)=b$. Define a function $g:B\ra A$ via $g(b)=a$ where $a$ fulfills
    $f(a)=b$. Then $f\circ g(b)=b$ by definition which implies that $g$ is a right
    inverse of $f$.

    Now suppose that there exists a function $g:B\ra A$ such that $f\circ g= id_B$. Then
    given $b\in B$ let $a=g(b)$. Then $f(a)=f\circ g(b)=b$. Since this holds for all
    elements of $b$ $f$ is surjective.

    Therefore $f$ is surjective if and only if it has a right inverse.
  \item[c)] Suppose that $f$ is a bijection. Then it is both injective and surjective
    which by the previous statements in the proposition implies that $f$ has both
    a left and right inverse.

    Otherwise suppose that $f$ has a left and right inverse. Then via the previous
    statements in the proposition we know that $f$ is both injective and surjective and
    thus a bijection.

    To show that the left and right inverse are unique let $g,h$ be a left and right
    inverse for $f$ respectively. Then
    \[g=g\circ id_B=g\circ(f\circ h)=(g\circ f)\circ h=id_A\circ h=h\]

    Therefore $f$ is a bijection if and only if it has a left and right inverse.
    Moreover these inverses are equal.
  \item[d)]
    Suppose that $|A|=|B|=1$. Then there is only one function $f:A\ra B$
    defined as $f(a_0)=b_0$. As such the function $f$ is injective, surjective and
    bijective.

    Next assume for sets of size $n$ that a function is injective if and only if it
    is surjective if and only if it is bijective. Let $|A|=|B|=n+1$.

    First consider the case where $f$ is bijective. Then by definition $f$ is also
    injective and surjective.

    Next, if $f$ is injective then take the pair $(a_n,b_n)$, where
    $f(a_n)=b_n$. Since $f$ is injective the restriction $f|_{a_n}$ will be well
    defined since no other element of $A$ maps to $b_n$. However via our inductive
    hypothesis this implies that $f|_{A\setminus\{a_n\}}$ is also bijective and surjective.
    Reintroduce $(a_n,b_n)$ to $f|_{A\setminus\{a_n\}}$ and this will maintain injectivity and
    surjectivity since no other element will map to $b_n$ and $b_n$ is mapped to
    by $a_n$.

    Finally suppose that $f$ is surjective. Then given any $b\in B$ we can find an $a\in A$
    that maps to it. There must be at least one $b_i\in B$ such that $f^{-1}(b_i)=a_i$
    since if the pullback for every element was greater than $1$ we would have
    $|A|>2|B|$ which contradicts our assumption. Now we can take the restriction
    $f_{A\setminus\{a_i\}}$ which will still be surjective. Which by our inductive
    hypothesis implies that $f_{A\setminus\{a_i\}}$ is injective and bijective. Reintroduce
    pair $(a_i,b_i)$ to $f$ and for the same reasoning as above we preserve injectivity,
    surjectivity, and bijectivity.

    Therefore via induction, if $|A|=|B|=n\in \bb{Z}_{>0}$ then a function $f:A\ra B$ is
    injective if and only if it is surjective if and only if it is bijective.
  \end{itemize}
\end{proof}

\sk

\begin{problem}
  Let $\sim$ be an equivalence relation of the set A. For any $a,b\in A$,
  \begin{itemize}
  \item[a)] $a\sim b$ if and only if $\bar{a}=\bar{b}$.
  \item[b)] if $\bar{a}\neq \bar{b}$, then $\bar{a}\cap\bar{b}=\phi$
  \end{itemize}
\end{problem}

\begin{proof}\ \\
  \begin{itemize}
  \item[a)] Suppose that $a\sim b$. Without loss of generality let $c\in\bar{a}$. Then
    $c\sim a$ which implies that $c\sim b$ by transitivity and as such $c\in\bar{b}$
    Therefore if $a\sim b$ then $\bar{a}=\bar{b}$.

    Now suppose that $\bar{a}=\bar{b}$. Since $a\in\bar{a}$ and $b\in\bar{b}$ by
    reflexivity we know that $a,b\in\bar{b}$ and as such $a\sim b$.

    Therefore $a\sim b$ if and only if $\bar{a}=\bar{b}$.
  \item[b)] Suppose that $\bar{a}\neq\bar{b}$ and that there existed a
    $c\in \bar{a}\cap\bar{b}$. Then $a\sim c$ and $c\sim b$ which would imply that
    $a\sim b$ by transitivity and that $\bar{a}=\bar{b}$ by the previous part of the
    proposition which is a contradiction.

    Therefore if $\bar{a}\neq\bar{b}$ then $\bar{a}\cap\bar{b}=\phi$
  \end{itemize}
\end{proof}

\sk

\begin{problem}
  Let $n$ be a fixed positive integer. Then
  \[ (\bb{Z}/n\bb{Z})^\times = \{ \bar{a}\in \bb{Z}/n\bb{Z}|1\leq a\leq n\ \text{and}\ a,n\ \text{are relatively prime}\}\]
\end{problem}

\begin{proof}
  Let $\bar{a}\in(\bb{Z}/n\bb{Z})^\times$. Then $\bar{a}$ has a multiplicative inverse
  $\bar{\alpha}$ such that $\bar{a}\bar{\alpha}=\bar{1}$. This means that
  $a\alpha=kn+1$ where $k\in\bb{Z}$. Rearrange and we get
  $\alpha a+kn=1$ which implies that the $\mathrm{gcd}(a,n)=1$.

  Otherwise suppose that $a,n$ are relatively prime. Then using the extended
  Euclidean algorithm we can get $\alpha,\beta\in\bb{Z}$ such that $\alpha a+\beta n = 1$.
  Rearrange to get $\alpha a = (-\beta)n +1$ and rewrite mod $n$ for
  $\bar{\alpha}\bar{a}=\bar{1}$ which implies that $\bar{a}\in(\bb{Z}/n\bb{Z})^\times$
\end{proof}

%%%%%%%%%%%%%%%%%%%%%%%%%%%%%%%%%%%%%%%%%%%%%%%%%%%%%%%%%%%%%%%%%%%%%%%%%%%%%
\end{document}
