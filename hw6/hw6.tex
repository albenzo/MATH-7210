\documentclass[10pt]{article}
\usepackage[utf8]{inputenc}
\usepackage{amscd}
\usepackage{amsmath}
\usepackage{amssymb}
\usepackage{amsthm}
\usepackage{listings}
\usepackage{enumerate}

\textwidth=15cm \textheight=22cm \topmargin=0.5cm \oddsidemargin=0.5cm \evensidemargin=0.5cm

\newcommand{\sk}{\vskip 10mm}
\newcommand{\bb}[1]{\mathbb{#1}}
\newcommand{\ra}{\rightarrow}

\theoremstyle{plain}
\newtheorem{problem}{Problem}
\newtheorem{lemma}{Lemma}[problem]

\theoremstyle{remark}
\newtheorem{tpart}{}[problem]
\newtheorem*{ppart}{}

\begin{document}

\begin{problem}[4.2]
  
\end{problem}

\begin{enumerate}
\item Let $G$ be a finite group and $H$ a subgroup of index $n$.
  Define $N:=\bigcap_{x\in G}xHx^{-1}$ which we know is a normal subgroup
  of $G$ contained in $H$ by a prior problem. Now let $G/N$ act
  on $G/H$ by $gN\cdot g'H\mapsto gg'H$. To see this is well defined
  let $g'H\in G/H$. Then
  \[ gN\cdot g'H=gg'NH=gg'H\]
  However this action is equivalent to a homomorphism
  $\varphi:G/N\rightarrow S_{|G/H|=n}$ which by the first isomorphism
  theorem implies that $G/N$ is isomorphic to some subgroup
  of $S_n$ and as such $|G/N|=|G:N|\bigr| n!$ completing the
  proof.
  
\item Let $G$ be a finite group where $p$ is the smallest
  prime factor of $|G|=n$. Let $H$ be a subgroup of $G$
  with index $p$. Then by problem $4.2.1$ there exists a
  subgroup $N\trianglelefteq G$ such that $N\leq H$ and
  $|G:N|\bigr| p!$. However $|G:N|$ cannot be less than $p$
  because if it were then with $|G|=|N||G:N|$ we would
  have $|G|$ divisible by a smaller prime. On the other
  hand $|G:N|$ cannot be larger than $p$. If it were
  then $pm\bigr||G|$ where $m$ is a product of numbers
  smaller than $p$ again contradicting that $p$ is the
  smallest prime that divides $|G|$.

  Thus $|G:N|=p$ which via Lagrange's Theorem
  gives us that $|H|=|N|$. However since $N\leq H$
  it must be the case that $N=H$.

  Therefore $H$ is a normal subgroup.
\item Let $G$ be a group and $H$ a subgroup of index 2.
  Then there are only two cosets for $H$ which are
  $H,gH$ for some  $g\in G\setminus H$. However since there are
  only two this implies that $gH= Hg$. Since this holds
  for all cosets of $H$ we have that $H$ is normal.

  Therefore any subgroup of index 2 is normal.
  
\item Let $N$ be a normal subgroup and $K$ a
  conjugacy class $K$ with some representative $k\in K$.
  If $K\cap N=\phi$ then we're done. Otherwise suppose
  that $K\cap N\neq \phi$. Then there is some $\alpha\in K\cap N$. Then
  $\alpha=gkg^{-1}$ for some $g\in G$. This implies that
  $g^{-1}\alpha g = k$ however since $\alpha\in N$ so is $g^{-1}\alpha g = k$.
  Therefore $k\in N$ and as such $K\subset N$.
\end{enumerate}

\sk

\begin{problem}[4.3]
  
\end{problem}

\begin{proof}
  
\end{proof}

\sk

\begin{problem}[4.4.1]
  
\end{problem}

\begin{proof}
  
\end{proof}

\sk

\begin{problem}[4.4.2]
  
\end{problem}

\begin{proof}
  
\end{proof}

\sk

\begin{problem}[4.4.3]
  
\end{problem}

\begin{proof}
  
\end{proof}

\sk

%%%%%%%%%%%%%%%%%%%%%%%%%%%%%%%%%%%%%%%%%%%%%%%%%%%%%%%%%%%%%%%%%%%%%%%%%%%%%
\end{document}
