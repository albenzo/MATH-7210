\documentclass[10pt]{article}
\usepackage[utf8]{inputenc}
\usepackage{amscd}
\usepackage{amsmath}
\usepackage{amssymb}
\usepackage{amsthm}
\usepackage{listings}
\usepackage{enumerate}

\textwidth=15cm \textheight=22cm \topmargin=0.5cm \oddsidemargin=0.5cm \evensidemargin=0.5cm

\newcommand{\sk}{\vskip 10mm}
\newcommand{\bb}[1]{\mathbb{#1}}
\newcommand{\ra}{\rightarrow}

\theoremstyle{plain}
\newtheorem{problem}{Problem}
\newtheorem{lemma}{Lemma}[problem]

\theoremstyle{remark}
\newtheorem{tpart}{}[problem]
\newtheorem*{ppart}{}

\begin{document}

\begin{problem}[7.6]
  \begin{enumerate}
  \item Let $F$ be a non-trivial field and $F[[x]]$ the set of all
    formal power series
    \[ f(x)=\sum_{n=0}^\infty a_nx^n \]
    where $a_i\in F$. Prove that $F[[x]]$ is an integral domain under the
    following addition and multiplication:
    \[ \sum_{n=0}^\infty a_nx^n+\sum_{n=0}^\infty b_nx^n
      =\sum_{n=0}^\infty (a_n+b_n)x^n \]
    and
    \[ \left(\sum_{n=0}^\infty a_nx^n\right)
      \left(\sum_{n=0}^\infty b_nx^n\right)
      =\sum_{n=0}^\infty\left(\sum_{i+j=n}a_ib_j\right)x^n \]
    Prove that $f(x)$ is a unit if and only if the constant term of $f(x)$ is
    non-zero.
  \item Let $R$ be a ring and $S$ a subring of $R$. Prove that $M_n(S)$ is a
    subring of $M_n(R)$ for any integer $n\geq 1$.
  \item Let $R$ be a commutative ring and $G$ a finite group.
    \begin{enumerate}
    \item Prove that $g$ is a unit of $R[G]$ for any $g\in G$.
    \item Prove or disprove that $G=R[G]^X$.
    \item If $S$ is a subring of $R$, then $S[G]$ is a subring of $R[G]$.
    \end{enumerate}
  \item Let $R$ be a commutative ring and $G$ be a finite group
    \begin{enumerate}
    \item Let $\Lambda=\sum_{g\in G}g$. Prove that $\Lambda$ is in the center of
      $R[G]$.
    \item Let $K$ be a conjugacy class in $G$. Prove that
      $k=\sum_{g\in K}g$ is in the center of $R[G]$.
    \item Let $K_1,\ldots,K_r$ be the conjugacy classes of $G$ and
      $k_i=\sum_{g\in K_i}g$ for $i=1,\ldots,r$. Prove that $x$ is in the center
      of $R[G]$ if, and only if, $x=\sum_{i=1}^ra_ik_i$ for some $a_i\in R$.
    \end{enumerate}
  \end{enumerate}
\end{problem}

\begin{proof}
  \begin{enumerate}
  \item In order to show that $F[[x]]$ is an integral domain we
    must show that addition is associative, commutative, has identity,
    and has inverse. That multiplication is associative, commutative,
    has identity, and that it distributes
    over addition. For each of the following statements consider
    $\sum_{n=0}^\infty a_n,\sum_{n=0}^\infty b_n,\sum_{n=0}^\infty c_n\in F[[x]]$.
    \begin{itemize}
    \item For additive identity we let all terms be $0$. Then
      \[ \sum_{n=0}^\infty a_nx^n +\sum_{n=0}^\infty 0x^n=\sum_{n=0}^\infty (a_n+0)x^n=\sum_{n=0}^\infty 0x^n+\sum_{n=0}^\infty a_nx^n=
        \sum_{n=0}^\infty a_nx^n \]
    \item For additive associativity we have
      \[ \left(\sum_{n=0}^\infty a_nx^n + \sum_{n=0}^\infty b_nx^n\right) \sum_{n=0}^\infty c_nx^n
        = \sum_{n=0}^\infty ((a_n+b_n)+c_n)x^n=\sum_{n=0}^\infty (a_n+(b_n+c_n))x^n\]\[
        = \sum_{n=0}^\infty a_nx^n+\left(\sum_{n=0}^\infty b_nx^n+\sum_{n=0}^\infty c_nx^n\right)\]
    \item For additive commutativity we have
      \[ \sum_{n=0}^\infty a_nx^n + \sum_{n=0}^\infty b_nx^n=\sum_{n=0}^\infty (a_n+b_n)x^n
        = \sum_{n=0}^\infty (b_n+a_n)x^n=\sum_{n=0}^\infty b_nx^n+\sum_{n=0}^\infty a_nx^n\]
    \item For additive inverse we have
      \[ \sum_{n=0}^\infty a_nx^n + \sum_{n=0}^\infty (-a_n)x^n = \sum_{n=0}^\infty (a_n-a_n)x^n=
        \sum_{n=0}^\infty 0x^n \]
    \item For multiplicative identity let $b_0=1$ and $b_n=0$ for $n>0$. Then
      \[ \sum_{n=0}^\infty a_nx^n\cdot\sum_{n=0}^\infty b_nx^n=  \sum_{n=0}^\infty\left(\sum_{i+j=n}a_ib_j\right)x^n\]
      However since $b_n=0$ for $n>0$ the only non-zero term in the inner sum
      will be when $i=n$ and $j=0$. Thus
      \[ \sum_{n=0}^\infty\left(\sum_{i+j=n}a_nb_n\right)x^n=\sum_{n=0}^\infty a_nx^n \]
    \item For associativity of multiplication we have
      \[ 
        \left(
          \sum_{n=0}^\infty a_nx^n\cdot \sum_{n=0}^\infty b_nx^n
        \right)\cdot \sum_{n=0}^\infty c_nx^n
        = 
        \left(
          \sum_{n=0}^\infty 
          \left(
            \sum_{i+j=n}a_ib_j
          \right)x^n
        \right)\cdot \sum_{n=0}^\infty c_nx^n\]\[
        = \sum_{n=0}^\infty 
        \left(
          \sum_{i+j=n}
          \left(
            \sum_{h+k=i}a_hb_k
          \right) c_j
        \right)x^n \]\[
        = \sum_{n=0}^\infty 
        \left(
          \sum_{i+j=n}a_i
          \left(
            \sum_{j+k=j} b_hc_k
          \right)
        \right)x^n
        =
        \sum_{n=0}^\infty a_nx^n\cdot 
        \left(
          \sum_{n=0}^\infty b_nx^n\cdot \sum_{n=0}^\infty c_nx^n
        \right)
      \]
    \item For commutativity of multiplication we have
      \[ \sum_{n=0}^\infty a_nx^n\cdot\sum_{n=0}^\infty b_nx^n=\sum_{n=0}^\infty 
        \left(
          \sum_{i+j=n}a_ib_j
        \right)x^n
        =\sum_{n=0}^\infty 
        \left(
          \sum_{i+j=n}b_ia_j
        \right)x^n
        = \sum_{n=0}^\infty b_nx^n\cdot \sum_{n=0}^\infty a_nx^n\]
    \item For distributivity we have
      \[ \sum_{n=0}^\infty c_nx^n
        \left(
          \sum_{n=0}^\infty a_nx^n+\sum_{n=0}^\infty b_nx^n
        \right)
        =
        \sum_{n=0}^\infty c_nx^n\cdot \sum_{n=0}^\infty (a_n+b_n)x^n \]\[
        =
        \sum_{n=0}^\infty 
        \left(
          \sum_{i+j=n}c_i(a_j+b_j)
        \right)x^n
        =
        \sum_{n=0}^\infty 
        \left(
          \sum_{i+j=n}c_ia_j+c_ib_j
        \right)x^n \]\[
        =
        \sum_{n=0}^\infty 
        \left(
          \sum_{i+j=n}c_ia_j+\sum_{i+j=n}c_ib_j
        \right)x^n
        =
        \sum_{n=0}^\infty
        \left(
          \sum_{i+j=n}c_ia_j
        \right)x^n
        +
        \sum_{n=0}^\infty
        \left(
          \sum_{i+j=n}c_ib_j
        \right)x^n \]\[
        =
        \sum_{n=0}^\infty c^nx^n\cdot\sum_{n=0}^\infty a_nx^n+\sum_{n=0}^\infty c_nx^n\cdot\sum_{n=0}^\infty b_nx^n
      \]
    \end{itemize}

    To show that $F[[x]]$ is an integral domain we must show that
    $F[[x]]$ has no zero-divisors. Suppose that $\sum_{n=0}^\infty a^nx^n\cdot\sum_{n=0}^\infty b_nx^n=0$.
    Let us assume that both $a_0$ and $b_0$ are not $0$.
    If the first $k$ terms consisted of only zeros we could factor out $x^{k}$ from
    each term. It would then reduce to multiplying two series with this assumption.
    Then the zero term would
    be $\sum_{i+j=0}a_ib_j=a_0b_0=0$. Since $F$ is a field either $a_0$ or $b_0$ are $0$.
    Without loss of generality assume that it's $b_0$. We then proceed by induction.
    Assume that $b_k=0$ for $k\leq m$. Then consider the term $b_{m+1}$. The sum
    for the $m+1$ coefficient is $\sum_{i+j=m+1}a_ib_j=0$. However by our inductive
    hypothesis $b_k=0$ for $k\leq m$. Which leaves us with $a_0b_{m+1}=0$. However
    we already assumed that $a_0\neq 0$ which implies that $b_{m+1}=0$.

    Therefore, by induction, all terms of $\sum_{n=0}^\infty b_nx^n$ are zero. As such
    the only way that $\sum_{n=0}^\infty a^nx^n\cdot\sum_{n=0}^\infty b_nx^n=0$ can hold is if one of
    the series is zero.

    Therefore since $F[[x]]$ is a commutative ring with no zero divisors it is
    an integral domain.

    Now we will show that a series $\sum_{n=0}^\infty a_nx^n$ is a unit if and only if
    $a_0\neq 0$. First suppose that $\sum_{n=0}^\infty a_nx^n$ is a unit. Then there
    is a series such that $\sum_{n=0}^\infty a_nx^n\cdot\sum_{n=0}^\infty b_nx^n=1$. However this
    implies that $a_0b_0=1$ and the only way this can occur is if $a_0$ is also
    a unit in $F$. However since $F$ is a field this will hold so long as $a_0\neq 0$.

    Now suppose that $\sum_{n=0}^\infty a_nx^n$ is a series such that $a_0\neq 0$. Then define
    $b_0=a_0^{-1}$ and $b_n=a_0^{-1}(-\sum_{i+j=n,i\neq 0}a_ib_j+a)$ for $n>0$. This will
    cause the sum $\sum_{i+j=n}a_ib_j=0$ for all $n>0$ as
    \[ \sum_{i+j=n}a_ib_j=a_0b_n+\sum_{i+j=n,i\neq 0}a_ib_j=-\sum_{i+j=n,i\neq 0}a_ib_j+\sum_{i+j=n,i\neq 0}a_ib_j=0\]
    Thus
    \[ \sum_{n=0}^\infty a_nx^n\cdot \sum_{n=0}^\infty b_nx^n=\sum_{n=0}^\infty\left(\sum_{i+j=n}a_ib_j\right)x^n =1\]
    Therefore $f\in F[[x]]$ is a unit if and only if the constant term is nonzero.
  \item Let $S$ be a subring of $R$. Then consider the matrices
    $A,B\in M_n(S)$. Let $A_{ij}$ denote the $ij$th entry in the
    matrix.
    
    For $A+B$ we have $(A+B)_{ij}=A_{ij}+B_{ij}\in S$ since $S$
    is a subring. Since this hold for all entries we have
    that $A+B\in M_n(S)$.

    For $AB$ we have $(AB)_{ij}=\sum_{k=1}^n A_{ik}B_{kj}$.
    However each $A_{ik},B_{kj}\in S$ since $S$ is a subring
    which implies that $(AB)_{ij}\in S$ and therefore $AB\in M_n(S)$.
    Since $M_n(S)$ is closed under multiplication and addition
    it is a subring of $M_n(R)$.
  \item
    \begin{enumerate}
    \item Let $g\in G$. Then it has an inverse $g^{-1}\in G$
      for which both $g,g^{-1}\in R[G]$. Thus we have
      $gg^{-1}=e=1\in R[G]$ which shows that $g$ is a unit of $R[G]$.
    \item Let $R$ be a ring with a non-trivial unit $r\in R$. Then
      $rg\cdot r^{-1}g^{-1}=1e$ which implies that $rg\notin G$ but that
      $rg\in R[G]^X$. Therefore $G$  may not equal $R[G]^X$.
    \item Let $S$ be a subring of $R$ and let $f:=\sum_{g\in G}a_gg,g:=\sum_{g\in G}b_gg\in S[G]$.

      For $f+g$ we have $f+g=\sum_{g\in G}(a_g+b_g)g$ and since $a_g+b_g\in S$ due to
      $S$ being a subring $f+g\in S[G]$.

      For $fg$ we have $fg=\sum_{k\in G}\left(\sum_{gh=k}a_gb_h\right)k$. However
      since $S$ is a subring $\sum_{gh=K}(a_gb_h)\in S$ as it is a sum of terms in $S$.
      It then follows that $fg\in S[G]$.

      Therefore if $S$ is a subring of $R$ then $S[G]$ is a subring of $R[G]$.
    \end{enumerate}
  \item
    \begin{enumerate}
    \item Let $\sum_{g\in G}a_gg\in R[G]$. Then
      \[ \sum_{g\in G}a_gg\cdot\sum_{g\in G}g=\sum_{k\in G}\left(\sum_{gh=k}a_g\right)k \]
      Since rings are associative under addition we have
      \[ \sum_{k\in G}\left(\sum_{gh=k}a_g\right)k= \sum_{k\in G}\left(\sum_{gh=k}a_h\right)k
        = \sum_{g\in G}g\cdot\sum_{g\in G}a_gg\sum\]
      Therefore $\Lambda$ is in the center of $R[G]$.
    \item Let $\sum_{g\in G}a_gg\in R[G]$. In order for $\sum_{g\in K}g\in Z(R[G])$ it must be
      the case that $\sum_{hk=g}a_hb_k=\sum_{hk=g}b_ha_k$ where $b_h=1$ if $h\in K$ and
      $b_h=0$ otherwise. Let $k\in K$. Then $hkh^{-1}=k'\in K$ which implies that
      $hk=k'h$. Since the $b_k=b_{k'}$ we know that $a_hb_k=b_{k'}a_h$ and we
      can map each term of $\sum_{hk=g}a_hb_k$ to an equal term $b_{k'}a_h$
      making the sums equal.
      Therefore
            \[ \sum_{g\in G}
        \left(
          \sum_{hk=g}a_hb_k
        \right)g
        =
        \sum_{g\in G}
        \left(
          \sum_{hk=g}b_{k'}a_h
        \right)g
        =
        \sum_{g\in K}g \cdot\sum_{g\in G}a_gg \]
      which shows that $\sum_{g\in G}g\in Z(R[G])$.
    \item Let $x=\sum_1^ra_ik_i$. Since $Z(R[G])$ is a subring of $R[G]$ it then
      follows that $x\in Z(R[G])$.

      Next suppose that $x\in Z$ and let $a=\sum_{g\in G}a_gg\in R[G]$. Since
      $xa=ax$ it follows that $\sum_{hk=g}x_ha_k=\sum_{hk=g}a_hx_k$ for all $g\in G$.
      Now suppose that $x$ was not a linear combination of $k_i$s. Then
      there would be a $K_i$ for which there would be $k,k'\in K_i$ such that
      $x_k\neq x_{k'}$. Since $k,k'$ are in the same conjugacy class it follows
      that there exists a $g\in G$ such that $gk=k'g$. However this means that
      \[ \sum_{hf=g}a_hx_f\neq \sum_{hf=g} x_hx_f \]
      since when we switch sides we will swap out $x_k$ for $x_{k'}$ making the
      sums inequal which is a contradiction.

      Therefore $x\in Z(R[G])$ is and only if $x=\sum_1^ra_ik_i$.
    \end{enumerate}
  \end{enumerate}
\end{proof}

\sk

\begin{problem}[7.7]
  For any nonzero integers $a,b,$ prove that
  $(a,b)=(\gcd(a,b)),(a)\cap(b)=(\text{lcm}(a,b))$
  and that $(a)(b)=(ab)$.
\end{problem}

\begin{proof}
  \begin{itemize}
  \item Let $na+mb\in (a,b)$. There exist $\alpha,\beta\in\bb{Z}$ such that
    $\alpha \gcd(a,b)=a$ and $\beta\gcd(a,b)=b$.
    Then $na+mb=n\alpha\gcd(a,b)+m\beta\gcd(a,b)\in(\gcd(a,b))$ which implies that
    $(a,b)\subset(\gcd(a,b))$.

    Now let $n\gcd(a,b)\in(\gcd(a,b))$. Then there exist $\alpha,\beta\in\bb{Z}$ such
    that $\alpha a+\beta b = \gcd(a,b)$. Thus $n\gcd(a,b)=n\alpha a+n\beta b\in (a,b)$ implying
    that $(\gcd(a,b))\subset(a,b)$.

    Therefore $(a,b)=(\gcd(a,b))$.
  \item Let $d=\text{lcm}(a,b)$. Then there exists $\alpha,\beta\in\bb{Z}$ such that
    $\alpha a= \beta b= d$. Then if $nd\in(d)$ we have $nd=n\alpha a = n\beta b$ which implies
    that $nd\in(a)\cap(b)$.

    Let $f\in(a)\cap(b)$. Then there exist $\alpha,\beta\in\bb{Z}$ such that $\alpha a = \beta b = f$.
    However this implies that both $a$ and $b$ divide $f$ and as such
    $d$ divides $f$ and there exists a $\delta\in\bb{Z}$ such that $\delta d=f$.
    It then follows that $f\in(d)$.

    Therefore $(a)\cap(b)=(\text{lcm}(a,b))$.
  \item Let $nab\in (ab)$. Then $na\cdot 1b\in(a)(b)$.

    Otherwise let $na\cdot mb\in(a)(b)$. Then $namb=(nm)(ab)\in(ab)$.

    Therefore $(a)(b)=(ab)$.
  \end{itemize}
\end{proof}

\sk

\begin{problem}[7.8]
  Let $G$ be a finite group and $R$ a commutative ring. Show that the map
  $\epsilon:R[G]\rightarrow R$ given by
  \[ \epsilon\left(\sum_{g\in G}a_gg\right)=\sum_{g\in G}a_g \]
  is a surjective ring homomorphism and $\ker \epsilon$ is the ideal generated by
  the set $\{g-e|g\in G\}$.
\end{problem}

\begin{proof}
  First we'll show that it preserves addition. Let $\sum_{g\in G}a_gg,\sum_{g\in G}b_gg\in R[G]$.
  Then
  \[ \epsilon
    \left(
      \sum_{g\in G}a_gg
    \right)+
    \epsilon
    \left(
      \sum_{g\in G}b_gg
    \right)
    = \sum_{g\in G}a_g+\sum_{g\in G}b_g
    = \sum_{g\in G}a_g+b_g
    = \epsilon
    \left(
      \sum_{g\in G}a_g+\sum_{g\in G}b_g
    \right)\]

  For multiplication we have
  \[
    \epsilon
    \left(
      \sum_{g\in G}a_gg
    \right)\cdot
    \epsilon
    \left(
      \sum_{g\in G}b_gg
    \right)
    =\sum_{g\in G}a_g\cdot\sum_{g\in G}b_g\]\[
    =\sum_{g\in G}
    \left(
      \sum_{hk=g}a_hb_k
    \right)
    =\epsilon
    \left(
      \sum_{g\in G}
      \left(
        \sum_{hk=g}a_hb_k
      \right)g
    \right)
    =\epsilon
    \left(
      \sum_{g\in G}a_gg\cdot\sum_{g\in G}b_gg
    \right)
  \]

  Therefore $\epsilon$ is a ring homomorphism.
  
  Let $a\in R$. Then $\epsilon(ag)=a$. Therefore the map $\epsilon$ is surjective.

  Now suppose that $\sum_{g\in G}a_gg\in \ker \epsilon$. Then $\sum_{g\in G}a_g=0$. It then
  follows that $\sum_{g\in G\setminus\{e\}}a_g+a_e=0$. Which implies that
  $a_e=-\sum_{g\in G\setminus\{e\}}a_g$. This implies that we can rewrite our original
  term as $\sum_{g\in G}a_g(g-e)$ which shows that $\ker \epsilon\subset\langle\{g-e|g\in G\}\rangle$.

  Now suppose that we have $\sum_{g\in G}a_g(g-e)=\sum_{g\in G}a_gg-\sum_{g\in G}a_g$. Then
  \[\epsilon
    \left(
      \sum_{g\in G}a_gg-\sum_{g\in G}a_g
    \right)
    = \sum_{g\in G}a_g-\sum_{g\in G}a_g=0\]
  Therefore the kernel of $\epsilon$ is the set generated by $\{g-e|g\in G\}$.
\end{proof}

\sk

\begin{problem}[7.10]
  \begin{enumerate}
  \item Prove that $x^2=0$ or $1$ for all $x\in\bb{Z}_4$
  \item Prove that the equation $x^2+y^2=3z^2$ has no nontrivial
    integer solution.
  \end{enumerate}
\end{problem}

\begin{proof}
  \begin{enumerate}
  \item For each case we have
    \begin{itemize}
    \item $0^2\equiv 0 \mod 4$
    \item $1^2\equiv 1 \mod 4$
    \item $2^2\equiv 0 \mod 4$
    \item $3^2\equiv 1 \mod 4$
    \end{itemize}
    Therefore the polynomial $x^2=0$ or $1$ for all $x\in\bb{Z}_4$.
  \item First we will verify that $x^2+y^2=3z^2$ has no non-trivial solutions in
    $\bb{Z}_3$. There scenarios are:
    \begin{itemize}
    \item If $x=y=1$ then $1+1=2$.
    \item If $x=0$, $y=1$ then $0+1=1$.
    \item If $x=1$, $y=0$ then $1+0=1$.
    \end{itemize}
    Therefore there are no non-trivial solutions of $x^2+y^2=3z^2$.

    Thus if a non-trivial integer solution does exist it must be that $x=3k$ and $y=3j$.
    Then the equation shifts to $9k^2+9j^2=3z^2$ which then gives us
    $3(k^2+j^2)=z^2$. This implies that $z=3h$. Rewrite again once
    more and we get $k^2+j^2=3h^2$. However this is the original equation
    which implies that we can factor out an arbitrary number of $3$s from
    $x,y,z$. Therefore the only solution where $x,y|3$ is the trivial solution.

    Therefore $x^2+y^2=3z^2$ has no non-trivial integer solutions.
  \end{enumerate}
\end{proof}

\sk

\begin{problem}[7.11]
  Let $D$ be a square-free integer and $I$ the ideal $(x^2-D)$ of $\bb{Q}[x]$.
  Prove that
  \[ \bb{Q}[x]/I\cong\bb{Q}(\sqrt{D}) \]
  as rings. Find all the ideals of $\bb{Q}[x]$ containing $I$.
\end{problem}

\begin{proof}
  First we show that $\bb{Q}[x]/I\cong\bb{Q}(\sqrt{D})$. Define
  $\varphi:\bb{Q}[x]\rightarrow \bb{Q}[\sqrt{D}]$ as $\varphi(f(x))=f(\sqrt{D})$. If
  $f(x)=\sum_0^nr_ix^i$ then $\varphi(f(x))=\sum_0^nr_i\sqrt{D}^{i}\in\bb{Q}[\sqrt{D}]$.
  To show that it is a homomorphism first consider addition for $f,g\in\bb{Q}[x]$.
  Then
  \[ \varphi(f+g)=\varphi(\sum_0^n(r_i+s_i)x^i)  = \sum_0^n(r_i+s_i)D^{i/2}=\sum_0^nr_iD^{i/2}+\sum_0^ns_iD^{i/2}=\varphi(f)+\varphi(g)\]

  For multiplication we have
  \[ \varphi(fg)=\varphi(\sum_0^n\left(\sum_{i+j=n}r_is_j\right)x^i)=\sum_0^n\left(\sum_{i+j=n}r_is_j\right)D^{i/2}
  = \left(\sum_0^nr_iD^{i/2}\right)\left(\sum_0^ns_iD^{i/2}\right)=\varphi(f)\varphi(g)\]

  To show that it is surjective consider $a+b\sqrt{D}\in\bb{Q}[\sqrt{D}]$.
  Then $\varphi(a+bx)=a+b\sqrt{D}$ which shows that $\varphi$ is a surjection.
  
  The kernel of $\varphi$ is the ideal $\langle(x^2-D)\rangle$. To see this let
  $f\in\langle(x^2-D)\rangle$. Then $f=g(x^2-D)$ which implies that
  \[ \varphi(f)=\varphi(g)\varphi(x^2-D)=\varphi(g)0=0\]
  Therefore $\langle(x^2-D)\rangle\subset\ker\varphi$. Now let $f\in\ker\varphi$. This implies that
  $f(\sqrt{D})=0$. However this also implies that $f(-\sqrt{D})=0$.
  It then follows that $f=g(x^2-D)$ for some $g$ and as such
  $f\in\langle(x^2-D)\rangle$. Therefore the kernel of $\varphi$ is the ideal $\langle(x^2-D)\rangle$.
  
  Therefore by the first isomorphism theorem of rings we have that
  $\bb{Q}[x]/I\cong \bb{Q}(\sqrt{D})$.

  The ideals that contain $\langle(x^2-D)\rangle$ will be those generated by
  the polynomials $f$ where there exists a $g$ such that $fg=(x^2-D)$.
  In this case the polynomials where this holds are $1,(x-\sqrt{D}),(x+\sqrt{D}),(x^2-D)$.
  However since $D$ is square-free the middle two do not exist inside $\bb{Q}[x]$
  and as such the only ideals containing $\langle(x^2-D)\rangle$ are itself and the whole
  ring.
  \end{proof}

  %%%%%%%%%%%%%%%%%%%%%%%%%%%%%%%%%%%%%%%%%%%%%%%%%%%%%%%%%%%%%%%%%%%%%%%%%%%%% 
\end{document}
