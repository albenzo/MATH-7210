\documentclass[10pt]{article}
\usepackage[utf8]{inputenc}
\usepackage{amscd}
\usepackage{amsmath}
\usepackage{amssymb}
\usepackage{amsthm}
\usepackage{listings}
\usepackage{enumerate}

\textwidth=15cm \textheight=22cm \topmargin=0.5cm \oddsidemargin=0.5cm \evensidemargin=0.5cm

\newcommand{\sk}{\vskip 10mm}
\newcommand{\bb}[1]{\mathbb{#1}}
\newcommand{\ra}{\rightarrow}

\theoremstyle{plain}
\newtheorem{problem}{Problem}
\newtheorem{lemma}{Lemma}[problem]

\theoremstyle{remark}
\newtheorem{tpart}{}[problem]
\newtheorem*{ppart}{}

\begin{document}

\begin{problem}[7.6]
  \begin{enumerate}
  \item Let $F$ be a non-trivial field and $F[[x]]$ the set of all
    formal power series
    \[ f(x)=\sum_{n=0}^\infty a_nx^n \]
    where $a_i\in F$. Prove that $F[[x]]$ is an integral domain under the
    following addition and multiplication:
    \[ \sum_{n=0}^\infty a_nx^n+\sum_{n=0}^\infty b_nx^n
      =\sum_{n=0}^\infty (a_n+b_n)x^n \]
    and
    \[ \left(\sum_{n=0}^\infty a_nx^n\right)
      \left(\sum_{n=0}^\infty b_nx^n\right)
      =\sum_{n=0}^\infty\left(\sum_{i+j=n}a_ib_j\right)x^n \]
    Prove that $f(x)$ is a unit if and only if the constant term of $f(x)$ is
    non-zero.
  \item Let $R$ be a ring and $S$ a subring of $R$. Prove that $M_n(S)$ is a
    subring of $M_n(R)$ for any integer $n\geq 1$.
  \item Let $R$ be a commutative ring and $G$ a finite group.
    \begin{enumerate}
    \item Prove that $g$ is a unit of $R[G]$ for any $g\in G$.
    \item Prove or disprove that $G=R[G]^X$.
    \item If $S$ is a subring of $R$, then $S[G]$ is a subring of $R[G]$.
    \end{enumerate}
  \item Let $R$ be a commutative ring and $G$ be a finite group
    \begin{enumerate}
    \item Let $\Lambda=\sum_{g\in G}g$. Prove that $\Lambda$ is in the center of
      $R[G]$.
    \item Let $K$ be a conjugacy class in $G$. Prove that
      $k=\sum_{g\in K}g$ is in the center of $R[G]$.
    \item Let $K_1,\ldots,K_r$ be the conjugacy classes of $G$ and
      $k_i=\sum_{g\in K_i}g$ for $i=1,\ldots,r$. Prove that $x$ is in the center
      of $R[G]$ if, and only if, $x=\sum_{i=1}^ra_ik_i$ for some $a_i\in R$.
    \end{enumerate}
  \end{enumerate}
\end{problem}

\begin{proof}
  \begin{enumerate}
  \item In order to show that $F[[x]]$ is an integral domain we
    must show that addition is associative, commutative, has identity,
    and has inverse. That multiplication is associative, commutative,
    has identity, and that it distributes
    over addition. For each of the following statements consider
    $\sum_{n=0}^\infty a_n,\sum_{n=0}^\infty b_n,\sum_{n=0}^\infty c_n\in F[[x]]$.
    \begin{itemize}
    \item For additive identity we let all terms be $0$. Then
      \[ \sum_{n=0}^\infty a_nx^n +\sum_{n=0}^\infty 0x^n=\sum_{n=0}^\infty (a_n+0)x^n=\sum_{n=0}^\infty 0x^n+\sum_{n=0}^\infty a_nx^n=
        \sum_{n=0}^\infty a_nx^n \]
    \item For additive associativity we have
      \[ \left(\sum_{n=0}^\infty a_nx^n + \sum_{n=0}^\infty b_nx^n\right) \sum_{n=0}^\infty c_nx^n
        = \sum_{n=0}^\infty ((a_n+b_n)+c_n)x^n=\sum_{n=0}^\infty (a_n+(b_n+c_n))x^n\]\[
        = \sum_{n=0}^\infty a_nx^n+\left(\sum_{n=0}^\infty b_nx^n+\sum_{n=0}^\infty c_nx^n\right)\]
    \item For additive commutativity we have
      \[ \sum_{n=0}^\infty a_nx^n + \sum_{n=0}^\infty b_nx^n=\sum_{n=0}^\infty (a_n+b_n)x^n
        = \sum_{n=0}^\infty (b_n+a_n)x^n=\sum_{n=0}^\infty b_nx^n+\sum_{n=0}^\infty a_nx^n\]
    \item For additive inverse we have
      \[ \sum_{n=0}^\infty a_nx^n + \sum_{n=0}^\infty (-a_n)x^n = \sum_{n=0}^\infty (a_n-a_n)x^n=
        \sum_{n=0}^\infty 0x^n \]
    \item For multiplicative identity let $b_0=1$ and $b_n=0$ for $n>0$. Then
      \[ \sum_{n=0}^\infty a_nx^n\cdot\sum_{n=0}^\infty b_nx^n=  \sum_{n=0}^\infty\left(\sum_{i+j=n}a_ib_j\right)x^n\]
      However since $b_n=0$ for $n>0$ the only non-zero term in the inner sum
      will be when $i=n$ and $j=0$. Thus
      \[ \sum_{n=0}^\infty\left(\sum_{i+j=n}a_nb_n\right)x^n=\sum_{n=0}^\infty a_nx^n \]
    \item For associativity of multiplication we have
      \[ 
        \left(
          \sum_{n=0}^\infty a_nx^n\cdot \sum_{n=0}^\infty b_nx^n
        \right)\cdot \sum_{n=0}^\infty c_nx^n
      = 
      \left(
        \sum_{n=0}^\infty 
        \left(
          \sum_{i+j=n}a_ib_j
        \right)x^n
      \right)\cdot \sum_{n=0}^\infty c_nx^n\]\[
      = \sum_{n=0}^\infty 
      \left(
        \sum_{i+j=n}
        \left(
          \sum_{h+k=i}a_hb_k
        \right) c_j
      \right)x^n \]\[
    = \sum_{n=0}^\infty 
    \left(
      \sum_{i+j=n}a_i
      \left(
       \sum_{j+k=j} b_hc_k
      \right)
    \right)x^n
    =
    \sum_{n=0}^\infty a_nx^n\cdot 
    \left(
      \sum_{n=0}^\infty b_nx^n\cdot \sum_{n=0}^\infty c_nx^n
    \right)
  \]
    \item For commutativity of multiplication we have
      \[ \sum_{n=0}^\infty a_nx^n\cdot\sum_{n=0}^\infty b_nx^n=\sum_{n=0}^\infty 
        \left(
          \sum_{i+j=n}a_ib_j
        \right)x^n
      =\sum_{n=0}^\infty 
        \left(
          \sum_{i+j=n}b_ia_j
        \right)x^n
      = \sum_{n=0}^\infty b_nx^n\cdot \sum_{n=0}^\infty a_nx^n\]
    \item For distributivity we have
      \[ \sum_{n=0}^\infty c_nx^n
        \left(
          \sum_{n=0}^\infty a_nx^n+\sum_{n=0}^\infty b_nx^n
        \right)
        =
        \sum_{n=0}^\infty c_nx^n\cdot \sum_{n=0}^\infty (a_n+b_n)x^n \]\[
        =
      \sum_{n=0}^\infty 
      \left(
        \sum_{i+j=n}c_i(a_j+b_j)
      \right)x^n
      =
      \sum_{n=0}^\infty 
    \left(
      \sum_{i+j=n}c_ia_j+c_ib_j
    \right)x^n \]\[
    =
    \sum_{n=0}^\infty 
    \left(
      \sum_{i+j=n}c_ia_j+\sum_{i+j=n}c_ib_j
    \right)x^n
    =
    \sum_{n=0}^\infty
    \left(
      \sum_{i+j=n}c_ia_j
    \right)x^n
    +
    \sum_{n=0}^\infty
    \left(
      \sum_{i+j=n}c_ib_j
    \right)x^n \]\[
    =
    \sum_{n=0}^\infty c^nx^n\cdot\sum_{n=0}^\infty a_nx^n+\sum_{n=0}^\infty c_nx^n\cdot\sum_{n=0}^\infty b_nx^n
  \]
    \end{itemize}

    To show that $F[[x]]$ is an integral domain we must show that
    $F[[x]]$ has no zero-divisors. Suppose that $\sum_{n=0}^\infty a^nx^n\cdot\sum_{n=0}^\infty b_nx^n=0$.
    Without loss of generality assume that both $a_0$ and $b_0$ are not $0$.
    If the first $k$ terms consisted of only zeros we could factor out $x^k$ and
    proceed with this assumption. Then the zero term would
    be $\sum_{i+j=0}a_ib_j=a_0b_0=0$. Since $F$ is a field either $a_0$ or $b_0$ are $0$.
    Without loss of generality assume that it's $b_0$. We then proceed by induction.
    Assume that $b_k=0$ for $k\leq m$. Then consider the term $b_{m+1}$. The sum
    for the $m+1$ coefficient is $\sum_{i+j=m+1}a_ib_j$. However by our inductive
    hypothesis $b_k=0$ for $k\leq m$. Which leaves us with $a_0b_{m+1}=0$. However
    we already assumed that $a_0\neq 0$ which implies that $b_{m+1}=0$.

    Therefore, by induction, all terms of $\sum_{n=0}^\infty b_nx^n$ are zero. As such
    the only way that $\sum_{n=0}^\infty a^nx^n\cdot\sum_{n=0}^\infty b_nx^n=0$ can hold is if one of
    the series is zero.

    Therefore since $F[[x]]$ is a commutative ring with no zero divisors it is
    an integral domain.

    Now we will show that a series $\sum_{n=0}^\infty a_nx^n$ is a unit if and only if
    $a_0\neq 0$. First suppose that $\sum_{n=0}^\infty a_nx^n$ is a unit. Then there
    is a series such that $\sum_{n=0}^\infty a_nx^n\cdot\sum_{n=0}^\infty b_nx^n=1$. However this
    implies that $a_0b_0=1$ and the only way this can occur is if $a_0$ is also
    a unit in $F$. However since $F$ is a field this will hold so long as $a_0\neq 0$.

    Now suppose that $\sum_{n=0}^\infty a_nx^n$ is a series such that $a_0\neq 0$. Then define
    $b_0=a_0^{-1}$ and $b_n=a_0^{-1}(-\sum_{i+j=n-1}a_ib_j)$ for $n>0$. This will
    cause the sum $\sum_{i+j=n}a_ib_j=0$ for all $n>0$. Thus
    \[ \sum_{n=0}^\infty a_nx^n\cdot \sum_{n=0}^\infty b_nx^n=\sum_{n=0}^\infty\left(\sum_{i+j=n}a_ib_j\right)x^n =1\]
    Therefore $f\in F[[x]]$ is a unit if and only if the constant term is nonzero.
  \item Let $S$ be a subring of $R$. Then consider the matrices
    $A,B\in M_n(S)$. Let $A_{ij}$ denote the $ij$th entry in the
    matrix.
    
    For $A+B$ we have $(A+B)_{ij}=A_{ij}+B_{ij}\in S$ since $S$
    is a subring. Since this hold for all entries we have
    that $A+B\in M_n(S)$.

    For $AB$ we have $(AB)_{ij}=\sum_{k=1}^n A_{ik}B_{kj}$.
    However each $A_{ik},B_{kj}\in S$ since $S$ is a subring
    which implies that $(AB)_{ij}\in S$ and therefore $AB\in M_n(S)$.
    Since $M_n(S)$ is closed under multiplication and addition
    it is a subring of $M_n(R)$.
  \item
    \begin{enumerate}
    \item Let $g\in G$. Then it has an inverse $g^{-1}\in G$
      for which both $g,g^{-1}\in R[G]$. Thus we have
      $gg^{-1}=e=1\in R[G]$ which shows that $g$ is a unit of $R[G]$.
    \item Let $R$ be a ring with a non-trivial unit $r\in R$. Then
      $rg\cdot r^{-1}g^{-1}=1e$ which implies that $rg\notin G$ but that
      $rg\in R[G]^X$. Therefore $G$  may not equal $R[G]^X$.
    \item Let $S$ be a subring of $R$ and let $f:=\sum_{g\in G}a_gg,g:=\sum_{g\in G}b_gg\in S[G]$.

      For $f+g$ we have $f+g=\sum_{g\in G}(a_g+b_g)g$ and since $a_g+b_g\in S$ due to
      $S$ being a subring $f+g\in S[G]$.

      For $fg$ we have $fg=\sum_{k\in G}\left(\sum_{gh=k}a_gb_h\right)k$. However
      since $S$ is a subring $\sum_{gh=K}(a_gb_h)\in S$ as it is a sum of terms in $S$.
      It then follows that $fg\in S[G]$.

      Therefore if $S$ is a subring of $R$ then $S[G]$ is a subring of $R[G]$.
    \end{enumerate}
  \item
    \begin{enumerate}
    \item Let $\sum_{g\in G}a_gg\in R[G]$. Then
      \[ \sum_{g\in G}a_gg\cdot\sum_{g\in G}g=\sum_{k\in G}\left(\sum_{gh=k}a_g\right)k \]
        Since rings are associative under addition we have
      \[ \sum_{k\in G}\left(\sum_{gh=k}a_g\right)k= \sum_{k\in G}\left(\sum_{gh=k}a_h\right)k
        = \sum_{g\in G}g\cdot\sum_{g\in G}a_gg\sum\]
      Therefore $\Lambda$ is in the center of $R[G]$.
    \item
    \item
    \end{enumerate}
  \end{enumerate}
\end{proof}

\sk

\begin{problem}[7.7]
  For any nonzero integers $a,b,$ prove that
  $(a,b)=(\gcd(a,b)),(a)\cap(b)=(\text{lcm}(a,b))$
  and that $(a)(b)=(ab)$.
\end{problem}

\begin{proof}
  
\end{proof}

\sk

\begin{problem}[7.8]
  Let $G$ be a finite group and $R$ a commutative ring. Show that the map
  $\epsilon:R[G]\rightarrow R$ given by
  \[ \epsilon\left(\sum_{g\in G}a_gg\right)=\sum_{g\in G}a_g \]
  is a surjective ring homomorphism and $\ker \epsilon$ is the ideal generated by
  the set $\{g-e|g\in G\}$.
\end{problem}

\begin{proof}
  
\end{proof}

\sk

\begin{problem}[7.10]
  \begin{enumerate}
  \item Prove that $x^2=0$ or $1$ for all $x\in\bb{Z}_4$
  \item Prove that the equation $x^2+y^2=3z^2$ has no nontrivial
    integer solution.
  \end{enumerate}
\end{problem}

\begin{proof}
  \begin{enumerate}
  \item For each case we have
    \begin{itemize}
    \item $0^2\equiv 0 \mod 4$
    \item $1^2\equiv 1 \mod 4$
    \item $2^2\equiv 0 \mod 4$
    \item $3^2\equiv 1 \mod 4$
    \end{itemize}
    Therefore the polynomial $x^2=0$ or $1$ for all $x\in\bb{Z}_4$.
  \item
  \end{enumerate}
\end{proof}

\sk

\begin{problem}[7.11]
  Let $D$ be a square-free integer and $I$ the ideal $(x^2-D)$ of $\bb{Q}[x]$.
  Prove that
  \[ \bb{Q}[x]/I\cong\bb{Q}(\sqrt{D}) \]
  as rings. Find all the ideals of $\bb{Q}[x]$ containing $I$.
\end{problem}

\begin{proof}
  
\end{proof}

%%%%%%%%%%%%%%%%%%%%%%%%%%%%%%%%%%%%%%%%%%%%%%%%%%%%%%%%%%%%%%%%%%%%%%%%%%%%%
\end{document}
