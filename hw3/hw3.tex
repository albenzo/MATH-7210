\documentclass[10pt]{article}
\usepackage[utf8]{inputenc}
\usepackage{amscd}
\usepackage{amsmath}
\usepackage{amssymb}
\usepackage{amsthm}
\usepackage{listings}
\usepackage{enumerate}

\textwidth=15cm \textheight=22cm \topmargin=0.5cm \oddsidemargin=0.5cm \evensidemargin=0.5cm

\newcommand{\sk}{\vskip 10mm}
\newcommand{\bb}[1]{\mathbb{#1}}
\newcommand{\ra}{\rightarrow}
\newcommand{\vc}[2]{\left(\begin{array}{cc} #1\\ #2\end{array}\right)}

\theoremstyle{plain}
\newtheorem{problem}{Problem}
\newtheorem{lemma}{Lemma}[problem]

\theoremstyle{remark}
\newtheorem{tpart}{}[problem]
\newtheorem*{ppart}{}

\begin{document}

\begin{problem}[1.6.2]
  
\end{problem}

\begin{proof}
  Let $\varphi:G\rightarrow H$ be an isomorphism and let $x\in G$ where $|x|=n$. Since
  $\varphi(x^n)=\varphi(x)^n$ it follows that $|x|\geq|\varphi(x)|$. However
  $\varphi$ is an isomorphism which implies that $\varphi^{-1}$ is also an
  isomorphism. Via the same reasoning this implies that
  $|\varphi(x)|\geq|\varphi^{-1}\circ\varphi(x)=x|$. Therefore $|x|=|\varphi(x)|$.

  If $\varphi$ is an isomorphism and $G_n$ is the set of elements of
  order $n$ in $G$ then $\varphi|_{G_n}$ is a bijection and since
  $\varphi$ preserves orders it follows that we have the same
  number of elements of order $n$ in $G$ and $H$ for any $n$.
\end{proof}

It does not hold if $\varphi$ is not an isomorphism. Consider
$\varphi:\bb{Z}/6\bb{Z}\rightarrow\{e\}$. Then the order of for any $\varphi(x)$ is
$1$.

\sk

\begin{problem}[1.6.3]
  
\end{problem}

\begin{proof}
  Let $\varphi:G\rightarrow H$ be an isomorphism and suppose that $H$ is Abelian.
  Then for $x,y\in G$ we have
  \[ xy = \varphi^{-1}\circ\varphi(xy)=\varphi^{-1}(\varphi(x)\varphi(y))=\varphi^{-1}(\varphi(y)\varphi(x))=yx \]
  which implies that $G$ is Abelian. If $G$ is Abelian
  swap $\varphi$ for $\varphi^{-1}$ and the reasoning will be identical to above.

  Therefore if $\varphi:G\rightarrow H$ is an isomorphism then $G$ is Abelian if and
  only if $H$ is Abelian.
\end{proof}

If $\varphi:G\rightarrow H$ is a homomorphism and $H$ is Abelian then we can show that
$G$ is Abelian if $\varphi$ is injective via the proof above as there will be
a well defined inverse on $\varphi(G)$.

Otherwise if $G$ is Abelian we can show that $H$ is Abelian if
$\varphi$ is surjective.

\begin{proof}
  Let $x,y\in H$. Since $\varphi$ is surjective there exist $a,b\in G$ such that
  $\varphi(a)=x$ and $\varphi(b)=y$. Then
  \[ xy = \varphi(a)\varphi(b)=\varphi(ab)=\varphi(ba)=\varphi(b)\varphi(a)=yx \]
  which implies that $H$ is Abelian.
\end{proof}

\sk

\begin{problem}[1.6.4]
  
\end{problem}

\begin{proof}
  Suppose that $\varphi:\bb{C}^*\rightarrow\bb{R}^*$. Then $\phi(z)=-1$ for some
  $z\in\bb{C}$. However there is a $w\in \bb{C}$ such that $w^2=z$
  which implies that there exists a $y\in\bb{R}^*$ such that
  $\varphi(w)=x$ and that $x^2=-1$ which is impossible.

  Therefore $\bb{C}^*$ and $\bb{R}^*$ are not isomorphic.
\end{proof}

\sk

\begin{problem}[1.6.7]
  
\end{problem}

\begin{proof}
  In $\mathcal{Q}_8$ the identity has order $1$, $|-1|=2$, and
  $|i|=|j|=|k|=|-i|=|-j|=|-k|=4$. However in $D_8$ the elements
  $r$ and $\rho^2$ are both of order two. Since $D_8$ has more
  elements of order two than $\mathcal{Q}_8$ they cannot be
  isomorphic.
\end{proof}

\sk

\begin{problem}[1.6.17]
  
\end{problem}

\begin{proof}
  Let $G$ be Abelian. Consider the inverse map $\varphi(g)=g^{-1}$.
  Then
  \[ \varphi(g)\varphi(h)=g^{-1}h^{-1}=h^{-1}g^{-1}=\varphi(gh)\]
  and since $e=e^{-1}$ it follows that $\varphi$ is a homomorphism.

  Otherwise suppose that $G$ is not Abelian. This implies that
  exist $g,h\in G$ such that $gh\neq hg$. Then
  \[ \varphi(g^{-1})\varphi(h^{-1})=gh\neq hg = \varphi(g^{-1}h^{-1})\]
  which implies that $\varphi$ cannot be a homomorphism.
\end{proof}

\sk

\begin{problem}[1.6.25]
  
\end{problem}

\begin{proof}
  \begin{itemize}
  \item[a)] Let $\vc{x}{y}$ be a point in $\bb{R}^2$. Then rewrite $\vc{x}{y}$ in
    polar coordinates as $\vc{r\cos\phi}{r\sin\phi}$. Then if we multiply by the
    rotation matrix we get
    \[
      \left(\begin{array}{cc}
        \cos\theta&-\sin\theta\\
        \sin\theta&\cos\theta
      \end{array}\right)
      \vc{r\cos\phi}{r\sin\phi}
      = \vc{r\cos\theta\cos\phi-r\sin\theta\sin\phi}{r\sin\theta\cos\phi+r\sin\phi\cos\theta}
      = \vc{r\cos(\theta+\phi)}{r\sin(\theta+\phi)}
    \]
    which corresponds to a rotation of $\vc{x}{y}$ clockwise by $\theta$.
  \item[b)] 
  \item[c)]
  \end{itemize}
\end{proof}

\sk

\begin{problem}[1.6.26]
  
\end{problem}

\begin{proof}
  Since $\mathcal{Q}_8$ is finite we can show that $\varphi:\mathcal{Q}_8\rightarrow \text{GL}_2(\bb{C})$
  by calculating the value of $\varphi$ for each $q\in\mathcal{Q}_8$.

  \[
    \begin{array}{ccc}
      
      \varphi(1)  &= \varphi(i^4) &=
                       \left(\begin{array}{cc}
                               1&0\\
                               0&1
                             \end{array}\right)\\\\
      \varphi(-1) &= \varphi(i^2) &=
                       \left(\begin{array}{cc}
                               -1&0\\
                               0&-1
                             \end{array}\right)\\\\
      \varphi(i)  &= \varphi(i)  &=
                       \left(\begin{array}{cc}
                               \sqrt{-1}&0\\
                               0&-\sqrt{-1}
                             \end{array}\right)\\\\
      \varphi(-i) &= \varphi(i^3) &=
                       \left(\begin{array}{cc}
                               -\sqrt{-1}&0\\
                               0&\sqrt{-1}
                             \end{array}\right)\\\\
      \varphi(j)  &= \varphi(j)  &=
                       \left(\begin{array}{cc}
                               0&-1\\
                               1&0
                             \end{array}\right)\\\\
      \varphi(-j) &= \varphi(j^3) &=
                       \left(\begin{array}{cc}
                               0&1\\
                               -1&0
                             \end{array}\right)\\\\
      \varphi(k)  &= \varphi(ij) &=
                       \left(\begin{array}{cc}
                               0&-\sqrt{-1}\\
                               -\sqrt{-1}&0
                             \end{array}\right)\\\\
      \varphi(-k) &= \varphi(ji) &=
                       \left(\begin{array}{cc}
                               0&\sqrt{-1}\\
                               \sqrt{-1}&0
                             \end{array}\right)
                                          
    \end{array}
  \]
  Since each element of $\mathcal{Q}_8$ maps to a distinct element of
  $\text{GL}_2(\bb{C})$ the homomorphism $\varphi$ is injective.
\end{proof}

\sk

\begin{problem}[2.1.3]
  
\end{problem}

\begin{proof}\ \\
  \begin{itemize}
  \item[a)]
    \[
      \begin{array}{c|cccc}
           &1  &r^2 &s &sr^2\\
        \hline
        1  &1  &r^2 &s  &sr^2\\
        r^2 &r^2 &1  &sr^2&s\\
        s  &s  &sr^2&1  &r^2\\
        sr^2&sr^2&s  &r^2 &1\\
      \end{array}
    \] 
  \item[b)]
    \[
      \begin{array}{c|cccc}
           &1  &r^2 &sr &sr^3\\
        \hline
        1  &1  &r^2 &sr &sr^3\\
        r^2 &r^2 &1  &sr^3&sr\\
        sr &sr &sr^3&1  &r^2\\
        sr^3&sr^3&sr &r^2 &1\\
      \end{array}
    \]
  \end{itemize}
\end{proof}

\sk

\begin{problem}[2.1.10(a)]
  
\end{problem}

\begin{proof}
  Let $H,K\leq G$. Then we will show that $H\cap K\leq G$.
  \begin{itemize}
  \item Since $H,K\leq G$ it follows that $e\in H$ and $e\in K$ which
    implies that $e\in H\cap K$
  \item Let $g\in H\cap K$. Then $g\in H$ and $g\in K$ which implies that
    $g^{-1}\in H$ and $g^{-1}\in K$. It then follows that $g^{-1}\in H\cap K$.
  \item Now let $g,h\in H\cap K$. Then $g,h\in H$ and $g,h\in K$ which implies
    that $gh\in H$ and $gh\in K$. It then follows that $gh\in H\cap K$.
  \end{itemize}

  Therefore if $H,K\leq G$ then $H\cap K\leq G$.
\end{proof}

\sk

\begin{problem}[2.3.1]
  
\end{problem}

The subgroups of the form $\left<x\right>$ of $\bb{Z}_{45}$ are
$\left<0\right>,\left<1\right>,\left<3\right>,\left<5\right>,\left<9\right>,$
and $\left<15\right>$. \textbf{Containment drawn below}.

\vskip 3in

\begin{problem}[2.3.3]
  
\end{problem}

The generators will be the elements of order $48$ which will consist of
the elements that are relatively prime to $48$. These are

\[ 1, 5, 7, 11, 13, 17, 19, 23, 25, 29, 31, 35, 37, 41, 43, 47 \]

\sk

\begin{problem}[2.3.12(a)]
  
\end{problem}

\begin{proof}
  A group is cyclic if it can be generated by a single element.
  There are a total of four elements in $Z_2\times Z_2$. These are
  $(0,0),(0,1),(1,0),$ and $(1,1)$. The order of $(0,0)$ is 1,
  and the order of the rest of the elements is $2$. However since
  the size of the group is $4$ it follows that none of the elements
  could generate the group as the size of the group generated is at
  most two.

  Therefore the group $Z_2\times Z_2$ is not cyclic.
\end{proof}

%%%%%%%%%%%%%%%%%%%%%%%%%%%%%%%%%%%%%%%%%%%%%%%%%%%%%%%%%%%%%%%%%%%%%%%%%%%%%
\end{document}
