\documentclass[10pt]{article}
\usepackage[utf8]{inputenc}
\usepackage{amscd}
\usepackage{amsmath}
\usepackage{amssymb}
\usepackage{amsthm}
\usepackage{listings}
\usepackage{enumerate}

\textwidth=15cm \textheight=22cm \topmargin=0.5cm \oddsidemargin=0.5cm \evensidemargin=0.5cm

\newcommand{\sk}{\vskip 10mm}
\newcommand{\bb}[1]{\mathbb{#1}}
\newcommand{\ra}{\rightarrow}

\theoremstyle{plain}
\newtheorem{problem}{Problem}
\newtheorem{lemma}{Lemma}[problem]

\theoremstyle{remark}
\newtheorem{tpart}{}[problem]
\newtheorem*{ppart}{}

\begin{document}

\begin{problem}[5.2.1]
  Find the isomorphism classes of Abelian groups of order 200.
\end{problem}

The isomorphism classes of Abelian groups of order 200 are:
\begin{enumerate}
\item $\bb{Z}_{200}$
\item $\bb{Z}_{40}\times\bb{Z}_{5}$
\item $\bb{Z}_{100}\times\bb{Z}_{2}$
\item $\bb{Z}_{20}\times\bb{Z}_{10}$
\item $\bb{Z}_{50}\times\bb{Z}_{2}\times\bb{Z}_{2}$
\item $\bb{Z}_{10}\times\bb{Z}_{10}\times\bb{Z}_{2}$
\end{enumerate}

\sk

\begin{problem}[5.2.2]
  Find the invariant factors and the elementary divisors of the Abelian
  group
  \[ G=\bb{Z}_2\times\bb{Z}_2\times\bb{Z}_2\times\bb{Z}_9\times\bb{Z}_5\times\bb{Z}_5\]
\end{problem}

If we combine relative prime numbers and rearrange we get
\[ G\cong \bb{Z}_{90}\times \bb{Z}_{10}\times \bb{Z}_2\]
giving us $90,10,2$ for the invariant factors.

We can also write $G$ as
$G\cong (\bb{Z}_2)^3\times(\bb{Z}_6)^5\times\bb{Z}_9$
which gives us the elementary divisors
$2^1,2^1,2^1,5^1,5^1,3^2$.

\sk

\begin{problem}[5.2.4]
  
\end{problem}

\begin{proof}
  Let $G$ be a finite group and $p$ a prime factor of $|G|$. Prove that the
  number of order $p$ elements in $G$ is congruent to $-1$ modulo $p$.
\end{proof}

\sk

\begin{problem}[5.3.2]
  Let $G$ be a finite group and $N_1,\ldots, N_n$ normal subgroups of $G$ such
  that $G=N_1\cdots N_n$ and $|G|=|N_1|\cdots|N_n|$. Prove that $G$ is the internal direct
  product of $G$.
\end{problem}

\begin{proof}
  The formula for the order of the product of groups is
  $|HK|=\frac{|H||K|}{|H\cap K|}$. As such the only way for
  $|G|=|N_1|\cdots|N_n|$ would be for $N_i\cap N_j=\{e\}$ for $i\neq j$.
  However this is equivalent to condition $2$ of Proposition
  $5.13$. Therefore $G$ is the internal direct product of
  $N_1,\ldots,N_n$.
\end{proof}

\sk

\begin{problem}[5.5.1]
  Let $G$ be a group, $H,K$ subgroups of $G$, and $H\trianglelefteq G$. Let
  $\varphi:K\rightarrow \text{Aut}(H)$ be the homomorphism associated with the conjugate
  action of $K$ on $H$. Then the following statements are equivalent:
  \begin{enumerate}
  \item $\phi:H\rtimes_\varphi K\rightarrow G$ defined by $\phi(h,k)=hk$ is an isomorphism.
  \item Every element $g\in G$ can be written as $g=hk$ with $h\in H$ and $k\in K$
    in a unique way.
  \item $G=HK$ and $H\cap K=\{e\}$.
  \end{enumerate}
\end{problem}

\begin{proof}
  \begin{enumerate}
  \item[$1\rightarrow 2$:] Since $\phi$ is an isomorphism, and thus surjective
    for any $g$ there is a pair $(h,k)$ such that $g=hk$.
    Writing $g=hk$ is unique due to $\phi$ being injective.
  \item[$2\rightarrow 3$:] Since we can write $g=hk$ for any $G$ we
    know that $G=HK$. To show that $H\cap K = \{e\}$ note
    that we can write $h=he$ and $k=ek$ for elements of
    $H$ and $K$. If $h=k_1k_2$ then it would have
    two representations $h=he=ek_1k_2$ which would be a contradiction.
  \item[$3\rightarrow 1$:] First we will show that $\phi(h,k)=hk$ is a homomorphism.
    Let $(h_1,k_1),(h_2,k_2)\in H\rtimes_\varphi K$. Then
    \[ \phi((h_1,k_1)(h_2,k_2))=\phi(h_1(k_1\cdot h_2),k_1k_2)=h_1\varphi(k_1)(h_2)k_1k_2
      =h_1k_1h_2k_2^{-1}k_1k_2=h_1k_1h_2k_2 =\phi(h_1,k_1)\phi(h_2,k_2) \]
    completing the proof that $\phi$ is a homomorphism.

    We know that $\phi$ is surjective as $G=HK$ and as such any element
    $g=hk$ for some $h\in H$ and $k\in K$.

    To show that $\phi$ is injective suppose that $\phi(h,k)=e$. Then
    $h^{-1}=k$ but since $H$ and $K$ have trivial intersection
    this means that $h=k=e$. Since the kernel of $\phi$ is trivial
    the map $\phi$ is injective.

    Therefore the map $\phi$ is an isomorphism.
  \end{enumerate}
\end{proof}

\sk

\begin{problem}[5.5.4]
  \begin{enumerate}
  \item[(a)] For any positive integer $n$, prove that $\text{Aut}(\bb{Z}_n)\cong \bb{Z}_n^*$.
  \item[(b)] For any primes $p<q$, if $p|q-1$, there exists a monomorphism
    $\varphi:\bb{Z}_p\rightarrow \text{Aut}(\bb{Z}_q)$ and $\bb{Z}_q\rtimes_\varphi \bb{Z}_p$ is a non-abelian
    group of order $pq$.
  \end{enumerate}
\end{problem}

\begin{proof}
  \begin{enumerate}
  \item[(a)] Define $\varphi:\bb{Z}_n^*\rightarrow \text{Aut}(\bb{Z}_n)$ as $m\mapsto \phi_m$ where $\phi_m(x)=mx$
    with multiplication done modulo $n$. To show that this is a homomorphism
    consider $m_1,m_2\in\bb{Z}_n^*$. Then
    \[ \varphi(m_1m_2)= \phi_{m_1m_2}\]
    For any $x\in \bb{Z}_n$ we have
    \[ \phi_{m_1m_2}(x)=(m_1m_2)x=m_1(m_2x)=m_1\phi_{m_2}(x)=\phi_{m_1}\circ\phi_{m_2}(x) \]
    Which implies that
    \[ \varphi(m_1m_2)=\phi_{m_1m_2}=\phi_{m_1}\circ\phi_{m_2}=\varphi(m_1)\circ\varphi(m_2) \]
    Therefore the map $\varphi$ is a homomorphism.

    To show it is injective suppose that for $m\in\bb{Z}_n^*$ we had
    $\phi_m(x)=x$ for all $x\in\bb{Z}_n$. Then $mx=x$ for
    all $x$ which would imply that $m=1$. Therefore the kernel
    of $\varphi$ is trivial and as such $\varphi$ is injective.

    Finally to show that it is surjective consider $f\in \text{Aut}(\bb{Z}_n)$. Then
    \textbf{Finish this.}
  \item[(b)] Since $p|q-1$ we know that $pk+1=q$ for some
    $k\in\bb{Z}^+$. Define a map $\varphi:\bb{Z}_p\rightarrow \text{Aut}(\bb{Z}_q)$ via
    $i\mapsto \phi_{2^{ik}}$ where $\phi_{2^{ik}}(x)=2^{ik}x$. To see that this is a
    homomorphism let $x\in \bb{Z}_q$ and $i,j\in \bb{Z}_p$
    \[ \varphi(i+j)(x)=\phi_{i+j}(x)=2^{k(i+j)}x=2^{ki}2^{kj}x=\phi_{i}\circ\phi_j(x)=\varphi(i)\circ\varphi(j) \]
    Therefore $\varphi$ is a group homomorphism.

    To see that it is injective suppose that $\phi_i(x)=x$. Then
    $2^ix=x$ which implies that $2^i=1$ and that $i=0$. Since
    the kernel is trivial $\varphi$ is injective.

    By definition the group $|\bb{Z}_q\rtimes_\varphi\bb{Z}_p$ has order $pq$.
    To show that it is not Abelian consider $(g,n)$ and $(h,m)$ where
    $m\neq n$. Then
    \[ (g,n)(h,m)=(gh2^{nk},n+m)\]
    and
    \[ (h,m)(g,n)=(gh2^{mk},m+n)\]
    which are only equal if $m=n$.
  \end{enumerate}
\end{proof}

\sk

\begin{problem}[5.5.11(book)]
  Classify groups of order 28 (there are four isomorphism types).
\end{problem}

The different groups of order 28 are:
\begin{enumerate}
\item $\bb{Z}_{28}$ cyclic
\item $\bb{Z}_{14}\times \bb{Z}_2$ product and abelian
\item $D_{28}$ Not abelian
\item $\bb{Z}_7\rtimes\bb{Z}_4$ \textbf{Put in a reason}
\end{enumerate}

\sk

%%%%%%%%%%%%%%%%%%%%%%%%%%%%%%%%%%%%%%%%%%%%%%%%%%%%%%%%%%%%%%%%%%%%%%%%%%%%%
\end{document}
