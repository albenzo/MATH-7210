\documentclass[10pt]{article}
\usepackage[utf8]{inputenc}
\usepackage{amscd}
\usepackage{amsmath}
\usepackage{amssymb}
\usepackage{amsthm}
\usepackage{listings}
\usepackage{enumerate}

\textwidth=15cm \textheight=22cm \topmargin=0.5cm \oddsidemargin=0.5cm \evensidemargin=0.5cm

\newcommand{\sk}{\vskip 10mm}
\newcommand{\bb}[1]{\mathbb{#1}}
\newcommand{\ra}{\rightarrow}

\theoremstyle{plain}
\newtheorem{problem}{Problem}
\newtheorem{lemma}{Lemma}[problem]

\theoremstyle{remark}
\newtheorem{tpart}{}[problem]
\newtheorem*{ppart}{}

\begin{document}

\begin{problem}[12.1]
  \begin{enumerate}
  \item Let $R$ be a ring and $M$ and $R$-module. Prove that $r0=0$
    for $r\in R$. If $R$ has the identity 1, then $(-1)x=-x$ for $x\in M$.
  \item Let $R$ be a ring and $M,N,L$ be $R$-modules. Prove:
    \begin{enumerate}
    \item $\hom_R(M,N)$ is an abelian group under addition
      \[ (\phi+\psi)(m)=\phi(m)+\psi(m) \]
      If $R$ is commutative, $\hom(M,N)$ is an $R$-module with the $R$-action
      given by
      \[ (r\phi)(m)=r\phi(m)\]
    \item If $\phi\in\hom_R(M,N)$ and $\psi\in\hom_R(N,L)$, then $\psi\circ\phi\in\hom_R(M,L)$.
    \item $\hom_R(M,M)$ is a ring with identity with composition as
      multiplication.
    \end{enumerate}
  \item Prove that $\hom_{\bb{Z}}(\bb{Z}_n,\bb{Z}_m)\cong\bb{Z}_d$ where
    $d=\gcd(m,n)$.
  \end{enumerate}
\end{problem}

\begin{proof}
  
\end{proof}

\sk

\begin{problem}[12.2]
  Let $A,B$ be submodules of an $R$-module $M$. Prove that $A+B$
  and $A\cap B$ are submodules of $M$. Moreover, the equality
  \[ A\cap(B+C)=B+(A\cap C)\]
  holds for all $R$-submodules $C$ if $B\subseteq A$.
\end{problem}

\begin{proof}
  
\end{proof}

\sk

\begin{problem}[12.4]
  Let $M$ be an $R$-module.
  \begin{enumerate}
  \item For any submodules $N_1,\ldots,N_n$ of $M$, their sum
    $N_1+\cdots+N_n$ is the smallest submodule of $M$ which contains
    $N_1\cup\cdots\cup N_n$.
  \item For any subset $A$ of $M$, $RA$ is the smallest submodule
    of $M$ which contains $A$.
  \end{enumerate}
\end{problem}

\begin{proof}
  
\end{proof}

\sk

\begin{problem}[12.5]
  Show that $\bb{Z}_{p^e}$, regarded as a $\bb{Z}$-module is not a
  direct sum of any two non-zero submodules, where $p$ is a prime and
  $e>0$. Does it hold for $\bb{Z}$? Does it hold for $\bb{Z}_{12}$?
\end{problem}

\begin{proof}
  
\end{proof}

\sk

\begin{problem}[12.7]
  Let $R$ be a PID and $p$ a prime in $R$.
  \begin{enumerate}
  \item If $M$ is a finitely generated $p$-primary $R$-module, then
    $M/pM$ is an $R/(p)$-module with the $R$-action given by $(12.6)$.
    Moreover, show that the mapping $\phi$ defined in $(12.7)$ is a
    $R/(p)$-module map.
  \item Let $\phi:M_1\rightarrow M_2$ be an isomorphism finitely generated $p$-primary
    $R$-modules. Prove that $\phi|_{pM_1}:pM_1\rightarrow pM_2$ is an isomorphism
    of $R$-module. Show that the map $\bar{\phi}:M_1/pM_1\rightarrow M_2/pM_2$ defined by
    \[ \bar{\phi}(m+pM_1)=\phi(m)+pM_2\]
    is an isomorphism of $R/(p)$-vector spaces.
  \end{enumerate}
\end{problem}

\begin{proof}
  
\end{proof}

\sk

\begin{problem}[12.8]
  \begin{enumerate}
  \item Find the Smith normal form of the integer matrix
    \[
      \left[
        \begin{array}{ccc}
          2&1&3\\
          1&-1&3\\
        \end{array}
      \right]
    \]
  \item Determine the invariant factor decomposition of $\bb{Z}^3/K$ where
    $K$ is generated by $f_1(2,1,-3)$ and $f_2=(1,-1,2)$.
  \end{enumerate}
\end{problem}

\begin{proof}
  
\end{proof}

\sk

\begin{problem}[12.9]
  \begin{enumerate}
  \item Find a basic for the submodule $K$ of $\bb{Q}[x]^3$ generated by
    \[ f_1=(2x-1,x,x^2+3),\quad f_2=(x,x,x^2),\quad f_3=(x+1,2x,2x^2-3)\]
  \item Find the invariant factors and elementary divisors of the
    $\bb{Q}[x]$-module $\bb{Q}[x]^3/K$.
  \end{enumerate}
\end{problem}

\begin{proof}
  
\end{proof}

\sk

\begin{problem}[12.11]
  Let $F$ be a field and $V$ an $n$-dimensional vector space
  over $F$ with an ordered basis $\mathcal{B}$.
  \begin{enumerate}
  \item Let $T$ be a linear operator on $V$. For any ordered basis
    $\mathcal{B}'$ of $V$, the matrices $[T]_{\mathcal{B}}$ and
    $[T]_{\mathcal{B}'}$ are similar over $F$. Conversely, if
    $A\in M_n(F)$ is similar to $[T]_{\mathcal{B}}$ over $F$, there
    exists a basis $\mathcal{B}'$ such that $[T]_{\mathcal{B}'}=A$.
  \end{enumerate}
\end{problem}

\begin{proof}
  
\end{proof}

\sk

\begin{problem}[12.13]
  Find the rational canonical form of the matrix
  \[
    A = 
    \left[
      \begin{array}{ccc}
        -1&-2&6\\
        -1&0&3\\
        -1&-1&4\\
      \end{array}
    \right] \in M_3(\bb{Q})
  \]
  Consider $A\in M_3(\bb{C})$ and find the Jordan canonical form of $A$.
\end{problem}

\begin{proof}
  
\end{proof}

\sk

%%%%%%%%%%%%%%%%%%%%%%%%%%%%%%%%%%%%%%%%%%%%%%%%%%%%%%%%%%%%%%%%%%%%%%%%%%%%%
\end{document}
