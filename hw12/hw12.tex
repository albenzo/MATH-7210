\documentclass[10pt]{article}
\usepackage[utf8]{inputenc}
\usepackage{amscd}
\usepackage{amsmath}
\usepackage{amssymb}
\usepackage{amsthm}
\usepackage{listings}
\usepackage{enumerate}

\textwidth=15cm \textheight=22cm \topmargin=0.5cm \oddsidemargin=0.5cm \evensidemargin=0.5cm

\newcommand{\sk}{\vskip 10mm}
\newcommand{\bb}[1]{\mathbb{#1}}
\newcommand{\ra}{\rightarrow}

\theoremstyle{plain}
\newtheorem{problem}{Problem}
\newtheorem{lemma}{Lemma}[problem]

\theoremstyle{remark}
\newtheorem{tpart}{}[problem]
\newtheorem*{ppart}{}

\begin{document}

\begin{problem}[12.1]
  \begin{enumerate}
  \item Let $R$ be a ring and $M$ and $R$-module. Prove that $r0=0$
    for $r\in R$. If $R$ has the identity 1, then $(-1)x=-x$ for $x\in M$.
  \item Let $R$ be a ring and $M,N,L$ be $R$-modules. Prove:
    \begin{enumerate}
    \item $\hom_R(M,N)$ is an abelian group under addition
      \[ (\phi+\psi)(m)=\phi(m)+\psi(m) \]
      If $R$ is commutative, $\hom(M,N)$ is an $R$-module with the $R$-action
      given by
      \[ (r\phi)(m)=r\phi(m)\]
    \item If $\phi\in\hom_R(M,N)$ and $\psi\in\hom_R(N,L)$, then $\psi\circ\phi\in\hom_R(M,L)$.
    \item $\hom_R(M,M)$ is a ring with identity with composition as
      multiplication.
    \end{enumerate}
  \item Prove that $\hom_{\bb{Z}}(\bb{Z}_n,\bb{Z}_m)\cong\bb{Z}_d$ where
    $d=\gcd(m,n)$.
  \end{enumerate}
\end{problem}

\begin{proof}
  \begin{enumerate}
  \item Note that $r0=r(0+0)=r0+r0$. Subtract an $r0$ from each side
    and you get that $0=r0$ for all $r\in R$.

    For the second part. We have that
    \[ 0= 0x = (1-1)x = 1x + (-1)x = x + (-1)x \]
    which implies that $(-1)x=-x$.
  \item
    \begin{enumerate}
    \item The group operation will be associative because it is adding group elements
      and the addition of group elements is associative. The identity will be
      the function $z(m)=0$ as
      \[ (\phi+z)(m)=\phi(m)+0=\phi(m) \]
      The inverse for $\phi\in\hom(M,N)$ will be $\psi(m)=-\phi(m)$ as
      \[ (\phi+\psi)(m)=\phi(m)-\phi(m)=0 \]
      Since we have all of the group axioms fulfilled $\hom_R(M,N)$ is a group.

      To show that $\hom_R(M,N)$ is an $R$-module when $R$ is commutative we will
      verify the four axioms from the notes. We will have $r,s\in R$ with 
      $\phi,\psi\in\hom_R(M,N)$, and $m\in M$ further down.
      \begin{enumerate}
      \item Start with $(r+s)\phi$. Then for an arbitrary element $m$ we have
        \[ (r+s)\phi(m)=r\phi(m)+s\phi(m)\]
        from the fact that $N$ is an $R$-module.
      \item For the next we have
        \[ (rs)\phi(m) = r(s\phi(m))\]
        which shows that
        \[ (rs)\phi = r(s\phi)\]
        following from $N$ being an $R$ module.
      \item Next we have
        \[ r(\phi+\psi)(m)=r(\phi(m)+\psi(m))=r\phi(m)+r\psi(m)\]
        following from $N$ being an $R$ module which shows that
        \[ r(\phi+\psi)=r\phi+r\psi \]
      \item Finally
        \[ 1\phi(m)=\phi(m)\]
        as $N$ is an $R$-module which implies that
        \[ 1\phi=\phi \]
      \end{enumerate}
      This completes the proof.
    \item We know that $\psi\circ\phi\in\hom(M,L)$ because they are group homomorphisms. As
      such all we have to show is that the composition preserves the $R$-module
      structure. Let $r\in R$. Then from the fact that $\phi,\psi$ are $R$-module
      homomorphisms we have that
      \[ \psi\circ\phi(rm)=\psi(r\phi(m))=r\psi\circ\phi(m) \]
      which verifies that $\psi\circ\phi$ is a homomorphism of $R$-modules and as such
      $\psi\circ\phi\in\hom_R(M,L)$.
    \item We know from above that $\hom_R(M,M)$ is a group under addition of
      maps, that the composition is well defined, and that composition is
      associative with identity ($id_M$) in general. Thus the only remaining
      portion to show is that composition distributes over addition of maps.
      Let $\phi,\psi,\varphi\in\hom_R(M,N)$. Then
      \[ \varphi\circ(\phi+\psi)(m)=\varphi(\phi(m)+\psi(m))=\varphi\circ\phi(m)+\varphi\circ\psi(m)\]
      Thus composition distributes over addition of maps and therefore
      $\hom_R(M,M)$ forms a ring.
    \end{enumerate}
  \item From a previous assignment we know that all homomorphisms in
    $\hom_{\bb{Z}}(\bb{Z}_n,\bb{Z}_m)$ are of the form $\phi_k(x)=kx$. The
    maps $\phi_k$ will be in $\hom_{\bb{Z}}(\bb{Z}_n,\bb{Z}_m)$ only when
    $nk\equiv 0 \mod m$ ($nk=mq$ for some $q\in\bb{Z}$). The number of $k$s
    that fulfill this requirement is $\gcd(m,n)=d$. Then consider
    the map $\phi_a$ where $a=\frac{m}{d}$. If we add $\phi_a$  to itself we
    will get $\frac{m}{a}=d$ different homomorphisms before reaching the
    identity. Since the group $\hom_{\bb{Z}}(\bb{Z}_n,\bb{Z}_m)$ has $d$
    elements and is cyclic it must be isomorphic to $\bb{Z}_d$.
  \end{enumerate}
\end{proof}

\sk

\begin{problem}[12.2]
  Let $A,B$ be submodules of an $R$-module $M$. Prove that $A+B$
  and $A\cap B$ are submodules of $M$. Moreover, the equality
  \[ A\cap(B+C)=B+(A\cap C)\]
  holds for all $R$-submodules $C$ if $B\subseteq A$.
\end{problem}

\begin{proof}
  As $A,B$ are submodules of $M$ they are also subgroups and as such $A+B,A\cap B$ are
  closed under the group operations. The only thing left to verify is that
  they are closed under the action of $R$.

  Let $r\in R$ and $a+b\in A+B$ with $a\in A$ and $b\in B$. Then $r(a+b)=ra+rb$ and
  since $ra\in A$ and $rb\in B$ we have that $r(a+b)=ra+rb\in A+B$. Therefore $A+B$ is
  a submodule.

  Next let $m\in A\cap B$. Then $rm\in A$ and $rm\in B$ which implies that $rm\in A\cap B$.
  Therefore $A\cap B$ is a submodule.

  Now suppose that $B\subseteq A$ where $A,B,C$ are submodules of $M$. Let
  $m\in A\cap(B+C)$. Then $m=b+c$ where $b\in B$ and $c\in C$ and $b+c\in A$. However since
  $b\in A$, as $B\subseteq A$, we have that $c=(b+c)-b\in A$. Therefore $m=b+c\in B+(A\cap C)$
  and thus $A\cap(B+C)\subseteq B+(A\cap C)$.

  Let $m\in B+(A\cap C)$. Then $m=b+c$ where $b\in B$ and $c\in A\cap C$. However since
  $B\subseteq A$ we have that $b\in A$ which implies that $b+c\in A$ and that $b+c\in B+C$.
  Thus $m\in A\cap(B+C)$ and therefore $B+(A\cap C)\subseteq A\cap(B+C)$.

  Therefore if $B\subseteq A$ then $A\cap(B+C)=B+(A\cap C)$.
\end{proof}

\sk

\begin{problem}[12.4]
  Let $M$ be an $R$-module.
  \begin{enumerate}
  \item For any submodules $N_1,\ldots,N_n$ of $M$, their sum
    $N_1+\cdots+N_n$ is the smallest submodule of $M$ which contains
    $N_1\cup\cdots\cup N_n$.
  \item For any subset $A$ of $M$, $RA$ is the smallest submodule
    of $M$ which contains $A$.
  \end{enumerate}
\end{problem}

\begin{proof}
  \begin{enumerate}
  \item Since we are summing a finite number of submodules the fact that
    $N_1+\cdots+N_n$ is a submodule follows from the previous problem. Let $N$
    be a submodule of $M$ such that $\bigcup_iN_i\subseteq N$ and let $\sum_ik_i\in N_1+\cdots+N_n$
    with $k_i\in N_i$. Then $k_i\in N$ for all $i$. However since submodules are
    closed under addition we have that $\sum_ik_i\in N$. As this holds for an
    arbitrary element of $N_1+\cdots+N_n$ it must be that $N_1+\cdots +N_n\subseteq N$.

    Therefore if $N$ is a submodule such that $\bigcup_i N_i\subseteq N$ then
    $N_1+\cdots+N_n\subseteq N$.
  \item We know that $RA$ will be a submodule from the notes. Let $N$ be a
    submodule of $M$ such that $A\subset N$ and let $ra\in RA$. Since $a\in N$ and
    $N$ is a submodule then $ra\in N$ which implies that $RA\subset N$.

    Therefore if $A\subseteq M$ then any submodule $N$ that contains $A$ will
    contain $RA$.
  \end{enumerate}
\end{proof}

\sk

\begin{problem}[12.5]
  Show that $\bb{Z}_{p^e}$, regarded as a $\bb{Z}$-module is not a
  direct sum of any two non-zero submodules, where $p$ is a prime and
  $e>0$. Does it hold for $\bb{Z}$? Does it hold for $\bb{Z}_{12}$?
\end{problem}

\begin{proof}
  Suppose that $\bb{Z}_{p^e}\cong\bb{Z}_{p^a}\oplus\bb{Z}_{p^b}$ where $b,a>0$. The
  decomposition would have to be of this form because of the orders.
  However this is the same as implies that $\bb{Z}_{p^e}\cong\bb{Z}_{p^a}\times\bb{Z}_{p^b}$.
  However the group on the right is not cyclic which is a contradiction.
\end{proof}

This does not hold for $\bb{Z}_{12}$ as $\bb{Z}_{12}\cong\langle 3\rangle+\langle 4\rangle$.

This does hold for $\bb{Z}$ as any submodules will be isomorphic to $\bb{Z}$
and as such the direct sum would be $\bb{Z}\oplus\bb{Z}$ which is not isomorphic to
$\bb{Z}$.

\sk

\begin{problem}[12.7]
  Let $R$ be a PID and $p$ a prime in $R$.
  \begin{enumerate}
  \item If $M$ is a finitely generated $p$-primary $R$-module, then
    $M/pM$ is an $R/(p)$-module with the $R$-action given by
    \[ (r+(p))(x+pM):= rx+pM \]
    Moreover, show that the mapping $\phi$ defined in
    \[ \phi(r_1x_1+\cdots + r_mx_m +pM) = (\bar{r}_1,\ldots,\bar{r}_m) \]
    is a $R/(p)$-module map.
  \item Let $\phi:M_1\rightarrow M_2$ be an isomorphism finitely generated $p$-primary
    $R$-modules. Prove that $\phi|_{pM_1}:pM_1\rightarrow pM_2$ is an isomorphism
    of $R$-module. Show that the map $\bar{\phi}:M_1/pM_1\rightarrow M_2/pM_2$ defined by
    \[ \bar{\phi}(m+pM_1)=\phi(m)+pM_2\]
    is an isomorphism of $R/(p)$-vector spaces.
  \end{enumerate}
\end{problem}

\begin{proof}
  \begin{enumerate}
  \item $M/pM$ is already a quotient group and as such we do not need to verify
    the group axioms. Thus the items that we need to check are
    \begin{enumerate}
    \item
    \item
    \item
    \item
    \end{enumerate}

    Now we will show that $\phi$ is an $R$-module map.
  \item Since the map $\phi$ is injective we know that the restriction will be as
    well. Thus the only portion left to show is that $\phi(pM_1)=pM_2$. To show
    this

    Since the map $\phi$ is an isomorphism and we are quotienting out by isomorphic
    submodules it then follows that $\bar{\phi}$ will be an isomorphism between
    $M_1/pM_1$ and $M_2/pM_2$ as it will respect the action of $R/(p)$.
  \end{enumerate}
\end{proof}

\sk

\begin{problem}[12.8]
  \begin{enumerate}
  \item Find the Smith normal form of the integer matrix
    \[
      \left[
        \begin{array}{ccc}
          2&1&3\\
          1&-1&2\\
        \end{array}
      \right]
    \]
  \item Determine the invariant factor decomposition of $\bb{Z}^3/K$ where
    $K$ is generated by $f_1(2,1,-3)$ and $f_2=(1,-1,2)$.
  \end{enumerate}
\end{problem}

\begin{proof}
  \begin{enumerate}
  \item The Smith normal form of the above matrix is
    \[
      \left(
        \begin{array}{rr}
          0 & 1 \\
          1 & 0
        \end{array}
      \right)
      \left(
        \begin{array}{rrr}
          1 & 0 & 0 \\
          0 & 1 & 0
        \end{array}
      \right)
      \left(
        \begin{array}{rrr}
          -3 & 2 & -5 \\
          0 & 0 & 1 \\
          2 & -1 & 3
        \end{array}
      \right)
    \]
  \item The Smith normal form of the matrix with rows of $f_1,f_2$
    is
    \[
      \left(
        \begin{array}{ccc}
          1&0&0\\
          0&1&0
        \end{array}
      \right)
    \]
    Which means that the invariant factor decomposition will be
    $\bb{Z}$.
  \end{enumerate}
\end{proof}

\sk

\begin{problem}[12.9]
  \begin{enumerate}
  \item Find a basis for the submodule $K$ of $\bb{Q}[x]^3$ generated by
    \[ f_1=(2x-1,x,x^2+3),\quad f_2=(x,x,x^2),\quad f_3=(x+1,2x,2x^2-3)\]
  \item Find the invariant factors and elementary divisors of the
    $\bb{Q}[x]$-module $\bb{Q}[x]^3/K$.
  \end{enumerate}
\end{problem}

\begin{proof}
  \begin{enumerate}
  \item The Smith normal form of the matrix with $f_i$s as rows will be
    \[
      \left(
      \begin{array}{ccc}
        1 & 0 & 0 \\
        0 & x & 0 \\
        0 & 0 & 0\\
      \end{array}
      \right)
    \]
    Thus the basis will be $(1,x)$
  \item The invariant factor will be $x$. The elementary divisor is $x$.
\end{enumerate}
\end{proof}

\sk

\begin{problem}[12.11]
  Let $F$ be a field and $V$ an $n$-dimensional vector space
  over $F$ with an ordered basis $\mathcal{B}$.
  \begin{enumerate}
  \item Let $T$ be a linear operator on $V$. For any ordered basis
    $\mathcal{B}'$ of $V$, the matrices $[T]_{\mathcal{B}}$ and
    $[T]_{\mathcal{B}'}$ are similar over $F$. Conversely, if
    $A\in M_n(F)$ is similar to $[T]_{\mathcal{B}}$ over $F$, there
    exists a basis $\mathcal{B}'$ such that $[T]_{\mathcal{B}'}=A$.
  \item Two $F$-linear operators $S,T$ on $V$ are similar if, and only if,
    the matrices $[T]_{\mathcal{B}}$ and $[S]_{\mathcal{B}}$ are similar.
  \end{enumerate}
\end{problem}

\begin{proof}
  \begin{enumerate}
  \item Let $\mathcal{B}$ and $\mathcal{B}'$ be ordered bases. Then there is
    an invertible matrix $P$ that changes from one basis to the other
    ($\mathcal{B}=P\mathcal{B}'$).
    Then
    \[[T]_{\mathcal{B}}=[T]_{P\mathcal{B}'}=P[T]_{\mathcal{B}'}P^{-1} \]
    which shows that they are similar.

    Now let $A\in M_n(F)$ and $A\sim [T]_{\mathcal{B}}$ over $F$. Then
    $A=P[T]_{\mathcal{B}}P^{-1}$ for an invertible matrix $P$. Let
    $\mathcal{B}'=P\mathcal{B}$. Then $[T]_{\mathcal{B}'}=A$ because of the prior
    part of this part.
  \item Let $S,T$ be similar linear transformations. Then there is an isomorphism
    of vector spaces $\varphi:V\rightarrow V$ such that $S=\varphi\circ T\circ\varphi^{-1}$. However given an ordered
    basis $\mathcal{B}$ we can express the prior equation as
    \[ [S]_{\mathcal{B}}=[\varphi]_{\mathcal{B}}[T]_{\mathcal{B}}[\varphi^{-1}]_{\mathcal{B}}\]
    which shows that $[S]_{\mathcal{B}}\sim[T]_{\mathcal{B}}$.

    Now suppose that $[S]_{\mathcal{B}}\sim[T]_{\mathcal{B}}$. Then there is an invertible
    matrix $P$ such that $[S]_{\mathcal{B}}=P[T]_{\mathcal{B}}P^{-1}$ which shows that
    the matrix $P$ corresponds to an isomorphism. It then follows that
    $S$ and $T$ are similar.

    Therefore two $F$-linear operators are similar if and only if their matrices
    with respect to an ordered basis are similar.
  \end{enumerate}
\end{proof}

\sk

\begin{problem}[12.13]
  Find the rational canonical form of the matrix
  \[
    A = 
    \left[
      \begin{array}{ccc}
        -1&-2&6\\
        -1&0&3\\
        -1&-1&4\\
      \end{array}
    \right] \in M_3(\bb{Q})
  \]
  Consider $A\in M_3(\bb{C})$ and find the Jordan canonical form of $A$.
\end{problem}

To find the canonical form we start with
\[ xI-A = 
  \left(
    \begin{array}{ccc}
      1+x&2&-6\\
      1&x&-3\\
      1&1&x-4
    \end{array}
  \right)\]
which we then reduce to
\[
  \left(
    \begin{array}{ccc}
      1&0&0\\
      0&x-1&0\\
      0&0&x^2+2x-1\\
    \end{array}
  \right)
\]
Which gives us the rational canonical form
\[
  \left(
    \begin{array}{ccc}
      1&0&0\\
      0&1&0\\
      0&1&1\\
    \end{array}
  \right)
\]
in $\bb{Q}$. This will be the same as it is in $\bb{C}$
since all of the polynomials had roots in $\bb{Q}$.

\sk

%%%%%%%%%%%%%%%%%%%%%%%%%%%%%%%%%%%%%%%%%%%%%%%%%%%%%%%%%%%%%%%%%%%%%%%%%%%%% 
\end{document}
