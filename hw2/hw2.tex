\documentclass[10pt]{article}
\usepackage[utf8]{inputenc}
\usepackage{amscd}
\usepackage{amsmath}
\usepackage{amssymb}
\usepackage{amsthm}
\usepackage{listings}
\usepackage{enumerate}

\textwidth=15cm \textheight=22cm \topmargin=0.5cm \oddsidemargin=0.5cm \evensidemargin=0.5cm

\newcommand{\sk}{\vskip 10mm}
\newcommand{\bb}[1]{\mathbb{#1}}
\newcommand{\ra}{\rightarrow}

\theoremstyle{plain}
\newtheorem{problem}{Problem}
\newtheorem{lemma}{Lemma}[problem]

\theoremstyle{remark}
\newtheorem{tpart}{}[problem]
\newtheorem*{ppart}{}

\begin{document}

\begin{problem}[1.1.21] 
  
\end{problem}

\begin{proof}
  Let $x\in G$ have order odd $2n+1$. Then $x^{2n+1}=e$ and if we multiply by $x$
  on both sides we get $x^{2n+2}=x^{2(n+1)}=x$.

  Therefore if $x$ has odd order then it $x^{2k}=x$ for some $k\in\bb{Z}_{>0}$.
\end{proof}

\sk

\begin{problem}[1.1.25] 
  
\end{problem}

\begin{proof}
  Suppose that $G$ is a group such that for all $x\in G$ that $x^2=1$.
  This implies that for any element $x^{-1}=x$. Then we have
  $xy=(xy)^{-1}=y^{-1}x^{-1}=yx$
  which shows that $G$ is commutative.
\end{proof}

\sk

\begin{problem}[1.1.35] 
  
\end{problem}

\begin{proof}
  Let $x\in G$ have order $n$. Then consider $x^m$. Using the division algorithm
  with $m$ and $n$ we can rewrite $m$ as $qn+d$ where $0\leq d<n$. This implies
  that
  \[ x^m=x^{qn+d}=x^{qn}x^d=(x^n)^qx^d=e^qx^d=x^d \]
  As such $x\in\{e,x^1,\ldots,x^{n-1}\}$.
\end{proof}

\sk

\begin{problem}[1.3.6] 
  
\end{problem}

\begin{itemize}
\item $(1\ 2\ 3\ 4)=(1\ 4)(1\ 3)(1\ 2)$
\item $(1\ 3\ 4\ 2)=(1\ 2)(1\ 4)(1\ 3)$
\item $(1\ 4\ 3\ 2)=(1\ 2)(1\ 3)(1\ 4)$
\item $(4\ 3\ 2\ 1)=(4\ 1)(4\ 2)(4\ 3)$
\item $(2\ 4\ 3\ 1)=(2\ 1)(2\ 3)(2\ 4)$
\item $(3\ 2\ 4\ 1)=(3\ 1)(3\ 4)(3\ 2)$
\end{itemize}

\sk

\begin{problem}[1.3.9] 
  
\end{problem}

\begin{itemize}
\item[a)] $1+12k,5+12k,7+12k,11+12k$ for $k\in\bb{Z}$.
\item[b)] $1+8k,3+8k,5+8k,7+8k$ for $k\in\bb{Z}$.
\item[c)] $1+14k,3+14k,5+14k,9+14k,11+14k,13+14k$ for $k\in\bb{Z}$.
\end{itemize}

\sk

\begin{problem}[1.3.13] 
  
\end{problem}

\begin{proof}
  Suppose that $g\in S_n$ is of the form $g=\prod_{i=1}^m(a_i\ b_i)$
  where each transposition commutes. Then
  \[ g^2 = \left(\prod_{i=1}^m(a_i\ b_i)\right)^2=
    \prod_{i=1}^m((a_i\ b_i)^2=e)=e \]
  which implies that $g$ is of order 2.

  Next suppose that $g\in S_n$ is of order 2. We can write $g$
  as a product of disjoint cycles $\prod_{i=1}^m\sigma_i$. Since
  the cycles are disjoint we can write $g^2$ as
  \[ g^2=\left(\prod_{i=1}^m\sigma_i\right)^2=\prod_{i=1}^m\sigma_i^2=e \]

  Then for a given $\sigma_i$ rewrite the above to
  \[ \prod_{i\in\{1\ldots m\}-j}\sigma_i^2 = \sigma_j^{-2} \]
  However since each $\sigma_i$ is disjoint this implies that
  $\sigma_j^{-2}=e=\sigma_j^2$. Since this is for an arbitrary
  $\sigma_i$ we have that $\sigma_i^2$ for all $i$.

  Therefore a permutation is of order two if and only if
  it is the product of disjoint $2-$cycles.
\end{proof}

\sk

\begin{problem}[1.5.2] 
  
\end{problem}

\begin{itemize}
\item
\item
\item
\end{itemize}
\sk

%%%%%%%%%%%%%%%%%%%%%%%%%%%%%%%%%%%%%%%%%%%%%%%%%%%%%%%%%%%%%%%%%%%%%%%%%%%%%
\end{document}
