\documentclass[10pt]{article}
\usepackage[utf8]{inputenc}
\usepackage{amscd}
\usepackage{amsmath}
\usepackage{amssymb}
\usepackage{amsthm}
\usepackage{listings}
\usepackage{enumerate}

\textwidth=15cm \textheight=22cm \topmargin=0.5cm \oddsidemargin=0.5cm \evensidemargin=0.5cm

\newcommand{\sk}{\vskip 10mm}
\newcommand{\bb}[1]{\mathbb{#1}}
\newcommand{\ra}{\rightarrow}

\theoremstyle{plain}
\newtheorem{problem}{Problem}
\newtheorem{lemma}{Lemma}[problem]

\theoremstyle{remark}
\newtheorem{tpart}{}[problem]
\newtheorem*{ppart}{}

\begin{document}

\begin{problem}[1.1.21] 
  
\end{problem}

\begin{proof}
  Let $x\in G$ have order odd $2n+1$. Then $x^{2n+1}=e$ and if we multiply by $x$
  on both sides we get $x^{2n+2}=x^{2(n+1)}=x$.

  Therefore if $x$ has odd order then it $x^{2k}=x$ for some $k\in\bb{Z}_{>0}$.
\end{proof}

\sk

\begin{problem}[1.1.25] 
  
\end{problem}

\begin{proof}
  Suppose that $G$ is a group such that for all $x\in G$ that $x^2=1$.
  This implies that for any element $x^{-1}=x$. Then we have
  $xy=(xy)^{-1}=y^{-1}x^{-1}=yx$
  which shows that $G$ is commutative.
\end{proof}

\sk

\begin{problem}[1.1.35] 
  
\end{problem}

\begin{proof}
  Let $x\in G$ have order $n$. Then consider $x^m$. Using the division algorithm
  with $m$ and $n$ we can rewrite $m$ as $qn+d$ where $0\leq d<n$. This implies
  that
  \[ x^m=x^{qn+d}=x^{qn}x^d=(x^n)^qx^d=e^qx^d=x^d \]
  As such $x\in\{e,x^1,\ldots,x^{n-1}\}$.
\end{proof}

\sk

\begin{problem}[1.3.6] 
  
\end{problem}

\begin{proof}
  
\end{proof}

\sk

\begin{problem}[1.3.9] 
  
\end{problem}

\begin{proof}
  
\end{proof}

\sk

\begin{problem}[1.3.13] 
  
\end{problem}

\begin{proof}
  
\end{proof}

\sk

\begin{problem}[1.5.2] 
  
\end{problem}

\begin{proof}
  
\end{proof}

\sk

%%%%%%%%%%%%%%%%%%%%%%%%%%%%%%%%%%%%%%%%%%%%%%%%%%%%%%%%%%%%%%%%%%%%%%%%%%%%%
\end{document}
